\documentclass[a4]{article}

\usepackage{amsmath}
\usepackage{amssymb}
\usepackage{framed}
\usepackage{mathrsfs}

\usepackage[left = 1cm,right = 1cm, top = 2cm]{geometry}

\begin{document}

    \title{Quantization of the Free Electromagnetic Field In the Coulomb Gauge}
    \maketitle

    This entire calculation is done in natural units. The Lagrangian Density of the free Electromagnetic Field is
    as follows:

    \begin{equation}
        \mathcal{L} = - \frac{1}{4} F_{\mu \nu} F^{\mu \nu}
    \end{equation}

    Where:

    \begin{equation}
        F_{\mu \nu} = \partial_{\mu} A_{\nu} - \partial_{\nu} A_{\nu}
    \end{equation}

    Because $A_{\mu}$ is a real vector field, and there is no mass term, we are dealing with a massless spin
    one chargeless field theory. Now let's consider gauge transformations:

    \begin{equation}
        A_{\mu} \rightarrow A_{\mu} + \partial_{\mu} \Lambda
    \end{equation}

    his gauge invariance will be important later. We will now try and quantize the theory as is. It turns out
    that very quickly we will now run into problems. The first step is of course to calculate the conjugate field:

    \begin{equation}
        \pi^{\mu} = \frac{\partial \mathcal{L}}{\partial (\partial_{0} A_{\mu})} = - \frac{1}{4} 2 F^{\alpha \beta} \frac{\partial F^{\alpha \beta}}{\partial (\partial_{0} A_{\mu})} = - F^{0 \mu} = F^{\mu 0}
    \end{equation}

    \begin{equation}
        \pi^{0} = 0 \qquad \pi^{i} = E^{i}
    \end{equation}

    Because $\pi^{0} = 0$, the cannonical equal-time commutation relations cannot possibly be satisfied here because they would demand that the commutator of $\phi$ with 0
    is some nonzero number, which is impossible. It turns out that one can solve this problem by breaking the gauge. There are many specific prescriptions for fixing this
    problem, because ther aremany different ways of breaking the gauge. In this case, we will use the Coulomb gauge (vector fields satisfying the Coulomb gauge are called
    transverse fields):

    \begin{equation}
        \nabla \cdot \vec{A} = 0 \qquad \phi = 0
    \end{equation}

    Where $\phi = 0$ follows from $\nabla \cdot \vec{A} = 0$. If appropriate boundary conditions are adhered to. Namely, if one demands that $\phi$ vanishes at infinity, then
    $\nabla \cdot \vec{A} = 0$ implies that $\phi = 0$. It is the fact that $A_{0} = \phi = 0$ that rescues the calculation when using the Coulomb Gauge. Specifically, the
    vanishing of the conjugate field to $\phi$ no longer matters because we can ignre the equal time commutation relation of $\phi$ and $\pi^{0}$. In this gauge, one might want
    to try imposing the following commutation relations:

    \begin{equation}
        [A_{i} (\vec{x}, t), A_{j} (\vec{x}, t)] = [E^{i} (\vec{x}, t), E^{j} (\vec{x}, t)] = 0
    \end{equation}

    \begin{equation}
        \overbrace{[E^{i} (\vec{x}, t), E^{j} (\vec{x}, t)] = i \delta^{j}_{i} \delta^{3} (\vec{x} - \vec{x'})}^{wrong}
    \end{equation}

    There, however is a problem. We can see that even after the fields and their Fourier coefficients have become operators, one can see that these relations $\nabla \cdot \vec{A}
    \nabla \cdot \vec{E} = 0$ are still preserved. The problem is that the last commutation relation above is inconsistent with these. Specificaly, if we take $\nabla \cdot \vec{A}
    \nabla \cdot \vec{E} = 0$ to be true, then it implies the following nonsense:

    \begin{equation}
        0 = [\nabla \cdot \vec{A} (\vec{x}, t), \nabla \cdot \vec{E} (\vec{x}, t)] = i \nabla^{2} \delta^{3} (\vec{x} \vec{x'}) \neq 0
    \end{equation}

    Clearly, the fix to this is to select the following commutator:

    \begin{equation}
        [E^{i} (\vec{x}, t), E^{j} (\vec{x}, t)] = i \bigg( \delta^{i}_{j} - \frac{\partial_{i} \partial^{j}}{\nabla^{2}} \bigg) \delta^{3} (\vec{x} \vec{x'})
    \end{equation}

    \begin{equation}
        [E^{i} (\vec{x}, t), E^{j} (\vec{x}, t)] = i \int \bigg( \delta^{i}_{j} - \frac{p_{i} p^{j}}{\nabla^{2}} \bigg) e^{i \vec{k} \cdot (\vec{x} - \vec{x'})} \frac{d^{3} k}{(2 \pi)^3}
    \end{equation}


    So, the complete set of correct anticommutation relations is:

    \begin{framed}
        \begin{equation}
            \begin{aligned}
                \quad [\vec{A_{i}} (\vec{x}, t), \vec{A_{j}} (\vec{x}, t)] = [\vec{E^{i}} (\vec{x}, t), \vec{E^{j}} (\vec{x}, t)] = 0 \\
                \quad [\vec{A_{i}} (\vec{x}, t), \vec{A_{j}} (\vec{x}, t)] = i \int \bigg( \delta^{i}_{j} - \frac{p_{i} p^{j}}{\nabla^{2}} \bigg) e^{i \vec{k} \cdot (\vec{x} - \vec{x'})} \frac{d^{3} k}{(2 \pi)^3}
            \end{aligned}
        \end{equation}
    \end{framed}

    Also, in this gauge, the equations of motion become:

    \begin{equation}
        \partial_{\mu} F^{\mu \nu} = 0
    \end{equation}

    \begin{equation}
        \partial_{\mu} (\partial_{\mu} A_{\nu} - \partial_{\nu} A_{\mu}) = 0
    \end{equation}

    \begin{equation}
        \partial_{\mu} (\partial_{\mu} A_{0} - \partial_{0} A_{\mu}) = 0 \quad \partial_{\mu} (\partial^{\mu} \vec{A} + \nabla A^{\mu}) = 0
    \end{equation}

    \begin{equation}
        \nabla^{2} + \frac{\partial}{\partial t} \nabla \cdot \vec{A} = 0 \qquad \frac{\partial^2}{\partial t^2} \vec{A} + \frac{\partial}{\partial t} \nabla \phi - \nabla^{2} \vec{A} + \nabla (\nabla \cdot \vec{A}) = 0
    \end{equation}

    \begin{equation}
        \nabla \cdot \vec{A} = 0 \qquad \phi = 0
    \end{equation}

    \begin{equation}
        0 = 0 \qquad \frac{\partial^2}{\partial t^2} \vec{A} - \nabla^{2} \vec{A} = 0
    \end{equation}

    \begin{framed}
        \begin{equation}
            \square \vec{A} = 0
        \end{equation}
    \end{framed}

    The plane wave solutions therefore take the following form:

    \begin{equation}
        \vec{A} = \vec{\epsilon} e^{-i (\vec{k} \cdot \vec{x} - \omega t)}
    \end{equation}

    However, given that this is in the Coulomb gauge, this field must be transverse, so:

    \begin{equation}
        \nabla \cdot \vec{A} = \vec{k} \cdot \vec{\epsilon} e^{-i (\vec{k} \cdot \vec{x} - \omega t)} = 0
    \end{equation}

    There are two linearly independent transverse polarizations. Therefore, there are two polarization possibilities
    for the field quanta (which, as you probably already know, are called photons). A subscript that can have two
    values added to the polarization vector to index this duplicity:

    \begin{equation}
        \vec{\epsilon} \rightarrow \vec{\epsilon}_{\lambda}
    \end{equation}

    For simplicity, the polarization vector is usually taken to be the unit norm, and they are usually taken to be
    orthogonal to each other:

    \begin{equation}
        \vec{\epsilon}_{\lambda} \cdot \vec{\epsilon}_{\lambda} = \delta_{\lambda \lambda'}
    \end{equation}

    One can then write the general solution by constructing an arbitrary linear combination of them (that transforms like
    a 3-vector):

    \begin{equation}
        A_{i} = \sum_{\lambda} \int - i \omega \bigg( \epsilon_{\lambda i} a_{\lambda} (\vec{k}) e^{- i (\omega t - \vec{k} \cdot \vec{x})} + \epsilon_{\lambda i} a_{\lambda} (\vec{k}) e^{i (\omega t - \vec{k} \cdot \vec{x})} \bigg) \frac{d^{3} k}{(2 \pi)^{3} 2 \omega}
    \end{equation}

    Where the Fourier coefficients are Hermitian conjugates of each other to ensure that the field is real. We can also
    compute for the conjugate Fields:

    \begin{equation}
        \pi^{i} = E^{i} = - \frac{\partial}{\partial t} A_{i} = \sum_{\lambda} \int - i \omega \bigg( \epsilon_{\lambda i} a_{\lambda} (\vec{k}) e^{- i (\omega t - \vec{k} \cdot \vec{x})} + \epsilon_{\lambda i} a_{\lambda} (\vec{k}) e^{i (\omega t - \vec{k} \cdot \vec{x})} \bigg) \frac{d^{3} k}{(2 \pi)^{3} 2 \omega}
    \end{equation}

    With these results established, we can now invert them

    \begin{equation}
        \begin{aligned}
            \int e^{i (\omega' t - \vec{k}' \cdot \vec{x})} A_{i} (\vec{x}, t) d^{3} x = & \\
            & \sum_{\lambda} \int \frac{d^{3} k}{(2 \pi)^{3} 2 \omega} - i \omega \bigg( \epsilon_{\lambda i} a_{\lambda} (\vec{k}) e^{- i (\omega t - \vec{k} \cdot \vec{x})} + \epsilon_{\lambda i} a_{\lambda} (\vec{k}) e^{i (\omega t - \vec{k} \cdot \vec{x})} \bigg)
        \end{aligned}
    \end{equation}

    \begin{equation}
        \int e^{i (\vec{k}' - \vec{k}) \cdot x} \frac{d^{3} x}{(2 \pi)^{3}} = \delta^{3} (\vec{k}' - \vec{k})
    \end{equation}

    \begin{equation}
        \int e^{i (\vec{k}' + \vec{k}) \cdot x} \frac{d^{3} x}{(2 \pi)^{3}} = \delta^{3} (\vec{k}' + \vec{k})
    \end{equation}

    \begin{equation}
        \begin{aligned}
            \int e^{i (\omega' t - \vec{k}' \cdot \vec{x})} A_{i} d^{3} x = & \\
            & \sum_{\lambda} \int \frac{d^{3} k}{(2 \pi)^{3} 2 \omega} - i \omega \bigg( \epsilon_{\lambda i} a_{\lambda} (\vec{k}) e^{- i (\omega t - \vec{k} \cdot \vec{x})} + \epsilon_{\lambda i} a_{\lambda} (\vec{k}) e^{i (\omega t - \vec{k} \cdot \vec{x})} \bigg)
        \end{aligned}
    \end{equation}

    \begin{equation}
        \begin{aligned}
            \int e^{i (\omega' t - \vec{k}' \cdot \vec{x})} A_{i} (\vec{x}, t) d^{3} x = & \\
            & \sum_{\lambda} \int \frac{d^{3} k}{(2 \pi)^{3} 2 \omega} - i \omega \bigg( \epsilon_{\lambda i} a_{\lambda} (\vec{k}) e^{- i (\omega t - \vec{k} \cdot \vec{x})} + \epsilon_{\lambda i} a_{\lambda} (\vec{k}) e^{i (\omega t - \vec{k} \cdot \vec{x})} \bigg) \\
            & \sum_{\lambda} \frac{1}{2 \omega'} [\epsilon_{\lambda i} a_{\lambda} (\vec{k}) + \epsilon_{\lambda i} a_{\lambda} (\vec{k}) e^{i2 \omega' t}]
        \end{aligned}
    \end{equation}

    \begin{equation}
        \int e^{i (\omega' t - \vec{k}' \cdot \vec{x})} \vec{\epsilon}_{\lambda'} \cdot A_{i} (\vec{x}, t) d^3 x = \sum_{\lambda} \frac{1}{2 \omega'} [\vec{\epsilon}_{\lambda'} \cdot \epsilon_{\lambda} a_{\lambda} (\vec{k}) + \vec{\epsilon}_{\lambda'} \cdot \epsilon_{\lambda} a_{\lambda} (\vec{k}) e^{i2 \omega' t}]
    \end{equation}

    \begin{equation}
        \int e^{i (\omega' t - \vec{k}' \cdot \vec{x})} \vec{\epsilon}_{\lambda'} \cdot A_{i} (\vec{x}, t) d^3 x = \sum_{\lambda} \frac{1}{2 \omega'} [\delta_{\lambda \lambda'} a_{\lambda} (\vec{k}) + \delta_{\lambda \lambda'} a_{\lambda} (\vec{k}) e^{i2 \omega' t}]
    \end{equation}

    \begin{equation}
        \int e^{i (\omega' t - \vec{k}' \cdot \vec{x})} \vec{\epsilon}_{\lambda'} \cdot A_{i} (\vec{x}, t) d^3 x = \frac{1}{2 \omega'} [a_{\lambda'} (\vec{k}) + a_{\lambda'} (\vec{k}) e^{i2 \omega' t}]
    \end{equation}
    
    The primes can now be dropped because the integrals and sums are all done. The final result for this integral is:

    \begin{framed}
        \begin{equation}
            \int e^{i (\omega' t - \vec{k}' \cdot \vec{x})} \vec{\epsilon}_{\lambda'} \cdot A_{i} (\vec{x}, t) = [a_{\lambda'} (\vec{k}) + a_{\lambda'} (\vec{k}) e^{i2 \omega' t}]
        \end{equation}
    \end{framed}

    Now for the second one:

    \begin{equation} % Double Check, timestamp = 9:34
        \begin{aligned}
            \int e^{i (\omega' t - \vec{k}' \cdot \vec{x})} \pi_{i} (\vec{x}, t) d^{3} x = & \\
            & \sum_{\lambda} \int \frac{d^{3} k}{(2 \pi)^{3} 2 \omega} - i \omega \bigg( \epsilon_{\lambda i} a_{\lambda} (\vec{k}) e^{- i (\omega t - \vec{k} \cdot \vec{x})} + \epsilon_{\lambda i} a_{\lambda} (\vec{k}) e^{i (\omega t - \vec{k} \cdot \vec{x})} \bigg)
        \end{aligned}
    \end{equation}

    \begin{equation} % Double Check, timestamp = 9:34
        \int e^{i (\vec{k}' - \vec{k}) \cdot x} \frac{d^{3} x}{(2 \pi)^{3}} = \delta^{3} (\vec{k}' - \vec{k})
    \end{equation}

    \begin{equation} % Double Check, timestamp = 9:34
        \int e^{i (\vec{k}' + \vec{k}) \cdot x} \frac{d^{3} x}{(2 \pi)^{3}} = \delta^{3} (\vec{k}' + \vec{k})
    \end{equation}

    \begin{equation} % Double Check, timestamp = 9:34
        \begin{aligned}
            \int e^{i (\omega' t - \vec{k}' \cdot \vec{x})} \pi_{i} d^{3} x = & \\
            & \sum_{\lambda} \int \frac{d^{3} k}{(2 \pi)^{3} 2 \omega} - i \omega \bigg( \epsilon_{\lambda i} a_{\lambda} (\vec{k}) e^{- i (\omega t - \vec{k} \cdot \vec{x})} + \epsilon_{\lambda i} a_{\lambda} (\vec{k}) e^{i (\omega t - \vec{k} \cdot \vec{x})} \bigg)
        \end{aligned}
    \end{equation}

    \begin{equation} % Double Check, timestamp = 9:34
        \begin{aligned}
            \int e^{i (\omega' t - \vec{k}' \cdot \vec{x})} \pi_{i} (\vec{x}, t) d^{3} x = & \\
            & \sum_{\lambda} \int \frac{d^{3} k}{(2 \pi)^{3} 2 \omega} - i \omega \bigg( \epsilon_{\lambda i} a_{\lambda} (\vec{k}) e^{- i (\omega t - \vec{k} \cdot \vec{x})} + \epsilon_{\lambda i} a_{\lambda} (\vec{k}) e^{i (\omega t - \vec{k} \cdot \vec{x})} \bigg) \\
            & \sum_{\lambda} \frac{1}{2 \omega'} [\epsilon_{\lambda i} a_{\lambda} (\vec{k}) + \epsilon_{\lambda i} a_{\lambda} (\vec{k}) e^{i2 \omega' t}]
        \end{aligned}
    \end{equation}

    \begin{equation} % Double Check, timestamp = 9:34
        \int e^{i (\omega' t - \vec{k}' \cdot \vec{x})} \vec{\epsilon}_{\lambda'} \cdot \pi_{i} (\vec{x}, t) d^3 x = \sum_{\lambda} \frac{1}{2 \omega'} [\epsilon_{\lambda}^{i} a_{\lambda} (\vec{k}) + \epsilon_{\lambda}^{i} a_{\lambda} (\vec{k}) e^{i2 \omega' t}]
    \end{equation}

    \begin{equation} % Double Check, timestamp = 9:34
        \int e^{i (\omega' t - \vec{k}' \cdot \vec{x})} \vec{\epsilon}_{\lambda'} \cdot \pi_{i} (\vec{x}, t) d^3 x = \sum_{\lambda} \frac{1}{2 \omega'} [\vec{\epsilon}_{\lambda'} \cdot \epsilon_{\lambda} a_{\lambda} (\vec{k}) + \vec{\epsilon}_{\lambda'} \cdot \epsilon_{\lambda} a_{\lambda} (\vec{k}) e^{i2 \omega' t}]
    \end{equation}

    \begin{equation} % Double Check, timestamp = 9:34
        \int e^{i (\omega' t - \vec{k}' \cdot \vec{x})} \vec{\epsilon}_{\lambda'} \cdot \pi_{i} (\vec{x}, t) d^3 x = \sum_{\lambda} \frac{1}{2 \omega'} [\delta_{\lambda \lambda'} a_{\lambda} (\vec{k}) + \delta_{\lambda \lambda'} a_{\lambda} (\vec{k}) e^{i2 \omega' t}]
    \end{equation}

    \begin{equation} % Double Check, timestamp = 9:34
        \int e^{i (\omega' t - \vec{k}' \cdot \vec{x})} \vec{\epsilon}_{\lambda'} \cdot \pi_{i} (\vec{x}, t) d^3 x = \frac{1}{2 \omega'} [a_{\lambda'} (\vec{k}) + a_{\lambda'} (\vec{k}) e^{i2 \omega' t}]
    \end{equation}

    The primes can now be dropped because the integrals and sums are all done. The final result for this integral is:

    \begin{framed}
        \begin{equation}
            \int e^{i (\omega' t - \vec{k}' \cdot \vec{x})} \vec{\epsilon}_{\lambda'} \cdot \pi_{i} (\vec{x}, t) = [a_{\lambda'} (\vec{k}) + a_{\lambda'} (\vec{k}) e^{i 2 \omega' t}]
        \end{equation}
    \end{framed}

    Collecting the two integrals we just found in one place, we have:

    \begin{framed}
        \begin{equation}
            \begin{aligned}
                \int e^{i (\omega' t - \vec{k}' \cdot \vec{x})} \vec{\epsilon}_{\lambda'} \cdot A_{i} (\vec{x}, t) = \frac{1}{2 \omega'} [\epsilon_{\lambda i} a_{\lambda} (\vec{k}) + \epsilon_{\lambda i} a_{\lambda} (\vec{k}) e^{i2 \omega' t}] \\
                \int e^{i (\omega' t - \vec{k}' \cdot \vec{x})} \vec{\epsilon}_{\lambda'} \cdot \pi_{i} (\vec{x}, t) = \frac{1}{2 \omega'} [\epsilon_{\lambda i} a_{\lambda} (\vec{k}) + \epsilon_{\lambda i} a_{\lambda} (\vec{k}) e^{i2 \omega' t}]
            \end{aligned}
        \end{equation}
    \end{framed}

    With these, we can find the desired inverted relations:

    \begin{equation}
        \int e^{i (\omega' t - \vec{k}' \cdot \vec{x})} \vec{\epsilon}_{\lambda'} \cdot A_{i} (\vec{x}, t) + i \int e^{i (\omega' t - \vec{k}' \cdot \vec{x})} \vec{\epsilon}_{\lambda'} \cdot A_{i} (\vec{x}, t) = a_{\lambda} (\vec{k})
    \end{equation}

    \begin{framed}
        \begin{equation}
            a_{\lambda} (\vec{k}) = \int e^{- i (\omega t - \vec{k} \cdot \vec{x})} [i \vec{\epsilon}_{\lambda} + \cdot \vec{\pi} (\vec{x}, t) + \omega \vec{\epsilon}_{\lambda} \cdot \vec{A} (\vec{x}, t)] d^{3} x
        \end{equation}
    \end{framed}

    Hermitian conjugation then gives the other one:

    \begin{framed}
        \begin{equation}
            a_{\lambda}^{\dagger} (\vec{k}) = \int e^{i (\omega t - \vec{k} \cdot \vec{x})} [- i \vec{\epsilon}_{\lambda} + \cdot \vec{\pi} (\vec{x}, t) + \omega \vec{\epsilon}_{\lambda} \cdot \vec{A} (\vec{x}, t)] d^{3} x
        \end{equation}
    \end{framed}

    Now we can start working out the commutation relations for the Fourier coefficients. Let's start with
    $[a_{\lambda} (\vec{k}), a^{k}_{\lambda} (\vec{k})]$:

    \begin{equation}
        \begin{aligned} []
            [a_{\lambda} (\vec{k}), a_{\lambda'}^{\dagger} (\vec{k'})] = \\
            \int e^{i (k-k') \cdot x} [(i \vec{\epsilon}_{\lambda} + \cdot \vec{\pi} (\vec{x}, t) + \omega \vec{\epsilon}_{\lambda} \cdot \vec{A} (\vec{x}, t))(- i \vec{\epsilon}_{\lambda} + \cdot \vec{\pi} (\vec{x}, t) + \omega \vec{\epsilon}_{\lambda} \cdot \vec{A} (\vec{x}, t)) \\
            (- i \vec{\epsilon}_{\lambda} + \cdot \vec{\pi} (\vec{x}, t) + \omega \vec{\epsilon}_{\lambda} \cdot \vec{A} (\vec{x}, t))(i \vec{\epsilon}_{\lambda} + \cdot \vec{\pi} (\vec{x}, t) + \omega \vec{\epsilon}_{\lambda} \cdot \vec{A} (\vec{x}, t))] d^{3} x d^{3} x'
        \end{aligned}
    \end{equation}

    Where $k \cdot x = k_{\mu} x^{\mu}$ and $t = t'$. Now applying $[A_{i} (\vec{x}, t), A_{j} (\vec{x}, t)] = 0$ and 
    $[E^{i} (\vec{x}, t), E^{j} (\vec{x}, t)] = 0$:

    \begin{equation}
        \begin{aligned} []
            [a_{\lambda} (\vec{k}), a_{\lambda'}^{\dagger} (\vec{k'})] = \\
            \int e^{i (k-k') \cdot x} i \omega [\epsilon_{\lambda i} \epsilon_{\lambda'}^{j} [\pi^{i} (\vec{x}, t), A_{j} (\vec{x}, t)] - \epsilon_{\lambda}^{j} \epsilon_{\lambda' i} [A_{j} (\vec{x}, t), \pi^{i} (\vec{x}, t)]] d^{3} x d^{3} x'
        \end{aligned}
    \end{equation}

    Now applying $[A_{i} (\vec{x}, t), E^{j} (\vec{x}, t)] = i \int (\delta_{i}^{j} - \frac{k_i k^j}{k^2}) e^{i \vec{k} \cdot (\vec{x} - \vec{x'})} \frac{d^3 k}{(2 \pi)^3}$:

    \begin{equation}
        \begin{aligned} []
            [a_{\lambda} (\vec{k}), a_{\lambda'}^{\dagger} (\vec{k'})] = \\
            \int e^{i (k-k') \cdot x} i \omega [\epsilon_{\lambda i} \epsilon_{\lambda'}^{j} i (\delta_{i}^{j} - \frac{k_i k^j}{k^2}) e^{i \vec{k} \cdot (\vec{x} - \vec{x'})} - \epsilon_{\lambda}^{j} \epsilon_{\lambda' i} i (\delta_{i}^{j} - \frac{k_i k^j}{k^2}) e^{i \vec{k} \cdot (\vec{x} - \vec{x'})}] \frac{d^3 p}{(2 \pi)^3} d^{3} x d^{3} x'
        \end{aligned}
    \end{equation}

    Remembering the transversality of the polarization:

    \begin{equation}
        [a_{\lambda} (\vec{k}), a_{\lambda'}^{\dagger} (\vec{k'})] = \int e^{i (k  \cdot x -k'  \cdot x')} 2 \omega \vec{\epsilon} \cdot \vec{\epsilon'} e^{i (p-p') \cdot x} \frac{d^3 p}{(2 \pi)^3} d^{3} x d^{3} x'
    \end{equation}

    Now, if we recall $\vec{\epsilon}_{\lambda} \cdot \vec{\epsilon}_{\lambda} = \delta_{\lambda \lambda'}$

    \begin{equation}
        [a_{\lambda} (\vec{k}), a_{\lambda'}^{\dagger} (\vec{k'})] = \int e^{i (k  \cdot x -k'  \cdot x')} 2 \omega \delta_{\lambda \lambda'} e^{i (p-p') \cdot x} \frac{d^3 p}{(2 \pi)^3} d^{3} x d^{3} x'
    \end{equation}

    \begin{equation}
        [a_{\lambda} (\vec{k}), a_{\lambda'}^{\dagger} (\vec{k'})] = \int e^{i (k  \cdot x -k'  \cdot x')} 2 \omega \delta_{\lambda \lambda'} \delta^{3} (\vec{x} - \vec{x'}) d^{3} x d^{3} x'
    \end{equation}
    
    \begin{equation}
        [a_{\lambda} (\vec{k}), a_{\lambda'}^{\dagger} (\vec{k'})] = \int e^{i (k-k') \cdot x} 2 \omega \delta_{\lambda \lambda'} d^{3} x d^{3} x'
    \end{equation}

    All the other Fourier coefficient commutators vanish trivially. So, the complete set are as folllows:

    \begin{framed}
        \begin{equation}
            \begin{aligned} []
                [a_{\lambda} (\vec{k}), a_{\lambda'} (\vec{k'})] = [a_{\lambda}^{\dagger} (\vec{k}), a_{\lambda'}^{\dagger} (\vec{k'})] = 0 \\
                [a_{\lambda} (\vec{k}), a_{\lambda'}^{\dagger} (\vec{k'})] = (2 \pi)^{3} 2 \omega \delta_{\lambda \lambda'} \delta^{3} (\vec{k} - \vec{k'})
            \end{aligned}
        \end{equation}
    \end{framed}

    The Hamiltonian is:

    \begin{equation}
        H = \int d^{3} x (\pi^{i} \dot{A}_{i} - \mathcal{L}) = \frac{1}{2} \int (\vec{E} \cdot \vec{E} + \vec{B} \cdot \vec{B})
    \end{equation}

    We have the following values for the fields:

    \begin{equation}
        E^{j} = \sum_{\lambda} \int - i \omega (\vec{\epsilon}_{\lambda}^{j} a^{\dagger}_{\lambda} (\vec{k}) a_{\lambda'} (\vec{k}) e^{2 i \omega t} - \vec{\epsilon}_{\lambda}^{j} a_{\lambda} (\vec{k}) a^{\dagger}_{\lambda'} (\vec{k}) e^{- 2 i \omega t}) \frac{d^{3} k}{(2 \pi)^{3} 2 \omega}
    \end{equation}

    \begin{equation}
        B^{j} = \sum_{\lambda} \int - i (\vec{k} \times \vec{\epsilon}_{\lambda})^{j} (a^{\dagger}_{\lambda} (\vec{k}) a_{\lambda'} (\vec{k}) e^{2 i \omega t} + a_{\lambda} (\vec{k}) a^{\dagger}_{\lambda'} (\vec{k}) e^{- 2 i \omega t}) \frac{d^{3} k}{(2 \pi)^{3} 2 \omega}
    \end{equation}

    Inserting these gives:

    \begin{equation}
        \begin{aligned}
            H = - \frac{1}{2} \sum_{\lambda \lambda'} \int (\vec{\epsilon}_{\lambda}^{j} a^{\dagger}_{\lambda} (\vec{k}) a_{\lambda'} (\vec{k}) e^{2 i \omega t} - \vec{\epsilon}_{\lambda}^{j} a_{\lambda} (\vec{k}) a^{\dagger}_{\lambda'} (\vec{k}) e^{- 2 i \omega t}) \\
            + (\vec{\epsilon}_{\lambda}^{j} a^{\dagger}_{\lambda} (\vec{k}) a_{\lambda'} (\vec{k}) e^{2 i \omega t} - \vec{\epsilon}_{\lambda}^{j} a_{\lambda} (\vec{k}) a^{\dagger}_{\lambda'} (\vec{k}) e^{- 2 i \omega t}) \\
            + (\vec{k} \times \vec{\epsilon}_{\lambda})^{j} (a^{\dagger}_{\lambda} (\vec{k}) a_{\lambda'} (\vec{k}) e^{2 i \omega t} + a_{\lambda} (\vec{k}) a^{\dagger}_{\lambda'} (\vec{k}) e^{- 2 i \omega t}) \\
            + (\vec{k} \times \vec{\epsilon}_{\lambda})^{j} (a^{\dagger}_{\lambda} (\vec{k}) a_{\lambda'} (\vec{k}) e^{2 i \omega t} + a_{\lambda} (\vec{k}) a^{\dagger}_{\lambda'} (\vec{k}) e^{- 2 i \omega t}) 
            \frac{d^{3} k}{(2 \pi)^{3} 2 \omega} \frac{d^3 k'}{(2 \pi)^3}
        \end{aligned}
    \end{equation}

    \begin{equation}
        \begin{aligned}
            H = - \frac{1}{2} \sum_{\lambda \lambda'} \int (\vec{\epsilon}_{\lambda}^{j} a^{\dagger}_{\lambda} (\vec{k}) a_{\lambda'} (\vec{k}) e^{2 i \omega t} - \vec{\epsilon}_{\lambda}^{j} a_{\lambda} (\vec{k}) a^{\dagger}_{\lambda'} (\vec{k}) e^{- 2 i \omega t}) \\
            + (\vec{\epsilon}_{\lambda}^{j} a^{\dagger}_{\lambda} (\vec{k}) a_{\lambda'} (\vec{k}) e^{2 i \omega t} - \vec{\epsilon}_{\lambda}^{j} a_{\lambda} (\vec{k}) a^{\dagger}_{\lambda'} (\vec{k}) e^{- 2 i \omega t}) \\
            + (\vec{k} \times \vec{\epsilon}_{\lambda})^{j} (a^{\dagger}_{\lambda} (\vec{k}) a_{\lambda'} (\vec{k}) e^{2 i \omega t} + a_{\lambda} (\vec{k}) a^{\dagger}_{\lambda'} (\vec{k}) e^{- 2 i \omega t}) \\
            + (\vec{k} \times \vec{\epsilon}_{\lambda})^{j} (a^{\dagger}_{\lambda} (\vec{k}) a_{\lambda'} (\vec{k}) e^{2 i \omega t} + a_{\lambda} (\vec{k}) a^{\dagger}_{\lambda'} (\vec{k}) e^{- 2 i \omega t}) 
            \frac{d^{3} k}{(2 \pi)^{3} 2 \omega} \frac{d^3 k'}{(2 \pi)^3}
        \end{aligned}
    \end{equation}

    Now the integration over $x$ can be done:

    \begin{equation}
        \begin{aligned}
            H = -\frac{1}{2} \sum_{\lambda \lambda'} \int \Big( \omega \omega' \vec{\epsilon}_{\lambda} \cdot \vec{\epsilon}_{\lambda'} \big(a_{\lambda} (\vec{k}) a_{\lambda'} (\vec{k}) e^{- i (\omega + \omega')} (2 \pi)^3 \delta^{3} (\vec{k} + \vec{k'}) \\
            - a_{\lambda}^{\dagger} (\vec{k}) a_{\lambda} (\vec{k}) e^{i (\omega - \omega')} \delta^{3} (\vec{k} - \vec{k'}) - a_{\lambda} (\vec{k}) a_{\lambda}^{\dagger}  (\vec{k}) e^{- i (\omega - \omega')} (2 \pi)^3 \delta^{3} (\vec{k} - \vec{k'}) \\
            + a_{\lambda}^{\dagger} (\vec{k}) a_{\lambda'}^{\dagger} (\vec{-k}) e^{i (\omega + \omega')} (2 \pi)^3 \delta^{3} (\vec{k} + \vec{k'}) \big) \\
            + (\vec{k} \times \vec{\epsilon}_{\lambda}) \cdot (\vec{k} \times \vec{\epsilon}_{\lambda'}) \big(a_{\lambda} (\vec{k}) a_{\lambda'} (\vec{k}) e^{- i (\omega + \omega')} (2 \pi)^3 \delta^{3} (\vec{k} + \vec{k'}) \\
            - a_{\lambda}^{\dagger} (\vec{k}) a_{\lambda} (\vec{k}) e^{i (\omega - \omega')} (2 \pi)^3 \delta^{3} (\vec{k} - \vec{k'}) \\
            - a_{\lambda} (\vec{k}) a_{\lambda'}^{\dagger} (\vec{k}) e^{- i (\omega - \omega')} (2 \pi)^3 \delta^{3} (\vec{k} - \vec{k'}) \\
            - a_{\lambda}^{\dagger} (\vec{k}) a_{\lambda'}^{\dagger} (\vec{k}) e^{i (\omega + \omega')} (2 \pi)^3 \delta^{3} (\vec{k} + \vec{k'}) \big) \Big) \frac{1}{2 \omega} \frac{d^{3} k}{(2 \pi)^{3} 2 \omega}
        \end{aligned}
    \end{equation}

    Now, we can simplify, partly using the properties of the delta function to simplify the remaining phase factors:

    \begin{equation}
        \begin{aligned}
            H = -\frac{1}{2} \sum_{\lambda \lambda'} \int \Big( \omega \omega' \vec{\epsilon}_{\lambda} \cdot \vec{\epsilon}_{\lambda'} \big(a_{\lambda} (\vec{k}) a_{\lambda'} (\vec{k}) e^{-2 i \omega t} \delta^{3} (\vec{k} + \vec{k'}) - a_{\lambda}^{\dagger} (\vec{k}) a_{\lambda} (\vec{k}) \delta^{3} (\vec{k} - \vec{k'}) - a_{\lambda} (\vec{k}) a_{\lambda}^{\dagger}  (\vec{k}) \delta^{3} (\vec{k} - \vec{k'}) \\
            + a_{\lambda}^{\dagger} (\vec{k}) a_{\lambda'}^{\dagger} (\vec{-k}) e^{2 i \omega t} \delta^{3} (\vec{k} + \vec{k'}) \big) + (\vec{k} \times \vec{\epsilon}_{\lambda}) \cdot (\vec{k} \times \vec{\epsilon}_{\lambda'}) \big(a_{\lambda} (\vec{k}) a_{\lambda'} (\vec{k}) e^{-2 i \omega t} \delta^{3} (\vec{k} + \vec{k'}) \\
            - a_{\lambda}^{\dagger} (\vec{k}) a_{\lambda} (\vec{k}) \delta^{3} (\vec{k} - \vec{k'}) - a_{\lambda} (\vec{k}) a_{\lambda'}^{\dagger} (\vec{k}) \delta^{3} (\vec{k} - \vec{k'}) - a_{\lambda}^{\dagger} (\vec{k}) a_{\lambda'}^{\dagger} (\vec{k}) e^{-2 i \omega t} \delta^{3} (\vec{k} + \vec{k'}) \big) \Big) \frac{1}{2 \omega} \frac{d^{3} k}{(2 \pi)^{3} 2 \omega}
        \end{aligned}
    \end{equation}

    Now, the $k'$ integration can be done using all of the delta functions:

    \begin{equation}
        \begin{aligned}
            H = -\frac{1}{2} \sum_{\lambda \lambda'} \int \Big( \omega^2 \vec{\epsilon}_{\lambda} \cdot \vec{\epsilon}_{\lambda'} \big(a_{\lambda} (\vec{k}) a_{\lambda'} (\vec{k}) e^{-2 i \omega t} - a_{\lambda}^{\dagger} (\vec{k}) a_{\lambda} (\vec{k}) - a_{\lambda} (\vec{k}) a_{\lambda}^{\dagger}  (\vec{k})  \\
            + a_{\lambda}^{\dagger} (\vec{k}) a_{\lambda'}^{\dagger} (\vec{-k}) e^{2 i \omega t} \big) + (\vec{k} \times \vec{\epsilon}_{\lambda}) \cdot (\vec{k} \times \vec{\epsilon}_{\lambda'}) \big(a_{\lambda} (\vec{k}) a_{\lambda'} (\vec{k}) e^{-2 i \omega t} \\
            - a_{\lambda}^{\dagger} (\vec{k}) a_{\lambda} (\vec{k}) \delta^{3} (\vec{k} - \vec{k'}) - a_{\lambda} (\vec{k}) a_{\lambda'}^{\dagger} - a_{\lambda}^{\dagger} (\vec{k}) a_{\lambda'}^{\dagger} (\vec{k}) e^{-2 i \omega t} \big) \Big) \frac{1}{2 \omega} \frac{d^{3} k}{(2 \pi)^{3} 2 \omega}
        \end{aligned}
    \end{equation}

    Now let's consider the $(\vec{k} \times \vec{\epsilon}_{\lambda}) \cdot (\vec{k} \times \vec{\epsilon}_{\lambda'})$ factor. We can evaluate it using the following identity:

    \begin{equation}
        (\vec{a} \times \vec{b}) \cdot (\vec{c} \times \vec{d}) = (\vec{a} \cdot \vec{c}) (\vec{b} \cdot \vec{d}) - (\vec{a} \cdot \vec{d}) (\vec{b} \cdot \vec{c})
    \end{equation}

    If we apply this identity to the quantity in question, we get (remembering that $\vec{\epsilon}_{\lambda} \cdot \vec{\epsilon}_{\lambda} = \delta_{\lambda \lambda'}$)

    \begin{equation}
        (\vec{k} \times \vec{\epsilon}_{\lambda}) \cdot (\vec{k} \times \vec{\epsilon}_{\lambda'}) = (\vec{k} \cdot \vec{k}) (\vec{\epsilon}_{\lambda} \cdot \vec{\epsilon}_{\lambda'}) - (\vec{k} \cdot \vec{\epsilon}_{\lambda'}) (\vec{\epsilon}_{\lambda} \cdot \vec{k}) = \vec{k}^{2} \delta_{\lambda \lambda'} = \omega^{2} \delta_{\lambda \lambda'}
    \end{equation}

    Inserting this into the Hamiltonian simplifies it a lot. Also we can apply the relation:

    \begin{equation}
        \begin{aligned}
            H = -\frac{1}{2} \sum_{\lambda \lambda'} \int \omega (a_{\lambda} (\vec{k}) a_{\lambda'} (\vec{k}) e^{-2 i \omega t} - a_{\lambda}^{\dagger} (\vec{k}) a_{\lambda} (\vec{k}) - a_{\lambda} (\vec{k}) a_{\lambda}^{\dagger}  (\vec{k})  \\
            + a_{\lambda}^{\dagger} (\vec{k}) a_{\lambda'}^{\dagger} (\vec{k}) e^{-2 i \omega t} - a_{\lambda} (\vec{k}) a_{\lambda'} (\vec{k}) e^{-2 i \omega t} - a_{\lambda}^{\dagger} (\vec{k}) a_{\lambda} (\vec{k}) - a_{\lambda} (\vec{k}) a_{\lambda}^{\dagger}  (\vec{k}) \\
            - a_{\lambda}^{\dagger} (\vec{k}) a_{\lambda'}^{\dagger} (\vec{k}) e^{-2 i \omega t}) \\
            \frac{1}{2 \omega} \frac{d^{3} k}{(2 \pi)^{3} 2 \omega}
        \end{aligned}
    \end{equation}

    \begin{equation}
        \begin{aligned}
            H = -\frac{1}{2} \sum_{\lambda \lambda'} \int \omega (- a_{\lambda}^{\dagger} (\vec{k}) a_{\lambda} (\vec{k}) - a_{\lambda} (\vec{k}) a_{\lambda}^{\dagger}  (\vec{k})  \\
            - a_{\lambda}^{\dagger} (\vec{k}) a_{\lambda} (\vec{k}) - a_{\lambda} (\vec{k}) a_{\lambda}^{\dagger}  (\vec{k}) ) \\
            \frac{1}{2 \omega} \frac{d^{3} k}{(2 \pi)^{3} 2 \omega}
        \end{aligned}
    \end{equation}

        \begin{equation}
            H = \sum_{\lambda} \int \omega (a_{\lambda}^{\dagger} (\vec{k}) a_{\lambda} (\vec{k}) + a_{\lambda} (\vec{k}) a_{\lambda}^{\dagger} (\vec{k})) \frac{d^{3} k}{(2 \pi)^{3} 2 \omega}
        \end{equation}

    We can now do the $\lambda'$ sum:

        \begin{equation}
            H = \sum_{\lambda} \int \omega \omega_{\lambda \lambda'} (a_{\lambda}^{\dagger} (\vec{k}) a_{\lambda} (\vec{k}) + a_{\lambda} (\vec{k}) a_{\lambda}^{\dagger} (\vec{k})) \frac{d^{3} k}{(2 \pi)^{3} 2 \omega}
        \end{equation}

    The usual normal ordering then gives the final answer:

    \begin{framed}
        \begin{equation}
            H = \sum_{\lambda} \int \omega a_{\lambda} (\vec{k})a_{\lambda}^{\dagger} (\vec{k}) \frac{d^{3} k}{(2 \pi)^{3} 2 \omega}
        \end{equation}
    \end{framed}

    Now for the momentum:

    \begin{equation}
        \vec{P} = \int \vec{E} \times \vec{B} \quad d^{3} x
    \end{equation}

    The field are as follows:

    \begin{equation}
        E^{j} = \sum_{\lambda} \int - i \omega (\vec{\epsilon}_{\lambda}^{j} a^{\dagger}_{\lambda} (\vec{k}) a_{\lambda'} (\vec{k}) e^{2 i \omega t} - \vec{\epsilon}_{\lambda}^{j} a_{\lambda} (\vec{k}) a^{\dagger}_{\lambda'} (\vec{k}) e^{- 2 i \omega t}) \frac{d^{3} k}{(2 \pi)^{3} 2 \omega}
    \end{equation}

    \begin{equation}
        B^{j} = \sum_{\lambda} \int - i (\vec{k} \times \vec{\epsilon}_{\lambda})^{j} (a^{\dagger}_{\lambda} (\vec{k}) a_{\lambda'} (\vec{k}) e^{2 i \omega t} + a_{\lambda} (\vec{k}) a^{\dagger}_{\lambda'} (\vec{k}) e^{- 2 i \omega t}) \frac{d^{3} k}{(2 \pi)^{3} 2 \omega}
    \end{equation}

    Inserting these:

    \begin{equation}
        \begin{aligned}
            \vec{P} = \sum_{\lambda \lambda'} \int \omega \vec{\epsilon}_{\lambda} \times (\vec{k} \times \vec{\epsilon}_{\lambda}) \\
            \big( a_{\lambda} (\vec{k}) e^{- i (\omega t - \vec{k} \cdot \vec{x})} - a_{\lambda}^{\dagger} (\vec{k}) e^{i (\omega t - \vec{k} \cdot \vec{x})} \big) \\
            \big( a_{\lambda'} (\vec{k}) e^{- i (\omega t - \vec{k} \cdot \vec{x})} - a_{\lambda'}^{\dagger} (\vec{k}) e^{i (\omega t - \vec{k} \cdot \vec{x})} \big) \\
            \frac{d^{3} k'}{(2 \pi)^{3} 2 \omega'} d^{3} x
        \end{aligned}
    \end{equation}

    \begin{equation}
        \begin{aligned}
            \vec{P} = \sum_{\lambda \lambda'} \int \omega \vec{\epsilon}_{\lambda} \times (\vec{k} \times \vec{\epsilon}_{\lambda}) \\
            \big( a_{\lambda} (\vec{k}) a_{\lambda'} (\vec{k}) e^{i (\omega + \omega')} e^{(\vec{k} + \vec{k'})} \\
            a^{\dagger}_{\lambda} (\vec{k}) a_{\lambda'} (\vec{k}) e^{i (\omega - \omega')} e^{(\vec{k} - \vec{k'})} + a_{\lambda} (\vec{k}) a^{\dagger}_{\lambda'} (\vec{k}) e^{i (\omega - \omega')} e^{(\vec{k} - \vec{k'})} \\
            a^{\dagger}_{\lambda} (\vec{k}) a^{\dagger}_{\lambda'} (\vec{k}) e^{i (\omega + \omega')} e^{(\vec{k} + \vec{k'})} \big)\\
            \frac{d^{3} k}{(2 \pi)^{3} 2 \omega} \frac{d^{3} k'}{(2 \pi)^{3} 2 \omega'} d^{3} x
        \end{aligned}
    \end{equation}

    Now the x integration can be done:

    \begin{equation}
        \begin{aligned}
            \vec{P} = \sum_{\lambda \lambda'} \int \omega \vec{\epsilon}_{\lambda} \times (\vec{k} \times \vec{\epsilon}_{\lambda}) \\
            \big( a_{\lambda} (\vec{k}) a_{\lambda'} (\vec{k}) e^{i (\omega + \omega')} (2\pi)^{3} \delta^{3} (\vec{k} + \vec{k'}) \\
            - a^{\dagger}_{\lambda} (\vec{k}) a_{\lambda'} (\vec{k}) e^{i (\omega - \omega')} (2\pi)^{3} \delta^{3} (\vec{k} - \vec{k'}) \\
            - a_{\lambda} (\vec{k}) a^{\dagger}_{\lambda'} (\vec{k}) e^{i (\omega - \omega')} (2\pi)^{3} \delta^{3} (\vec{k} - \vec{k'}) \\
            + a^{\dagger}_{\lambda} (\vec{k}) a^{\dagger}_{\lambda'} (\vec{k}) e^{i (\omega + \omega')} (2\pi)^{3} \delta^{3} (\vec{k} + \vec{k'}) \big) \\
            \frac{d^{3} k}{(2 \pi)^{3} 2 \omega} \frac{d^{3} k'}{(2 \pi)^{3} 2 \omega'}
        \end{aligned}
    \end{equation}

    Now, we can simplify, partly using the properties of the delta function to simplify the remaining phase factors:
    
    \begin{equation}
        \begin{aligned}
            \vec{P} = \sum_{\lambda \lambda'} \int \frac{1}{2} & \vec{\epsilon}_{\lambda} \times (\vec{k} \times \vec{\epsilon}_{\lambda}) \\
            (a^{\dagger}_{\lambda} (\vec{k}) a_{\lambda'} (\vec{k}) e^{2 i \omega t} \delta^{3} (\vec{k} + \vec{k'}) - (a^{\dagger}_{\lambda} (\vec{k}) a_{\lambda'} (\vec{k}) e^{2 i \omega t} \delta^{3} (\vec{k} - \vec{k'}) \\
            - (a^{\dagger}_{\lambda} (\vec{k}) a_{\lambda'} (\vec{k}) e^{2 i \omega t} \delta^{3} (\vec{k} - \vec{k'}) + a_{\lambda} (\vec{k}) a^{\dagger}_{\lambda'} (\vec{k}) e^{- 2 i \omega t}) \delta^{3} (\vec{k} + \vec{k'}) \\
            \frac{d^{3} k}{(2 \pi)^{3} 2 \omega}
        \end{aligned}
    \end{equation}

    Now the $k'$ integration  can be done using all of the delta functions:

    \begin{equation}
        \begin{aligned}
            \vec{P} = \sum_{\lambda \lambda'} \int \frac{1}{2} \vec{\epsilon}_{\lambda} \times (\vec{k} \times \vec{\epsilon}_{\lambda}) (a_{\lambda} (\vec{k}) a_{\lambda'} (\vec{k}) e^{2 i \omega t} - a^{\dagger}_{\lambda} (\vec{k}) a_{\lambda'} (\vec{k}) - a_{\lambda} (\vec{k}) a^{\dagger}_{\lambda'} (\vec{k}) + a^{\dagger}_{\lambda} (\vec{k}) a^{\dagger}_{\lambda'} (\vec{k}) e^{- 2 i \omega t}) \frac{d^{3} k}{(2 \pi)^{3} 2 \omega} \\
        \end{aligned}
    \end{equation}

    \begin{equation}
        \begin{aligned}
            \vec{P} = \sum_{\lambda \lambda'} \int \frac{1}{2} \vec{\epsilon}_{\lambda} \times (\vec{k} \times \vec{\epsilon}_{\lambda}) (a_{\lambda} (\vec{k}) a_{\lambda'} (\vec{k}) e^{2 i \omega t} + a^{\dagger}_{\lambda} (\vec{k}) a^{\dagger}_{\lambda'} (\vec{k}) e^{- 2 i \omega t}) \frac{d^{3} k}{(2 \pi)^{3} 2 \omega} \\
            + \sum_{\lambda \lambda'} \int \frac{1}{2} \vec{\epsilon}_{\lambda} \times (\vec{k} \times \vec{\epsilon}_{\lambda}) (- a^{\dagger}_{\lambda} (\vec{k}) a_{\lambda'} (\vec{k}) - a_{\lambda} (\vec{k}) a^{\dagger}_{\lambda'} (\vec{k}) \frac{d^{3} k}{(2 \pi)^{3} 2 \omega} \\
        \end{aligned}
    \end{equation}

    The first integral has an integrand that is odd in $\vec{k}$, so it vanishes, therefore:

    \begin{equation}
        \vec{P} = \sum_{\lambda \lambda'} \int \frac{1}{2} \vec{\epsilon}_{\lambda} \times (\vec{k} \times \vec{\epsilon}_{\lambda}) (a^{\dagger}_{\lambda} (\vec{k}) a_{\lambda'} (\vec{k}) + a_{\lambda} (\vec{k}) a^{\dagger}_{\lambda'} (\vec{k})) \frac{d^{3} k}{(2 \pi)^{3} 2 \omega}
    \end{equation}

    Normal ordering then gives:

    \begin{equation}
        \vec{P} = \sum_{\lambda \lambda'} \int \vec{\epsilon}_{\lambda} \times (\vec{k} \times \vec{\epsilon}_{\lambda}) a^{\dagger}_{\lambda} (\vec{k}) a_{\lambda'} (\vec{k}) \frac{d^{3} k}{(2 \pi)^{3} 2 \omega}
    \end{equation}

    \begin{equation}
        \vec{e}_{\lambda} \times (\vec{k} \times \vec{e}_{\lambda'}) = \vec{k} \vec{\epsilon}_{\lambda} \cdot \vec{\epsilon}_{\lambda'} - \vec{\epsilon}_{\lambda'} \vec{\epsilon}_{\lambda} \cdot \vec{k} = \vec{k} \delta_{\lambda \lambda'}
    \end{equation}

    \begin{equation}
        \vec{P} = \sum_{\lambda \lambda'} \int \vec{k} \delta_{\lambda \lambda'} a^{\dagger}_{\lambda} (\vec{k}) a_{\lambda'} (\vec{k}) \frac{d^{3} k}{(2 \pi)^{3} 2 \omega}
    \end{equation}

    Now, doing the $\lambda'$ sum gives the final answer:

    \begin{framed}
        \begin{equation}
            \vec{P} = \sum_{\lambda} \int \vec{k} a^{\dagger}_{\lambda} (\vec{k}) a_{\lambda} (\vec{k}) \frac{d^{3} k}{(2 \pi)^{3} 2 \omega}
        \end{equation}
    \end{framed}

    Now, we can easily verify that $a^{\dagger}_{\lambda} (\vec{k})$ and $a_{\lambda} (\vec{k})$ do behave like creation and annihilation operators.
    Let's start with $a_{\lambda} (\vec{k})$:

    \begin{equation}
        H_{n} | E \rangle = | E \rangle
    \end{equation}

    \begin{equation}
        \begin{aligned}
            H_{n} & = \sum_{\lambda'} \int \omega' a_{\lambda'}^{\dagger} (\vec{k'}) a_{\lambda'} (\vec{k'}) a_{\lambda} (\vec{k}) | E \rangle \frac{d^3 k'}{(2 \pi^{3} 2 \omega')} \\
            & = \sum_{\lambda'} \int \omega' a_{\lambda'}^{\dagger} (\vec{k'}) a_{\lambda} (\vec{k}) a_{\lambda'} (\vec{k'}) \frac{d^3 k'}{(2 \pi^{3} 2 \omega')} \\
            & = \sum_{\lambda'} \int \omega' [a_{\lambda} (\vec{k}) a_{\lambda'} (\vec{k'})^{\dagger} - [a_{\lambda} (\vec{k}), a_{\lambda'} (\vec{k'})^{\dagger}]] \frac{d^3 k'}{(2 \pi^{3} 2 \omega')} \\
            & = \sum_{\lambda'} \int \omega' [a_{\lambda} (\vec{k}) a_{\lambda'} (\vec{k'})^{\dagger} - (2 \pi)^{3} 2 \omega \delta^{3} (\vec{k} - \vec{k'})] \frac{d^3 k'}{(2 \pi^{3} 2 \omega')} \\
            & = a_{\lambda} (\vec{k}) \sum_{\lambda'} \int \omega' a_{\lambda'}^{\dagger} (\vec{k'}) a_{\lambda'} (\vec{k'}) | E \rangle - \omega a_{\lambda} (\vec{k}) | E \rangle \frac{d^3 k'}{(2 \pi^{3} 2 \omega')} \\
            & = a_{\lambda} (\vec{k}) H_{n} | E \rangle - \omega a_{\lambda} (\vec{k}) | E \rangle = a_{\lambda}^{\dagger} (\vec{k}) E | E \rangle - \omega a_{\lambda} (\vec{k}) | E \rangle = (E - \omega) a_{\lambda} (\vec{k}) | E \rangle
        \end{aligned}
    \end{equation}

    \begin{framed}
        \begin{equation}
            H_{n} a_{\lambda} (\vec{k}) | E \rangle = (E - \omega) a_{\lambda} (\vec{k}) | E \rangle
        \end{equation}
    \end{framed}

    Now, we can do the same for $a^{\dagger}_{\lambda} (\vec{k})$:

    \begin{equation}
        \begin{aligned}
            H_{n} & = \sum_{\lambda'} \int \omega' a_{\lambda'}^{\dagger} (\vec{k}) a_{\lambda'} (\vec{k}') a_{\lambda}^{\dagger} (\vec{k}) | E \rangle \frac{d^3 k'}{(2 \pi^{3} 2 \omega')} \\
            & = \sum_{\lambda'} \int \omega' [a_{\lambda}^{\dagger} (\vec{k}) a_{\lambda'} (\vec{k'}) - [a_{\lambda} (\vec{k}), a_{\lambda'} (\vec{k'})^{\dagger}]] \frac{d^3 k'}{(2 \pi^{3} 2 \omega')} \\
            & = \sum_{\lambda'} \int \omega' [a_{\lambda}^{\dagger} (\vec{k}) a_{\lambda'} (\vec{k'}) + [a_{\lambda}^{\dagger} (\vec{k}), a_{\lambda'} (\vec{k'})]] \frac{d^3 k'}{(2 \pi^{3} 2 \omega')} \\
            & = \sum_{\lambda'} \int \omega' [a_{\lambda} (\vec{k}) a_{\lambda'} (\vec{k'})^{\dagger} + (2 \pi)^{3} 2 \omega \delta^{3} (\vec{k} - \vec{k'})] \frac{d^3 k'}{(2 \pi^{3} 2 \omega')} \\
            & = \sum_{\lambda'} \int \omega' a_{\lambda'}^{\dagger} (\vec{k}) a_{\lambda}^{\dagger} (\vec{k}) a_{\lambda} (\vec{k}) - \omega a_{\lambda} (\vec{k}) | E \rangle \frac{d^3 k'}{(2 \pi^{3} 2 \omega')} \\
            & = a_{\lambda} (\vec{k}) \sum_{\lambda'} \int \omega' a_{\lambda'}^{\dagger} (\vec{k}) a_{\lambda} (\vec{k}) | E \rangle + \omega a_{\lambda}^{\dagger} (\vec{k}) | E \rangle \frac{d^3 k'}{(2 \pi^{3} 2 \omega')} \\
            & = a_{\lambda}^{\dagger} (\vec{k}) H_{n} | E \rangle + \omega a_{\lambda}^{\dagger} (\vec{k}) | E \rangle = a_{\lambda}^{\dagger} (\vec{k}) E | E \rangle + \omega a_{\lambda}^{\dagger} (\vec{k}) | E \rangle = (E + \omega) a_{\lambda} (\vec{k}) | E \rangle
        \end{aligned}
    \end{equation}

    \begin{framed}
        \begin{equation}
            H_{n} a_{\lambda} (\vec{k}) | E \rangle = (E + \omega) a_{\lambda} (\vec{k}) | E \rangle
        \end{equation}
    \end{framed}

    Now, as usual, because the Hamiltonian is an integral over the product $a^{\dagger}_{\lambda} (\vec{k}) a_{\lambda} (\vec{k})$. It's expectation
    values are non-negative:

    \begin{equation}
        \langle \psi | H | \psi \rangle \geq 0
    \end{equation}

    Therefore, we can make the normal argument that there must be a lower state annihilated by $a_{\lambda} (\vec{k})$:

    \begin{equation}
        a_{\lambda} (\vec{k}) | 0 \rangle = 0
    \end{equation}

    And therefore, that one can construct ll of the states by applying creation operators to the vacuum state. One particle states can be written as follows:

    \begin{equation}
        a^{\dagger}_{\lambda} (\vec{k}) | \vec{k}, \lambda \rangle 
    \end{equation}

    An arbitrary multi particle state is therefore given by:

    \begin{equation}
        \prod_{i = 1}^{K} \frac{[a^{\dagger}_{\lambda} (\vec{k})]}{\sqrt{n (\vec{k}_i)!}} | 0 \rangle = n (\vec{k}_{1}, \lambda_{1}) | n (\vec{k}_{1}, \lambda_{1}) ... n (\vec{k}_{K}, \lambda_{K}) \rangle
    \end{equation}

    Additionally, one can construct number operators that behaves in the usual way:

    \begin{equation}
        N_{\lambda} (\vec{k}) = a^{\dagger}_{\lambda} (\vec{k}) a_{\lambda} (\vec{k})
    \end{equation}

    \begin{equation}
        N_{\lambda_{1}} (\vec{k}_{1}) | n (\vec{k}_{1}, \lambda_{1}) ... n (\vec{k}_{K}, \lambda_{K}) \rangle = n (\vec{k}_{1}, \lambda_{1}) | n (\vec{k}_{1}, \lambda_{1}) ... n (\vec{k}_{K}, \lambda_{K}) \rangle
    \end{equation}

    The zero point on the energy scale is set such that the energy of the vacuum has the value zero. As a consequence:

    \begin{equation}
        H | \vec{k}, \lambda \rangle = \omega | \vec{k}, \lambda \rangle
    \end{equation}

    We also have:

    \begin{equation}
        \vec{P} | \vec{k}, \lambda \rangle = \vec{k} | \vec{k}, \lambda \rangle
    \end{equation}

    This completes the quantization of the Electromagnetic Field.

\end{document}