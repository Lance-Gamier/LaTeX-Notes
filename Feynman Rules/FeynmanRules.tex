\documentclass[a4]{article}

\usepackage{braket}
\usepackage{amssymb}
\usepackage{amsmath}
\usepackage{slashed}
\usepackage{float}
\usepackage{framed}
\usepackage{xcolor}
\usepackage[compat = 1.1.0]{tikz-feynman}
\usepackage{tcolorbox}

\usepackage[left = 1cm,right = 1.4cm, top = 2cm]{geometry}
\usepackage{array}
\usepackage{longtable}

\newcommand{\highlight}[2][yellow]{\mathchoice
  {\colorbox{#1}{$\displaystyle#2$}}
  {\colorbox{#1}{$\textstyle#2$}}
  {\colorbox{#1}{$\scriptstyle#2$}}
  {\colorbox{#1}{$\scriptscriptstyle#2$}}}

\begin{document}

    \title{Deriving the Feynman Rules for Quantum Electrodynamics (from the S-matrix expansion)}
    \maketitle

    \tableofcontents


    \section{Introduction}

        In this section, I will show you how to derive the Feynman Rules for Quantum Electrodynamics (QED),
        These rules are one of the most famous parts of Feynman's Legacy. The derivation will begin by expanding
        the S-Operator to $2^{nd}$ Order. We will then compute the scattering matrix up to $2^{nd}$ Order
        (Ignoring the zeroth order forward scattering term) for a variety of physical processes by evaluating
        the corresponding matrix elements of the S-Operator expanded to $2^{nd}$ Order. Once we have the raw
        matrix elements written out (but before we have evaluated them) I will also show how to interpret each
        S-matrix expansion term for each physical process. The complete set of physical processes that we will
        be considering in our effort to Derive the Feynman Rules can be found on the Table of Contents, and in the
        table below:

        \vspace{0.75cm}

        \begin{center}
        \begin{tabular}{|c c c|}
            \hline
            PROCESS & REACTION & S-MATRIX ELEMENT \\
            $e^{+}$ Compton Scattering & $\gamma + e^{-} \rightarrow \gamma + e^{-}$ & $\braket{\gamma , e^{-}|S|\gamma , e^{-}}$ \\
            $e^{-}$ Compton Scattering & $\gamma + e^{+} \rightarrow \gamma + e^{+}$ & $\braket{\gamma , e^{+}|S|\gamma , e^{+}}$ \\
            Pair Annihilation & $e^{-} + e^{+} \rightarrow \gamma + \gamma$ & $\braket{e^{-} , e^{+}|S|\gamma , \gamma}$ \\
            Pair Production & $\gamma + \gamma \rightarrow e^{-} + e^{+}$ & $\braket{\gamma , \gamma|S|e^{-} , e^{+}}$ \\
            $e^{+}$ Moller Scattering & $e^{+} + e^{+} \rightarrow e^{+} + e^{+}$ & $\braket{e^{+} , e^{+}|S|e^{+} , e^{+}}$ \\
            $e^{-}$ Moller Scattering & $e^{-} + e^{-} \rightarrow e^{-} + e^{-}$ & $\braket{e^{-} , e^{-}|S|e^{-} , e^{-}}$ \\
            Bhabha Scattering & $e^{-} + e^{+} \rightarrow e^{-} + e^{+}$ & $\braket{e^{-} , e^{+}|S|e^{-} , e^{+}}$ \\
            Electron Self-Energy & $e^{-} \rightarrow e^{-}$ & $\braket{e^{-}|S|e^{-}}$ \\
            Positron Self-Energy & $e^{+} \rightarrow e^{+}$ & $\braket{e^{+}|S|e^{+}}$ \\
            Photon Self-Energy & $\gamma \rightarrow \gamma$ & $\braket{\gamma|S|\gamma}$ \\
            Vacuum Self-Energy & $Vacuum \rightarrow Vacuum$ & $\braket{0|S|0}$ \\
            \hline
        \end{tabular}
        \end{center}

        \vspace{0.75cm}

        When proceeding through this process, it won't be quite as straightforward as I have made it sound so far.
        There will be lots of little steps to complete as we go. For example, we will have to prove that the first
        order term doesn't yield a non-vanishing contribution to anything. This will show that we are justified in
        not including processes that one would expect to have first order contributions in the above table. It is
        actually easy to see that this must be the case for the listed process above, but I will show that all first
        order contributions vanish even for matrix elements that, at face value, seem like they ought to have first
        order contribution.

    \section{Scattering Order to the $2^{nd}$ Order}

        Obviously, before we expand the S-operator to 2nd order, we need an expression for the S-Operator. This
        requires us to know what the Interacting Hamiltonian is for the theory.

        So let's begin with the Lagrangian density for QED:

        \begin{equation}
            \mathcal{L_{QED}} = i \overline{\psi} \gamma^{\mu} D_{\mu} \psi - m \overline{\psi} \psi - \frac{1}{4} F_{\mu\nu} F^{\mu\nu}
        \end{equation}

        Where the U(1) Gauge Covariant Derivative is:

        \begin{equation}
            D_{\mu} \psi = [\partial_{\mu} + i e A_{\mu}] \psi
        \end{equation}

        We can insert this into the Lagrangian density:

        \begin{equation}
            \mathcal{L_{QED}} = i \overline{\psi} \gamma^{\mu} \partial_{\mu} \psi - e \overline{\psi} \gamma^{\mu} A_{\mu} \psi - m \overline{\psi} \psi - \frac{1}{4} F_{\mu\nu} F^{\mu\nu}
        \end{equation}

        So the interaction Lagrangian is:

        \begin{equation}
            \mathcal{L_{INT}} = - e \overline{\psi} \slashed{A}_{\mu} \psi
        \end{equation}

        Therefore in this case the interaction Hamiltonian is:

        \begin{equation}
            \mathcal{H_{INT}} = e \overline{\psi} \slashed{A}_{\mu} \psi
        \end{equation}

        Now we insert this into the S-Operator, and then expand it. For mathematical convenience, I will actually do this in reverse order:

        \begin{equation}
            S = T \bigg[ exp \bigg( -i \int_{\infty}^{\infty} \mathcal{H_{INT}} (x) d^{4}x \bigg) \bigg] = T \bigg[ \sum_{n = 0}^{\infty} \frac{(-1)^n}{n!} \int d^{4} x_{1} \int \mathcal{H_{INT}} (x_1) ... \mathcal{H_{INT}} (x_n) d^{4} x_{1} \bigg] = \sum_{n = 0}^{\infty} S^{(n)}
        \end{equation}

        I will denote a spacetime point argument $(\vec{x}, t)$ = $(x^{\mu})$ as $(x)$. $S^{0}$ is just the identity and therefore just gives the forward scattering part. It can be ignored because all the cross section and decay rate formulas
        that we might be interested in, ignore the forward scattering part. So, now let's look at the first order term:

        \begin{equation}
            S^{(1)} = - i e \int T \Big( \overline{\psi} (x) A_{\mu} (x) \psi (x) \Big) d^{4} x
        \end{equation}

        Wick Expansion is required here, but the wick expansion containts only one term:
        
        \begin{equation}
            S^{(1)} = - i e \int :\overline{\psi} A_{\mu} \psi: d^{4} x
        \end{equation}

        One might think that one would have to account for the contributions made by this term to various S-matrix elements, but it turns out that this term actually only gives a vanishing contribution to physical processes (as mentioned in the intro). This will be shown
        in section 4 starting from the last expression for $S^{(1)}$, just after I introduce the operation of field operators, which I make use of in that section. Before we get there, let's look at the second term. According to the expansion abive, the second order term
        is 

        \begin{equation}
            S^{(2)} = \frac{(-ie)^2}{2!} \int T \Big( \overline{\psi} (x_1) A_{\mu} (x_1) \psi (x_1) \overline{\psi} (x_2) A_{\mu} (x_2) \psi (x_2) \Big) d^{4} x_{1} d^{4} x_{2}
        \end{equation}

        We can use Wick's Theorem to write $S^{(2)}$ in terms of normal ordered products and contracted products of two fields. Here, "contraction" refers to taking vacuum expectation value of the product
        of two quantum fields, which is just the propagator:

        \begin{equation}
            \highlight[green]{A_{\mu} (x_1) A_{\nu} (x_2)} = \braket{0|A_{\mu} (x_1) A_{\nu} (x_2)|0} = i D_{F}^{\mu\nu} (x_2 - x_1)
        \end{equation}

        \begin{equation}
            \highlight[green]{\psi (x_1) \overline{\psi} (x_2)} = \braket{0|\psi (x_1) \overline{\psi} (x_2)|0} = i S_{F} (x_2 - x_1)
        \end{equation}

        I will initially denote contraction with highlighting. With this Notation, the Wick expanded second order term of the S-Operator is as follows:

        \begin{equation}
            S^{(2)} = S^{(2)}_{0} + S^{(2)}_{1} + S^{(2)}_{2} + S^{(2)}_{3}
        \end{equation}

        Where

        \begin{framed}
            \begin{center}
                Zero Contraction Term
            \end{center}

            \begin{equation}
                S^{(2)}_{0} = \frac{(-ie)^2}{2!} \int :\overline{\psi} (x_1) \gamma^{\mu} A_{\mu} (x_1) \psi (x_1) \overline{\psi} (x_2) \gamma^{\nu} A_{\nu} (x_2) \psi (x_2): d^{4} x_{1} d^{4} x_{2}
            \end{equation}

            \begin{center}
                One Contraction Term
            \end{center}

            \begin{equation}
                S^{(2)}_{1,1} = \frac{(-ie)^2}{2!} \int :\overline{\psi} (x_1) \gamma^{\mu} A_{\mu} (x_1) \highlight[green]{\psi (x_1) \overline{\psi} (x_2)} \gamma^{\nu} A_{\nu} (x_2) \psi (x_2): d^{4} x_{1} d^{4} x_{2}
            \end{equation}

            \begin{equation}
                S^{(2)}_{1,2} = \frac{(-ie)^2}{2!} \int :\overline{\psi} (x_1) \gamma^{\mu} \highlight[green]{A_{\mu} (x_1)} \psi (x_1) \overline{\psi} (x_2) \gamma^{\nu} \highlight[green]{A_{\nu} (x_2)} \psi (x_2): d^{4} x_{1} d^{4} x_{2}
            \end{equation}

            \begin{equation}
                S^{(2)}_{1,3} = \frac{(-ie)^2}{2!} \int :\highlight[green]{\overline{\psi} (x_1)} \gamma^{\mu} A_{\mu} (x_1) \psi (x_1) \overline{\psi} (x_2) \gamma^{\nu} A_{\nu} (x_2) \highlight[green]{\psi (x_2)}: d^{4} x_{1} d^{4} x_{2}
            \end{equation}

            \begin{center}
                Two Contraction Term
            \end{center}

            \begin{equation}
                S^{(2)}_{2,1} = \frac{(-ie)^2}{2!} \int :\overline{\psi} (x_1) \gamma^{\mu} \highlight[cyan]{A_{\mu} (x_1)} \highlight[green]{\psi (x_1) \overline{\psi} (x_2)} \gamma^{\nu} \highlight[cyan]{A_{\nu} (x_2)} \psi (x_2): d^{4} x_{1} d^{4} x_{2}
            \end{equation}

            \begin{equation}
                S^{(2)}_{2,2} = \frac{(-ie)^2}{2!} \int :\highlight[green]{\overline{\psi} (x_1)} \gamma^{\mu} \highlight[cyan]{A_{\mu} (x_1)} \psi (x_1) \overline{\psi} (x_2) \gamma^{\nu} \highlight[cyan]{A_{\nu} (x_2)} \highlight[green]{\psi (x_2)}: d^{4} x_{1} d^{4} x_{2}
            \end{equation}

            \begin{equation}
                S^{(2)}_{2,3} = \frac{(-ie)^2}{2!} \int :\highlight[cyan]{\overline{\psi} (x_1)} \gamma^{\mu} A_{\mu} (x_1) \highlight[green]{\psi (x_1) \overline{\psi} (x_2)} \gamma^{\nu} A_{\nu} (x_2)  \highlight[cyan]{\psi (x_2)}: d^{4} x_{1} d^{4} x_{2}
            \end{equation}

            \begin{center}
                Three Contraction Term
            \end{center}

            \begin{equation}
                S^{(2)}_{3} = \frac{(-ie)^2}{2!} \int :\highlight[magenta]{\overline{\psi} (x_1)} \gamma^{\mu} \highlight[cyan]{A_{\mu} (x_1)} \highlight[green]{\psi (x_1) \overline{\psi} (x_2)} \gamma^{\nu} \highlight[cyan]{A_{\nu} (x_2)} \highlight[magenta]{\psi (x_2)}: d^{4} x_{1} d^{4} x_{2}
            \end{equation}

        \end{framed}

        The various contracted fields are contracted with the field carrying the matching highlighting. We can now insert the propagators. For some of the terms this will require manipulations. I will explain for those that require them here.
        First, I will discuss $S^{(2)}_{1,3}$. Not only can we manipulate it so that the propagator is convenient to substitute in, we can prove that it is equal to $S^{(2)}_{1,1}$, so that we will work the sum of the two.

        \begin{equation}
            \begin{aligned}
            S^{(2)}_{1,1} + S^{(2)}_{1,3} = & \frac{(-ie)^2}{2!} \int \big[ :\overline{\psi} (x_1) \gamma^{\mu} A_{\mu} (x_1) \highlight[green]{\psi (x_1) \overline{\psi} (x_2)} \gamma^{\nu} A_{\nu} (x_2) \psi (x_2): \\
            + & :\highlight[green]{\overline{\psi} (x_1)} \gamma^{\mu} A_{\mu} (x_1) \psi (x_1) \overline{\psi} (x_2) \gamma^{\nu} A_{\nu} (x_2) \highlight[green]{\psi (x_2)}: \big] d^{4} x_{1} d^{4} x_{2}
            \end{aligned}
        \end{equation}
        
        In the second term, we will bring the second group of three fields to the left without changing any signs. This is because the fields being commuted or anticommuted are evaluated at different points, so the various necessary commutators
        and anticommutators vanish, and there are even number of fermion field anticommutations, so all of the minus signs cancel out. This gives:

        \begin{equation}
            \begin{aligned}
            S^{(2)}_{1,1} + S^{(2)}_{1,3} = & \frac{(-ie)^2}{2!} \int \big[ :\overline{\psi} (x_1) \gamma^{\mu} A_{\mu} (x_1) \highlight[green]{\psi (x_1) \overline{\psi} (x_2)} \gamma^{\nu} A_{\nu} (x_2) \psi (x_2): \\
            + & :\overline{\psi} (x_1) \gamma^{\mu} A_{\mu} (x_1) \highlight[green]{\psi (x_1) \overline{\psi} (x_2)} \gamma^{\nu} A_{\nu} (x_2) \psi (x_2): \big] d^{4} x_{1} d^{4} x_{2}
            \end{aligned}
        \end{equation}

        Because both $x_{1}$ and $x_{2}$ are completely integrated over, they are just dummy variables that can be interchanged. I will therefore flip the labels in the second term. 
        We can now see that the two terms are identical. So we can now combine like terms:

        \begin{equation}
            S^{(2)}_{1,1} + S^{(2)}_{1,3} = - e^{2} \int :\overline{\psi} (x_1) \gamma^{\mu} A_{\mu} (x_1) \highlight[green]{\psi (x_1) \overline{\psi} (x_2)} \gamma^{\nu} A_{\nu} (x_2) \psi (x_2): d^{4} x_{1} d^{4} x_{2}
        \end{equation}

        We can now insert the fermion propagator:

        \begin{framed}
            \begin{equation}
                S^{(2)}_{1,1} + S^{(2)}_{1,3} = - e^{2} \int :\overline{\psi} (x_1) \gamma^{\mu} A_{\mu} (x_1) i S_{F} (x_2 - x_1) \gamma^{\nu} A_{\nu} (x_2) \psi (x_2): d^{4} x_{1} d^{4} x_{2}
            \end{equation}
        \end{framed}

        We essentially find the same situation for $S^{(2)}_{2,1}$ and $S^{(2)}_{2,2}$ as we did for $S^{(2)}_{1,1}$ and $S^{(2)}_{1,3}$, so we will treat the sum of them.

        \begin{equation}
            \begin{aligned}
            S^{(2)}_{2,1} + S^{(2)}_{2,2} = & \frac{(-ie)^2}{2!} \int \big[ :\overline{\psi} (x_1) \gamma^{\mu} \highlight[cyan]{A_{\mu} (x_1)} \highlight[green]{\psi (x_1) \overline{\psi} (x_2)} \gamma^{\nu} \highlight[cyan]{A_{\nu} (x_2)} \psi (x_2): \\
            + & :\highlight[green]{\overline{\psi} (x_1)} \gamma^{\mu} \highlight[cyan]{A_{\mu} (x_1)} \psi (x_1) \overline{\psi} (x_2) \gamma^{\nu} \highlight[cyan]{A_{\nu} (x_2)} \highlight[green]{\psi (x_2)}: \big] d^{4} x_{1} d^{4} x_{2}
            \end{aligned}
        \end{equation}

        In the second term, we can again bring the second group of the three fields to the left without changing any signs. The reason for this is as before:

        \begin{equation}
            \begin{aligned}
            S^{(2)}_{2,1} + S^{(2)}_{2,2} = & \frac{(-ie)^2}{2!} \int \big[ :\overline{\psi} (x_1) \gamma^{\mu} \highlight[cyan]{A_{\mu} (x_1)} \highlight[green]{\psi (x_1) \overline{\psi} (x_2)} \gamma^{\nu} \highlight[cyan]{A_{\nu} (x_2)} \psi (x_2): \\
            + & :\overline{\psi} (x_1) \gamma^{\mu} \highlight[cyan]{A_{\mu} (x_1)} \highlight[green]{\psi (x_1) \overline{\psi} (x_2)} \gamma^{\nu} \highlight[cyan]{A_{\nu} (x_2)} \psi (x_2): \big] d^{4} x_{1} d^{4} x_{2}
            \end{aligned}
        \end{equation}

        Both $x_1$ and $x_2$ are dummy variables, so I am free to flip them in the second term. The two terms are now clearly identical:

        \begin{equation}
            S^{(2)}_{2,1} + S^{(2)}_{2,2} = - e^{2} \int :\overline{\psi} (x_1) \gamma^{\mu} \highlight[cyan]{A_{\mu} (x_1)} \highlight[green]{\psi (x_1) \overline{\psi} (x_2)} \gamma^{\nu} \highlight[cyan]{A_{\nu} (x_2)} \psi (x_2): d^{4} x_{1} d^{4} x_{2}
        \end{equation}

        Now, we can substitute in the propagators:

        \begin{framed}
            \begin{equation}
                S^{(2)}_{2,1} + S^{(2)}_{2,2} = - e^{2} \int :\overline{\psi} (x_1) \gamma^{\mu} i S_{F} (x_2 - x_1) \gamma^{\nu} i D_{F}^{\mu\nu} (x_2 - x_1) \psi (x_2): d^{4} x_{1} d^{4} x_{2}
            \end{equation}
        \end{framed}

        Some special manipulations are also required for the convenient substitution of the propagators into $S^{(2)}_{1,3}$. Namely, we will substitute the propagators in, and we will take
        the trace of the scalar integrand and make use of the cyclic property of trace:

        \begin{equation}
            \begin{aligned}
            S^{(2)}_{2,3} = & \frac{(-ie)^2}{2!} \int :\highlight[cyan]{\overline{\psi} (x_1)} \gamma^{\mu} A_{\mu} (x_1) \highlight[green]{\psi (x_1) \overline{\psi} (x_2)} \gamma^{\nu} A_{\nu} (x_2)  \highlight[cyan]{\psi (x_2)}: d^{4} x_{1} d^{4} x_{2} \\
            i S_{F} & (x_2 - x_1) = \highlight[green]{\psi (x_1) \overline{\psi} (x_2)} = \braket{0|\psi (x_1) \overline{\psi} (x_2)|0}
            \end{aligned}
        \end{equation}

        \begin{equation}
            S^{(2)}_{2,3} = \frac{(-ie)^2}{2!} \int :\highlight[cyan]{\overline{\psi} (x_1)} \gamma^{\mu} A_{\mu} (x_1) i S_{F} (x_2 - x_1) \gamma^{\nu} A_{\nu} (x_2)  \highlight[cyan]{\psi (x_2)}: d^{4} x_{1} d^{4} x_{2}
        \end{equation}

        \begin{equation}
            S^{(2)}_{2,3} = \frac{(-ie)^2}{2!} \int :Tr[\highlight[cyan]{\overline{\psi} (x_1)} \gamma^{\mu} A_{\mu} (x_1) i S_{F} (x_2 - x_1) \gamma^{\nu} A_{\nu} (x_2)  \highlight[cyan]{\psi (x_2)}]: d^{4} x_{1} d^{4} x_{2}
        \end{equation}

        \begin{equation}
            S^{(2)}_{2,3} = \frac{(-ie)^2}{2!} \int :Tr[ \highlight[cyan]{\psi (x_2)} \highlight[cyan]{\overline{\psi} (x_1)} \gamma^{\mu} A_{\mu} (x_1) i S_{F} (x_2 - x_1) \gamma^{\nu} A_{\nu} (x_2)]: d^{4} x_{1} d^{4} x_{2}
        \end{equation}

        \begin{framed}
            \begin{equation}
                S^{(2)}_{2,3} = \frac{(-ie)^2}{2!} \int :Tr[ i S_{F} (x_2 - x_1) \gamma^{\mu} A_{\mu} (x_1) i S_{F} (x_2 - x_1) \gamma^{\nu} A_{\nu} (x_2)]: d^{4} x_{1} d^{4} x_{2}
            \end{equation}
        \end{framed}

        The three contraction term requires similar treatment of the trace:

        \begin{equation}
            S^{(2)}_{3} = \frac{(-ie)^2}{2!} \int :\highlight[magenta]{\overline{\psi} (x_1)} \gamma^{\mu} \highlight[cyan]{A_{\mu} (x_1)} \highlight[green]{\psi (x_1) \overline{\psi} (x_2)} \gamma^{\nu} \highlight[cyan]{A_{\nu} (x_2)} \highlight[magenta]{\psi (x_2)}: d^{4} x_{1} d^{4} x_{2}
        \end{equation}

        \begin{equation}
            S^{(2)}_{3} = \frac{(-ie)^2}{2!} \int :\highlight[magenta]{\overline{\psi} (x_1)} \gamma^{\mu} S_{F} (x_2 - x_1) \gamma^{\nu} \highlight[cyan]{A_{\nu} (x_2)} \highlight[cyan]{A_{\mu} (x_1)} \highlight[magenta]{\psi (x_2)}: d^{4} x_{1} d^{4} x_{2}
        \end{equation}

        \begin{equation}
            S^{(2)}_{3} = \frac{(-ie)^2}{2!} \int :Tr[\highlight[magenta]{\overline{\psi} (x_1)} \gamma^{\mu} S_{F} (x_2 - x_1) \gamma^{\nu} i D_{F}^{\mu\nu} (x_2 - x_1) \highlight[magenta]{\psi (x_2)}]: d^{4} x_{1} d^{4} x_{2}
        \end{equation}

        \begin{equation}
            S^{(2)}_{3} = \frac{(-ie)^2}{2!} \int :\highlight[magenta]{\psi (x_2)} \highlight[magenta]{\overline{\psi} (x_1)} \gamma^{\mu} S_{F} (x_2 - x_1) \gamma^{\nu} i D_{F}^{\mu\nu} (x_2 - x_1): d^{4} x_{1} d^{4} x_{2}
        \end{equation}

        \begin{equation}
            S^{(2)}_{3} = \frac{(-ie)^2}{2!} \int :Tr[\highlight[magenta]{\psi (x_2)} \highlight[magenta]{\overline{\psi} (x_1)} \gamma^{\mu} S_{F} (x_2 - x_1) \gamma^{\nu} i D_{F}^{\mu\nu} (x_2 - x_1)]: d^{4} x_{1} d^{4} x_{2}
        \end{equation}

        \begin{framed}
            \begin{equation}
                S^{(2)}_{3} = \frac{(-ie)^2}{2!} \int :Tr[S_{F} (x_1 - x_2) \gamma^{\mu} S_{F} (x_2 - x_1) \gamma^{\nu} i D_{F}^{\mu\nu} (x_2 - x_1)]: d^{4} x_{1} d^{4} x_{2}
            \end{equation}    
        \end{framed}

        Substitution of the propagators into the other terms is trivial. The complete list of $2^{nd}$ order S-operator terms of fermion and photon propagators is as follows

        \begin{equation}
            S^{(2)} = S^{(2)}_{0} + S^{(2)}_{1} + S^{(2)}_{2} + S^{(2)}_{3}
        \end{equation}

        \begin{framed}
            \begin{center}
                Zero Contraction Term
            \end{center}

            \begin{equation}
                S^{(2)}_{0} = \frac{(-ie)^2}{2!} \int :\overline{\psi} (x_1) \slashed{A}_{\mu} (x_1) \psi (x_1) \overline{\psi} (x_2) \slashed{A}_{\nu} (x_2) \psi (x_2): d^{4} x_{1} d^{4} x_{2}
            \end{equation}

            \begin{center}
                One Contraction Term
            \end{center}

            \begin{equation}
                S^{(2)}_{1,1} +  S^{(2)}_{1,3} = \frac{(-ie)^2}{2!} \int :\overline{\psi} (x_1) \slashed{A} (x_1) \psi (x_1) i S_{F} (x_{2} - x_{1}) \slashed{A}_{\mu} (x_2) \psi (x_2): d^{4} x_{1} d^{4} x_{2}
            \end{equation}

            \begin{equation}
                S^{(2)}_{1,2} = \frac{(-ie)^2}{2!} \int :\overline{\psi} (x_1) \gamma^{\mu} \psi (x_1) i D^{F}_{\mu \nu} (x_{2} - x_{1}) \overline{\psi} (x_2) \gamma^{\nu} \psi (x_2): d^{4} x_{1} d^{4} x_{2}
            \end{equation}

            \begin{center}
                Two Contraction Term
            \end{center}

            \begin{equation}
                S^{(2)}_{2,1} + S^{(2)}_{2,2} = \frac{(-ie)^2}{2!} \int :\overline{\psi} (x_1) \gamma^{\mu} \psi (x_1) i S_{F} (x_{2} - x_{1}) \gamma^{\nu} i D^{F}_{\mu \nu} (x_{2} - x_{1}) \psi (x_2): d^{4} x_{1} d^{4} x_{2}
            \end{equation}

            \begin{equation}
                S^{(2)}_{2,3} = \frac{(-ie)^2}{2!} \int :Tr[i S_{F} (x_{2} - x_{1}) \gamma^{\mu} A_{\mu} i S_{F} (x_{2} - x_{1}) \gamma^{\nu} A_{\nu}]: d^{4} x_{1} d^{4} x_{2}
            \end{equation}

            \begin{center}
                Three Contraction Term
            \end{center}

            \begin{equation}
                S^{(2)}_{3} = \frac{(-ie)^2}{2!} \int :Tr[i S_{F} (x_{2} - x_{1}) \gamma^{\mu} i S_{F} (x_{2} - x_{1}) \gamma^{\nu}: d^{4} x_{1} d^{4} x_{2}
            \end{equation}

        \end{framed}

        This is the S-operator expansion to the $2^{nd}$ order that we need. Now, we will consider the physical interpretation of field operators and propagators that appear
        in the S-operator terms. Then we can finally return to the first order term, and prove that its contribution vanishes before moving any further consideration of the
        second order terms.
        
    \section{Interpretation of Field Operators and Propagators}

    Let's now consider the fields themselves. These are interacting quantum fields in the interacting picture, so they have a mathematical form of Heisenberg Free Fields:

    \begin{equation}
        A_{\mu} (x) = \int \frac{1}{2 \omega} (\epsilon^{\lambda}_{\mu} a_{\lambda} (\vec{k}) e^{i k x} + \epsilon^{\lambda}_{\mu} a_{\lambda}^{\dag} (\vec{k}) e^{-i k x} ) \frac{d^3k}{(2 \pi)^2}
    \end{equation}

    \begin{equation}
        \psi (x) = \sum_{s} \int \frac{m}{\omega} [c_{s} (\vec{p}) u_{s} (\vec{p}) e^{ip \cdot x} + d_{s}^{\dagger} (\vec{p}) \bar{v}_{s} (\vec{p}) e^{ip \cdot x}] \frac{d^3k}{(2 \pi)^2}
    \end{equation}

    \begin{equation}
        \bar{\psi} (x) = \sum_{s} \int \frac{m}{\omega} [d_{s} (\vec{p}) \bar{v}_{s} (\vec{p}) e^{ip \cdot x} + c_{s}^{\dagger} (\vec{p}) u_{s} (\vec{p}) e^{ip \cdot x}] \frac{d^3k}{(2 \pi)^2}
    \end{equation}

    From the free field theory, $c^{\uparrow}_{s} (c_{s})$ creates(annihilates) an electron with a specific momentum, and $d^{\uparrow}_{s} (d_{s})$ creates (annihilates) a
    positron with a specific momentum. In addition, we know that for the photon, $a^{\uparrow}_{s} (a_{s})$ creates (annihilates) a photon with a specific momentum. Because
    these states are all particles with a specific momentum, they are all plane wave states. If one were to apply these Fourier coefficient operators times its associated
    spinor and normalization factor to a vacuum, and then project it onto position space, one would obtain the plane wavefunction for a particle of that momentum. 

    The quantum field terms simply inverse Fourier transforms (Fourier transform) into position space of these plane wavefunction creating (annihilating) operators. Therefore
    they have all the same effect as they did before the Fourier transform was applied, but at specific momentum. Putting this into table, we have:

    \begin{framed}
        \begin{tabular}{c c c}
            $\psi = \psi^{+} + \psi^{-}$ & $\bar{\psi} = \bar{\psi}^{+} + \bar{\psi}^{-}$ & $A_{\mu} = A_{\mu}^{+} + A_{\mu}^{-}$ \\[0.5cm]
            $\overbrace{\psi^{+} = \sum_{s} \int \frac{m}{\omega} c_{s} (\vec{p}) u_{s} (\vec{p}) e^{ip \cdot x} \frac{d^3k}{(2 \pi)^2}}^{\text{annihilates an electron at x}}$ & $\overbrace{\bar{\psi}^{+} = \sum_{s} \int \frac{m}{\omega} d_{s} (\vec{p}) \bar{v}_{s} (\vec{p}) e^{ip \cdot x} \frac{d^3k}{(2 \pi)^2}}^{\text{annihilates a positron at x}}$ & $\overbrace{A_{\mu}^{+} = \int \frac{1}{2 \omega} \epsilon^{\lambda}_{\mu} a_{\lambda}^{\dag} (\vec{k}) e^{-i k x} \frac{d^3k}{(2 \pi)^2}}^{\text{annihilates a photon at x}}$ \\[0.5cm]
            $\overbrace{\psi^{-} = \sum_{s} \int \frac{m}{\omega} d_{s}^{\dagger} (\vec{p}) \bar{v}_{s} (\vec{p}) e^{ip \cdot x} \frac{d^3k}{(2 \pi)^2}}^{\text{creates a positron at x}}$ & $\overbrace{\bar{\psi}^{-} = \sum_{s} \int \frac{m}{\omega} c_{s}^{\dagger} (\vec{p}) u_{s} (\vec{p}) e^{ip \cdot x}  \frac{d^3k}{(2 \pi)^2}}^{\text{creates an electron at x}}$ & $\overbrace{A_{\mu}^{-} = \int \frac{1}{2 \omega} \epsilon^{\lambda}_{\mu} a_{\lambda} (\vec{k}) e^{i k x}  \frac{d^3k}{(2 \pi)^2}}^{\text{creates a photon at x}}$
        \end{tabular}
    \end{framed}

    With these sorted, we can now finally get back to showing that the first order term doesn't yield any nonvanishing contribution.

    \section{Proof that First Order Term Always Vanishes}

    \begin{framed}
        For all the key processes, we can see quite quickly that the first order term in the S-operator expansion can't contribute to anything nonvanishing. This is because of
        an inevitable mismatch between the contents of the initial and final states, and the operators in $S^{(1)}$. One of the states would inevitably get annihilated.
        However, one can conceive of other physical processes for which one might expect the first order term to yield a nonvanishing contribution. I will show in this section
        that even for those processes, the first order contribution is zero.

        Let's recall the form of $S^{(1)}$ from earlier:

        \begin{equation}
            S^{(1)} = - i e \int :\bar{\psi} \slashed{A} \psi: d^{4} x
        \end{equation}

        We can now insert the two break down of the quantum fields into it. Doing that yields the term like following:

        \begin{equation}
            S_{1}^{(1)} = - i e \int :\bar{\psi}^{-} (x) \gamma^{\mu} A_{\mu}^{-} (x) \psi^{+} (x):
        \end{equation}

        We will show that even the most natural matrix element of this part of $S^{(1)}$ vanishes as an example. All the other terms in $S^{(1)}$ can be shown to vanish via essentially
        identical calculations. Using the field operator interpretations that we have established, we can see that this operator will annihilate an electron at x. This is described by
        the following Feynman Diagram: 
        
        \begin{center}

            \begin{tikzpicture}
                \begin{feynman}
                    \vertex (a);
                    \vertex [above right = of a] (b);
                    \vertex [below right = of a] (c);
                    \vertex [left = of a] (d);
    
                    \diagram{
                        (b) -- [boson, edge label = $\gamma$] (a) -- [fermion, edge label = $e^{-}$] (c);
                        (d) -- [fermion, edge label = $e^{-}$] (a);
                    };
                \end{feynman}
            \end{tikzpicture}

        \end{center}

        In $S^{(1)}$, $:\bar{\psi}^{-} (x) \gamma^{\mu} A_{\mu}^{-} (x) \psi^{+} (x):$ is integrated over all spacetime points to give its complete contribution to the scattering
        operator, and by extension the scattering matrix. This integration makes sense because this interaction could potentially happen anywhere. Each spacetime point could therefore
        be expected to contribute to the probability amplitude for any transition that coud possibly mediated by this proccess. Of course, as I have said repeatedly, $S^{(1)}$
        actually doesn't give a nonvanishing contribution to anything , in the end. However, without knowing this, the proccess that we would otherwise most expect this part of $S^{(1)}$
        to yield a non-zero contribution to, is the following:
        
        \begin{equation}
            \begin{aligned}[]
                | in \rangle = | e^{-} \rangle \\
                | out \rangle = | e^{-}, \gamma \rangle
            \end{aligned}
        \end{equation}

        \begin{equation}
            e^{-} \rightarrow \gamma e^{-}
        \end{equation}

        So the matrix element of interest is:

        \begin{equation}
            \langle e^{-}, \gamma | s_{1}^{(1)} | e^{-} \rangle = - i e \int \langle e^{-}, \gamma | \bar{\psi}^{-} (x) \gamma^{\mu} A_{\mu}^{-} (x) \psi^{+} (x) | e^{-} \rangle
        \end{equation}

        In order to show that this matrix element is zero, we need some equations previously presented in this document, and the cannonical commutation and anticommutation relations
        for the Fourier coefficient operators. The latter are as follows:

        \begin{equation}
            \begin{aligned}[]
                [a_{\lambda} (\vec{k}), a^{\dag}_{\lambda'} (\vec{k'}) ] = (2 \pi)^{3} 2 \omega \eta{\lambda \lambda'} \delta^{3} (\vec{k} - \vec{k'}) \\
                \{c_{r} (\vec{p}), c^{\dag}_{s} (\vec{p'}) \} = (2 \pi)^{3} \frac{\omega}{m} \delta^{3} (\vec{p} - \vec{p'}) \delta_{rs} \\
                \{d_{r} (\vec{p}), d^{\dag}_{s} (\vec{p'}) \} = (2 \pi)^{3} \frac{\omega}{m} \delta^{3} (\vec{p} - \vec{p'}) \delta_{rs}
            \end{aligned}
        \end{equation}

        All the other femionic operators are zero, all the other bosonic commutators are zero, and all commutators of fermionic operators with bosonic ones are zero. The first step
        in the calculation is to derive a couple of identities for later use. Namely, we want to evaluate the the following quantities:

        \begin{equation}
            \psi^{+} | e^{-} \rangle \quad A_{\mu}^{+} | \gamma \rangle
        \end{equation}

        Let's start with the first one. Recall the following two facts:

        \begin{equation}
            | e^{+} \rangle = c_{s}^{\dag} (p) | 0 \rangle \quad \psi^{+} (x) = \sum_{s} \int \frac{m}{\omega} c_{s} (\vec{p}) u_{s} (\vec{p}) e^{-i p \cdot x} \frac{d^3k}{(2 \pi)^2 2 \omega}
        \end{equation}

        Inserting it into these quantity that we wish to evaluate gives:

        \begin{equation}
            \begin{aligned}
                \psi^{+} (x) | e^{-} \rangle = \sum_{r} \int \frac{m}{\omega} c_{r} (\vec{k}) u_{r} (\vec{k}) e^{-i p \cdot x} c_{s}^{\dag} (\vec{p}) \frac{d^3 p}{(2 \pi)^2 2 \omega} \\
                = \int \frac{m}{\omega} e^{-i p \cdot x} \sum_{r} u_{r} (\vec{k}) c_{r} (\vec{k}) c_{s}^{\dag} (\vec{p})  | 0 \rangle \frac{d^3 p}{(2 \pi)^2 2 \omega}
            \end{aligned}
        \end{equation}

        We can now rewrite the operator product in terms of an anticommutator:

        \begin{equation}
            c_{r} (\vec{k}) c_{s}^{\dag} (\vec{p}) | 0 \rangle = [\{ c_{r} (\vec{k}) , c_{s}^{\dag} (\vec{p}) \} - c_{s}^{\dag} (\vec{p}) c_{r} (\vec{k})] | 0 \rangle
        \end{equation}

        Annihilation operators acting on a vacuum give zero, so:

        \begin{equation}
            c_{r} (\vec{k}) c_{s}^{\dag} (\vec{p}) | 0 \rangle = [\{ c_{r} (\vec{k}) , c_{s}^{\dag} (\vec{p}) \}] | 0 \rangle \frac{d^3 p}{(2 \pi)^2 2 \omega}
        \end{equation}

        Therefore:

        \begin{equation}
            \psi^{+} (x) | e^{-} \rangle = \int \frac{m}{\omega} e^{-i p \cdot x} \sum_{r} u_{r} (\vec{k}) \{ c_{r} (\vec{k}) , c_{s}^{\dag} (\vec{p}) \} | 0 \rangle
        \end{equation}

        We can now remember the anticommtation relation from above:
        
        \begin{equation}
            \begin{aligned}[]
                \{ c_{r} (\vec{p}), c_{s}^{\dagger} (\vec{p}) \} = (2 \pi)^3 \frac{\omega}{m} \delta^{3} (\vec{p} - \vec{p'}) \delta_{rs}
            \end{aligned}
        \end{equation}

        In the k-integration, the time phase factor becomes dependent on the energy associated with $\vec{p}$ instead of that of $\vec{k}$, because the energy is a function
        of 3-Momentum, which is usually set to $\vec{p}$ by the delta function. Our final result is therefore:

        \begin{framed}
            \begin{equation}
                \psi^{+} (x) | e^{-} \rangle = u_{s} (\vec{p}) e^{i p \cdot x} | 0 \rangle
            \end{equation}
        \end{framed}

        Now let's evaluate the quantity we wanted an identity for:

        \begin{equation}
            A_{\mu}^{+} (x) | \gamma \rangle = A_{\mu}^{+} (x) a_{\lambda}^{\dagger} (\vec{k}) | 0 \rangle
        \end{equation}

        The formula for $A_{\mu}^{+} (x)$ is:

        \begin{equation}
            A_{\mu}^{+} (x) = \int \epsilon^{\lambda}_{\mu} a_{\lambda} (\vec{k}) e^{- i k \cdot x} \frac{d^3 p}{(2 \pi)^2 2 \omega}
        \end{equation}

        Inserting this gives:

        \begin{equation}
            A_{\mu}^{+} (x) | \gamma \rangle = \int \epsilon^{\lambda'}_{\mu} a_{\lambda'} (\vec{k}) a_{\lambda}^{\dagger} (\vec{k}) e^{- i k' \cdot x} \frac{d^3 p}{(2 \pi)^2 2 \omega} | 0 \rangle
        \end{equation}

        Using the same rational as in the previous calculation, we can substitute in a commutator:

        \begin{equation}
            A_{\mu}^{+} (x) | \gamma \rangle = \int \epsilon^{\lambda'}_{\mu} [a_{\lambda'} (\vec{k}), a_{\lambda}^{\dagger} (\vec{k})] e^{- i k' \cdot x} \frac{d^3 p}{(2 \pi)^2 2 \omega} | 0 \rangle
        \end{equation}

        \begin{equation}
            [a_{\lambda'} (\vec{k}), a_{\lambda}^{\dagger} (\vec{k})] = - (2 \pi)^{3} 2 \omega \eta_{\lambda \lambda'} \delta^{3} (\vec{k} - \vec{k'})
        \end{equation}

        \begin{equation}
            A_{\mu}^{+} (x) | \gamma \rangle = \int \epsilon^{\lambda'}_{\mu} \eta_{\lambda \lambda'} e^{- i k' \cdot x} \delta^{3} (\vec{k} - \vec{k'}) \frac{d^3 p}{(2 \pi)^2 2 \omega} | 0 \rangle = - \epsilon^{\lambda'}_{\mu} \eta_{\lambda \lambda'} e^{- i k' \cdot x} | 0 \rangle
        \end{equation}

        If we take $| \gamma \rangle$ to be a transverse state, then the longitudinal state, then the longitudinal and temporal annihilation operator terms \big( in the sum in $A_{\mu}^{+} (x)$ \big) will just annihilate
        $| \gamma \rangle$, and they therefore vanish from the sum. This reduces the $\lambda'$ to a two-term sum where $\eta_{\lambda \lambda'}$ is replaced with a $- \delta_{\lambda \lambda'}$, where the indices on the
        Kronecker delta are now two dimensional. This switch to the two dimensional Kronecker delta also means that upper and lower indices are identical, and can be used interchangably. This allows the sum $\lambda'$ to
        be performed:

        \begin{equation}
            A_{\mu}^{+} (x) | \gamma \rangle = \epsilon^{\lambda'}_{\mu} \delta_{\lambda \lambda'} e^{- i k' \cdot x} | 0 \rangle = \epsilon^{\lambda'}_{\mu} e^{- i k' \cdot x} | 0 \rangle
        \end{equation}

        So, the final result is:

        \begin{framed}
            \begin{equation}
                A_{\mu}^{+} (x) | \gamma \rangle = \epsilon^{\lambda}_{\mu} e^{- i k \cdot x} | 0 \rangle
            \end{equation}
        \end{framed}

        To summarize, the two identities that we wanted are:

        \begin{equation}
            \begin{aligned}
                A_{\mu}^{+} (x) | \gamma \rangle = \epsilon^{\lambda}_{\mu} e^{- i k \cdot x} | 0 \rangle \\
                \psi^{+} (x) | e^{-} \rangle = u_{s} (\vec{p}) e^{i p \cdot x} | 0 \rangle
            \end{aligned}
        \end{equation}

        These same identities will also be used for second order matrix evaluations later. So now let's use these
        identities to evaluate target amplitude:

        \begin{equation}
            \begin{aligned}
                \langle e^{-} (p', s') \gamma (k', \lambda') | s^{(1)} | e^{-} (p, s) \rangle = \\
                \langle e^{-} (p', s') \gamma (k', \lambda') | [- i e \int \bar{\psi}^{-} (x) \gamma^{\mu} A_{\mu}^{-} (x) \psi^{+} (x) d^{4} (x)] | e^{-} (p, s) \rangle \\
                - i e \int \langle e^{-} (p', s') \gamma (k', \lambda') | [\bar{\psi}^{-} (x) \gamma^{\mu} A_{\mu}^{-} (x) \psi^{+} (x)] | e^{-} (p, s) \rangle d^{4} (x)
            \end{aligned}
        \end{equation}

        We can now do a substitution using one of our identities:

        \begin{equation}
            \psi^{+} (x) | e^{-} \rangle = u_{s} (\vec{p}) e^{- i p \cdot x} | 0 \rangle
        \end{equation}

        Doing this gives:

        \begin{equation}
            \begin{aligned}
                \langle e^{-} (p', s') \gamma (k', \lambda') | s^{(1)} | e^{-} (p, s) \rangle = \\
                - i e \int \langle e^{-} (p', s') \gamma (k', \lambda') | [\bar{\psi}^{-} (x) \gamma^{\mu} A_{\mu}^{-} (x) u_{s} (\vec{p}) e^{- i p \cdot x}] | 0 \rangle d^{4} (x)
            \end{aligned}
        \end{equation}

        Now we can deduce another identity from one of the ones that we already derived:

        \begin{equation}
            \psi^{+} (x) | e^{-} \rangle = u_{s} (\vec{p}) e^{- i p \cdot x} | 0 \rangle \rightarrow \langle e^{-} | \bar{\psi}^{-} (x) = \langle 0 | e^{i p \cdot x} \bar{u}_{s} (\vec{p})
        \end{equation}

        Inserting this gives: 

        \begin{equation}
            \begin{aligned}
                \langle e^{-} (p', s') \gamma (k', \lambda') | s^{(1)} | e^{-} (p, s) \rangle = \\
                - i e \int \langle\gamma (k', \lambda') | [e^{i p' \cdot x} \bar{u}_{s'} (\vec{p}) \gamma^{\mu} A_{\mu}^{-} (x) u_{s} (\vec{p}) e^{- i p \cdot x}] | 0 \rangle d^{4} (x)
            \end{aligned}
        \end{equation}

        \begin{equation}
            \begin{aligned}
                \langle e^{-} (p', s') \gamma (k', \lambda') | s^{(1)} | e^{-} (p, s) \rangle = \\
                - i e \int \langle\gamma (k', \lambda') | [\bar{u}_{s'} (\vec{p}) \gamma^{\mu} A_{\mu}^{-} (x) u_{s} (\vec{p}) e^{- i (p - p') \cdot x}] | 0 \rangle d^{4} (x)
            \end{aligned}
        \end{equation}

        We can similarly derive another helpful identity from the photon field identity we already derived:

        \begin{equation}
            A_{\mu}^{+} (x) | \gamma \rangle = \epsilon^{\lambda}_{\mu} e^{- i k \cdot x} | 0 \rangle \rightarrow \langle \gamma | A_{\mu}^{-} (x) = \langle 0 | \epsilon^{\lambda}_{\mu} e^{i k \cdot x}
        \end{equation}

        Inserting this gives:

        \begin{equation}
            \begin{aligned}
                \langle e^{-} (p', s') \gamma (k', \lambda') | s^{(1)} | e^{-} (p, s) \rangle = \\
                - i e \int \langle 0 | [\epsilon^{\lambda}_{\mu} e^{i k' \cdot x} \bar{u}_{s'} (\vec{p}) \gamma^{\mu} u_{s} (\vec{p}) e^{- i (p - p') \cdot x}] | 0 \rangle d^{4} (x)
            \end{aligned}
        \end{equation}

        \begin{equation}
            \begin{aligned}
                \langle e^{-} (p', s') \gamma (k', \lambda') | s^{(1)} | e^{-} (p, s) \rangle = \\
                - i e \int \langle 0 | [\epsilon^{\lambda}_{\mu} \bar{u}_{s'} (\vec{p}) \gamma^{\mu} u_{s} (\vec{p}) e^{- i (p - p' - k') \cdot x}] | 0 \rangle d^{4} (x)
            \end{aligned}
        \end{equation}

        Now, there are no operator left. Everything in the matrix element is a c-number an can be pulled out:

        \begin{equation}
            \begin{aligned}
                \langle e^{-} (p', s') \gamma (k', \lambda') | s^{(1)} | e^{-} (p, s) \rangle = \\
                - i e \int [\epsilon^{\lambda}_{\mu} \bar{u}_{s'} (\vec{p}) \gamma^{\mu} u_{s} (\vec{p}) e^{- i (p - p' - k') \cdot x}] \langle 0 | 0 \rangle d^{4} (x)
            \end{aligned}
        \end{equation}

        We can take the vacuum state to be normalized, as we will do for the rest of the video:

        \begin{equation}
            \begin{aligned}
                \langle e^{-} (p', s') \gamma (k', \lambda') | s^{(1)} | e^{-} (p, s) \rangle = \\
                - i e \int [\epsilon^{\lambda}_{\mu} \bar{u}_{s'} (\vec{p}) \gamma^{\mu} u_{s} (\vec{p}) e^{- i (p - p' - k') \cdot x}] d^{4} (x)
            \end{aligned}
        \end{equation}

        We can recognize the delta function showing up in here. Basically, doing the x integration yields a delta function. One must remember the factor of $1/(2 \pi)^{4}$ that
        is part of the definition of the delta function:

        \begin{equation}
            \begin{aligned}
                \langle e^{-} (p', s') \gamma (k', \lambda') | s^{(1)} | e^{-} (p, s) \rangle = \\
                - i e (2 \pi)^{4} \delta^{4} (p - p' - k) \epsilon^{\lambda}_{\mu} \bar{u}_{s'} (\vec{p'}) \gamma^{\mu} u_{s} (\vec{p})
            \end{aligned}
        \end{equation}

        Now we can identify the Feynman Amplitude:

        \begin{equation}
            \begin{aligned}
                \langle e^{-} (p', s') \gamma (k', \lambda') | s^{(1)} | e^{-} (p, s) \rangle = \\
                - i e (2 \pi)^{4} \delta^{4} (p - p' - k) \epsilon^{\lambda}_{\mu} \mathcal{M}_{fi}
            \end{aligned}
        \end{equation}

        The Feynman Amplitude is 

        \begin{equation}
            \mathcal{M}_{fi} = - i e \bar{u}_{s'} (\vec{p'}) \epsilon^{\lambda}_{\mu} \gamma^{\mu} u_{s} (\vec{p})
        \end{equation}

        Now, the we see this delta function showing up: $\delta^{4} (p - p' - k)$, we can answer the question of whether or not it's argument can actually be zero. This delta
        function forces the following constraint:

        \begin{equation}
            p'_{\mu} + k'_{\mu} = p_{\mu}
        \end{equation}

        Which gives the following energy relation:

        \begin{equation}
            \begin{aligned}
                p'_{0} + k'_{0} = p_{0} \\
                \sqrt{|\vec{p'}|^{2} + m^{2}} + |\vec{k}| = \sqrt{|\vec{p'}|^{2} + m^{2}}
            \end{aligned}
        \end{equation}

        And the following three-momentum relation:

        \begin{equation}
            \vec{p'} + \vec{k'} = \vec{p}
        \end{equation}

        We can take the rest frame of the incoming electron without loss of generality because the s-matrix is Lorentz Invariant:

        \begin{equation}
            \begin{aligned}
                \sqrt{|\vec{p'}|^{2} + m^{2}} + |\vec{k'}| = \pm m \\
                \vec{p'} + \vec{k'} = \vec{p}
            \end{aligned}
        \end{equation}

        The momentum relation implies: 

        \begin{equation}
            \vec{p'} = \vec{k'}
        \end{equation}

        Let's insert this into the energy relation:

        \begin{equation}
            \sqrt{|\vec{p'}|^{2} + m^{2}} + |\vec{p'}| = \pm m
        \end{equation}

        The equality is only satisfied when $\vec{p} + \vec{k} = 0$. Therefore, the delta function only won't zero the first order contribution 
        tp the scattering matrix if both particles don't leave with zero momentum when in the rest frame of the initial electron. In other words,
        the emmitted photon doesn't exists, and the initial electron remains unaffected by interaction. Lorentz Invariance of the scattering
        matrix guarantees this effect if not specific to the rest frame, but it is completely general. Matrix elements of all of the other terms
        in $s^{1}$ vanish for the same reason. The calculations of all of them are essentially identical to the one presented here.

    \end{framed}

    Now that we have completed our discussion on the $s^{1}$ matrix elements, we can move on further discussing the second order terms. The first
    step is to simplify our discussion by figuring out exactly which second order S-operator terms yield nonvanishing contributions to which matrix
    elements from the list of physical processes that we care about (see introduction). We can then ignore all of the rest of the S-operator terms.
    This step comprises of section 5. Once this is done, the following step is to use already discussed interpretations of the field operators to
    associate Feynman Diagrams with each contributing second order matrix element term (section 6). These terms must then be computed further, to
    the point where the Feynman amplitudes can't be identified (section 7 and 8), then we compare the Feynman amplitudes to their associated Feynman
    Diagrams to reveal the Feynman rules (section 9).

    \section{Which scattering Operator Terms Contribute to Each Scattering Matrix Elements}

    When we take the matrix elements of the S-operator for the value processes, many of the terms in the second-order S-matrix will vanish. The first
    reason why some S-operator terms will have vanishing matrix elements is that, for a given process, the initial or final state may not be 
    exclusively populated with a set of particles matching the annihilation operators that show up in a particular S-operator term. Annihilation 
    operators zero states not containing their corresponding particle type in the matching quantum state. This is actually by far the most common 
    reason why S-operator terms may yield vanishing matrix element terms. There is, however one other reason. $S^{(2)}_{3}$ has no operators in it at
    all, but it yields vanishing contributions to all but the vacuum to vacuum transition. What happens is this: even when the particle matches in the
    initial and final states, their quantum states won't, because we are ignoring fowrward scattering. Therefore, the matrix element contribution will
    vanish because of orthogonality.

    As it happens, we will find that most terms don't contribute to any of the processes that we are considering. The goal of this section is to figure
    out which S-operator terms have a chance of giving a nonzero contribution to each matrix element. To begin this process let's consider the following
    S-operator term:

    \begin{framed}
        \begin{equation}
            S^{(2)}_{1,1} + S^{(2)}_{1,3} = - e^{2} \int :\overline{\psi} (x_1) \gamma^{\mu} A_{\mu} (x_1) i S_{F} (x_2 - x_1) \gamma^{\nu} A_{\nu} (x_2) \psi (x_2): d^{4} x_{1} d^{4} x_{2}
        \end{equation}
    \end{framed}

    Remember from section 3 that these field operators break apart into creation and annihilation parts that create and annihilate particles that localized
    at a specific spacetime point. Also, remember that propagators propagate a virtual particle between two space time points. Inspection of Eq. A therefore
    tells us that this term in the S-operator has the possibility to contribute to any proccess that has two photons in the initial or final state, and two
    fermions (electrons and positrons) in the initial and final state. Of the proccesses of interest, we therefore expect this term to contribute to the
    elements describing the Compton Scattering of both types, pair creation and pair annihilation. By the same arguments, we see that Eq. A cannot contribute
    to any of the other processes. The matrix elements simply vanish because it gets annihilated. 

    To find out which specific terms from Eq. A contribute to each matrix element, we must insert the fields as a sum of creation and annihilation parts,

    \begin{equation}
        \psi (x) = \psi^{+} (x) + \psi^{-} (x) \quad \bar{\psi} (x) = \bar{\psi}^{+} (x) + \bar{\psi}^{-} (x) \quad A_{\mu} = A_{\mu}^{+} + A_{\mu}^{-}
    \end{equation}

    And them multiply them out. Each quantum field is a sum of two parts, so we expect Eq. A to produce 16 terms. We can then decide whether or not a given
    one of the 16 terms contributes to each matrix element, by looking for annihilation operators, in the S-operator terms that are of a particle type that
    is not present in the state in which it is being applied. If there is no such case, then the particular S-operator term in question probably contributes
    to the given matrix element. I say probably because in the case of first order contribution, we found the matrix elements to be still vanishing. This
    will not turn out to be the case for the second order matrix elements.

    In doing this inspection, it is important to remember that creation operators, when acting left, are annihilation operators. For each of the four
    processes that Eq. A has the potential to contribute to, this inspection yields TABLE A:

    \begin{center}
        TABLE A
    \end{center}

    \begin{framed}
        \begin{tabular}{c c c c}
            PROCESS & REACTION & \shortstack{General \\ S-Matrix \\ Element} & \shortstack{Contributing Eq. A \\ Integrand \\ Terms} \\
            \vspace{0.1mm}
            $e^{+}$ Compton Scattering & $\gamma + e^{-} \rightarrow \gamma + e^{-}$ & $\langle \gamma , e^{-} | S | \gamma , e^{-} \rangle$ & $:\overline{\psi}^{-} (x_1) \gamma^{\mu} A^{-}_{\mu} (x_1) S_{F} (x_{2} - x_{1}) A^{+}_{\mu} (x_2) \gamma^{\nu} \psi^{+} (x_2):$ \\
            & & & $:\overline{\psi}^{-} (x_1) \gamma^{\mu} A^{+}_{\mu} (x_1) S_{F} (x_{2} - x_{1}) A^{-}_{\mu} (x_2) \gamma^{\nu} \psi^{-} (x_2):$ \\
            $e^{-}$ Compton Scattering & $\gamma + e^{+} \rightarrow \gamma + e^{+}$ & $\langle \gamma , e^{+} | S | \gamma , e^{+} \rangle$ & $:\overline{\psi}^{+} (x_1) \gamma^{\mu} A^{+}_{\mu} (x_1) i S_{F} (x_{2} - x_{1}) A^{-}_{\mu} (x_2) \gamma^{\nu} \psi^{+} (x_2):$ \\
            & & & $:\overline{\psi}^{+} (x_1) \gamma^{\mu} A^{-}_{\mu} (x_1) S_{F} (x_{2} - x_{1}) A^{+}_{\mu} (x_2) \gamma^{\nu} \psi^{-} (x_2):$ \\
            pair annihilation & $e^{-} + e^{+} \rightarrow \gamma + \gamma$ & $\langle e^{-} , e^{+} | S | \gamma , \gamma \rangle$ & $:\overline{\psi}^{-} (x_1) A^{-}_{\mu} (x_1) i S_{F} (x_{2} - x_{1}) A^{-}_{\mu} (x_2) \gamma^{\nu} \psi^{-} (x_2):$ \\
            pair production & $\gamma + \gamma \rightarrow e^{-} + e^{+}$ & $\langle \gamma , \gamma | S | e^{-} , e^{+} \rangle$ & $:\overline{\psi}^{+} (x_1) A^{+}_{\mu} (x_1) i S_{F} (x_{2} - x_{1}) A^{+}_{\mu} (x_2) \gamma^{\nu} \psi^{+} (x_2):$ \\
        \end{tabular}
    \end{framed}

    If it isn't clear, one can simply insert these integrand terms back in Eq. A to get some S-operator terms which yield a non-vanishing matrix element
    contribution. See the table in the next section to see this done for all the processes that we are discussing in this video.

    The 10 remaining terms in the integrand of Eq. A don't contribute to any of the processes under consideration in this video. This is becase they all
    contain annihilation operators that would annihilate either the initial or final state in every matrix element we are considering. 

    These are the 10 terms that follow:

    \begin{framed}
        \begin{tabular}{c c}
            3 Particles in, 1 Particle out & 1 Particle in, 3 Particles out \\
            $:\overline{\psi}^{-} (x_1) \gamma^{\mu} A^{+}_{\mu} (x_1) S_{F} (x_{2} - x_{1}) A^{+}_{\mu} (x_2) \gamma^{\nu} \psi^{+} (x_2):$ & $:\overline{\psi}^{+} (x_1) \gamma^{\mu} A^{-}_{\mu} (x_1) S_{F} (x_{2} - x_{1}) A^{-}_{\mu} (x_2) \gamma^{\nu} \psi^{-} (x_2):$ \\
            $:\overline{\psi}^{+} (x_1) \gamma^{\mu} A^{-}_{\mu} (x_1) S_{F} (x_{2} - x_{1}) A^{+}_{\mu} (x_2) \gamma^{\nu} \psi^{+} (x_2):$ & $:\overline{\psi}^{-} (x_1) \gamma^{\mu} A^{+}_{\mu} (x_1) S_{F} (x_{2} - x_{1}) A^{-}_{\mu} (x_2) \gamma^{\nu} \psi^{-} (x_2):$\\
            $:\overline{\psi}^{+} (x_1) \gamma^{\mu} A^{+}_{\mu} (x_1) S_{F} (x_{2} - x_{1}) A^{-}_{\mu} (x_2) \gamma^{\nu} \psi^{+} (x_2):$ & $:\overline{\psi}^{-} (x_1) \gamma^{\mu} A^{-}_{\mu} (x_1) S_{F} (x_{2} - x_{1}) A^{+}_{\mu} (x_2) \gamma^{\nu} \psi^{-} (x_2):$\\
            $:\overline{\psi}^{-} (x_1) \gamma^{\mu} A^{+}_{\mu} (x_1) S_{F} (x_{2} - x_{1}) A^{+}_{\mu} (x_2) \gamma^{\nu} \psi^{-} (x_2):$ & $:\overline{\psi}^{-} (x_1) \gamma^{\mu} A^{-}_{\mu} (x_1) S_{F} (x_{2} - x_{1}) A^{-}_{\mu} (x_2) \gamma^{\nu} \psi^{+} (x_2):$ \\
            0 Particles in, 4 Particles out & 4 Particles in, 0 Particles out \\
            $:\overline{\psi} (x_1) \gamma^{\mu} A^{+}_{\mu} (x_1) S_{F} (x_{2} - x_{1}) A^{-}_{\mu} (x_2) \gamma^{\nu} \psi (x_2):$ & $:\overline{\psi} (x_1) \gamma^{\mu} A^{+}_{\mu} (x_1) S_{F} (x_{2} - x_{1}) A^{-}_{\mu} (x_2) \gamma^{\nu} \psi (x_2):$
        \end{tabular}
    \end{framed}

    We have identified that $S^{(2)}_{1,1} + S^{(2)}_{1,3}$ contributes to the matrix elements of the processes of TABLE A, but we can go a step further.
    We can see that there are no other term there is a possibility between matchup between the particle content of the initial and final states, and the
    term's annihilation operators for any TABLE A proccess. They therefore give vanishing contribution to the matrix element. $S^{(2)}_{3}$ gives a
    vanishing contribution because of the orthogonality argument already given. Let's now perform the same inspection analysis with another second order
    term. Specifically, let's consider $S^{(2)}_{1,2}$:
    
    \begin{framed}
        \begin{equation}
            S^{(2)}_{1,2} = = \frac{(-ie)^2}{2!} \int :\overline{\psi} (x_1) \gamma^{\mu} \psi (x_1) D_{F}^{\mu\nu} (x_2 - x_1) \overline{\psi} (x_2) D \gamma^{\nu} \psi (x_2): d^{4} x_{1} d^{4} x_{2}
        \end{equation}
    \end{framed}

    This term clearly has the capacity to contribute processes that involve two fermions scattering off to each other, and will yield a vanishing contribution
    to any other processes we are considering. By again inserting:

    \begin{equation}
        \psi (x) = \psi^{+} (x) + \psi^{-} (x) \quad \bar{\psi} (x) = \bar{\psi}^{+} (x) + \bar{\psi}^{-} (x)
    \end{equation}

    and then remembering the interpretation of individual quantum fields, the inspection analysis yields the results expressed in TABLE B:

    \begin{center}
        TABLE B
    \end{center}

    \begin{framed}
        \begin{tabular}{c c c c}
            PROCESS & REACTION & \shortstack{General \\ S-Matrix \\ Element} & \shortstack{Contributing Eq. B \\ Integrand \\ Terms} \\
            $e^{-}$ Moller Scattering & $e^{-} + e^{-} \rightarrow e^{-} + e^{-}$ & $\langle e^{-} , e^{-} | S | e^{-} , e^{-} \rangle$ & $:\overline{\psi}^{-} (x_1) \gamma^{\mu} \psi^{+} (x_1) D_{F}^{\mu\nu} (x_2 - x_1) \overline{\psi}^{-} (x_2) D \gamma^{\nu} \psi^{+} (x_2):$ \\
            $e^{+}$ Moller Scattering & $e^{+} + e^{+} \rightarrow e^{+} + e^{+}$ & $\langle e^{+} , e^{+} | S | e^{+} , e^{+} \rangle$ & $:\overline{\psi}^{+} (x_1) \gamma^{\mu} \psi^{-} (x_1) D_{F}^{\mu\nu} (x_2 - x_1) \overline{\psi}^{+} (x_2) D \gamma^{\nu} \psi^{-} (x_2):$ \\
            & & & $:\overline{\psi}^{-} (x_1) \gamma^{\mu} \psi^{-} (x_1) D_{F}^{\mu\nu} (x_2 - x_1) \overline{\psi}^{+} (x_2) D \gamma^{\nu} \psi^{+} (x_2):$ \\
        Bhabha Scattering & $e^{-} + e^{+} \rightarrow e^{-} + e^{+}$ & $\langle e^{-} , e^{+} | S | e^{-} , e^{+} \rangle$ & $:\overline{\psi}^{+} (x_1) \gamma^{\mu} \psi^{+} (x_1) D_{F}^{\mu\nu} (x_2 - x_1) \overline{\psi}^{-} (x_2) D \gamma^{\nu} \psi^{-} (x_2):$ \\
            & & & $:\overline{\psi}^{-} (x_1) \gamma^{\mu} \psi^{+} (x_1) D_{F}^{\mu\nu} (x_2 - x_1) \overline{\psi}^{+} (x_2) D \gamma^{\nu} \psi^{-} (x_2):$ \\
            & & & $:\overline{\psi}^{+} (x_1) \gamma^{\mu} \psi^{-} (x_1) D_{F}^{\mu\nu} (x_2 - x_1) \overline{\psi}^{-} (x_2) D \gamma^{\nu} \psi^{+} (x_2):$
        \end{tabular}
    \end{framed}

    It is important to take note that there aren't four distinct contributions to Bhabha Scattering, from Eq. B. When these terms are inserted back into the
    integral in Eq. B, one obtains 2 pairs of identical terms, because two pairs of integrand terms become identical upon interchanging the dummy integration
    variables in one term each pair. Remember, the integrations are all over the space, so both integrations are identical. This is why the integration
    variables are interchangable. One can therefore much more compactly write the Bhabha scattering entry as follows:

    \begin{framed}
        \begin{tabular}{c c c c}
            PROCESS & REACTION & \shortstack{General \\ S-Matrix \\ Element} & \shortstack{Contributing Eq. B \\ Integrand \\ Terms} \\
            Bhabha Scattering & $e^{-} + e^{+} \rightarrow e^{-} + e^{+}$ & $\langle e^{-} , e^{+} | S | e^{-} , e^{+} \rangle$ & $2 :\overline{\psi}^{-} (x_1) \gamma^{\mu} \psi^{-} (x_1) D_{F}^{\mu\nu} (x_2 - x_1) \overline{\psi}^{+} (x_2) D \gamma^{\nu} \psi^{+} (x_2):$ \\
            & & & $2 :\overline{\psi}^{-} (x_1) \gamma^{\mu} \psi^{+} (x_1) D_{F}^{\mu\nu} (x_2 - x_1) \overline{\psi}^{+} (x_2) D \gamma^{\nu} \psi^{-} (x_2):$
        \end{tabular}
    \end{framed}

    just like when the first term we analyzed, the 10 remaining terms that come from multiplying out Eq. B don't contribute to any of the processes under
    consideration in this video. (in fact, they contribute to nothing generally) for exactly the same reason as before. These ten remaining terms are:

    \begin{framed}
        \begin{tabular}{c c}
            3 Particles in, 1 Particle out & 1 Particle in, 3 Particles out \\
            $:\overline{\psi}^{-} (x_1) \gamma^{\mu} \psi^{+} (x_1) D^{F}_{\mu\nu} (x_{2} - x_{1}) \psi^{+} (x_2) \gamma^{\nu} \psi^{+} (x_2):$ & $:\overline{\psi}^{+} (x_1) \gamma^{\mu} \overline{\psi}^{-} (x_1) D^{F}_{\mu\nu} (x_{2} - x_{1}) \overline{\psi}^{-} (x_2) \gamma^{\nu} \psi^{-} (x_2):$ \\
            $:\overline{\psi}^{+} (x_1) \gamma^{\mu} \psi^{-} (x_1) D^{F}_{\mu\nu} (x_{2} - x_{1}) \psi^{+} (x_2) \gamma^{\nu} \psi^{+} (x_2):$ & $:\overline{\psi}^{-} (x_1) \gamma^{\mu} \overline{\psi}^{+} (x_1) D^{F}_{\mu\nu} (x_{2} - x_{1}) \overline{\psi}^{-} (x_2) \gamma^{\nu} \psi^{-} (x_2):$\\
            $:\overline{\psi}^{+} (x_1) \gamma^{\mu} \psi^{+} (x_1) D^{F}_{\mu\nu} (x_{2} - x_{1}) \psi^{-} (x_2) \gamma^{\nu} \psi^{+} (x_2):$ & $:\overline{\psi}^{-} (x_1) \gamma^{\mu} \overline{\psi}^{-}(x_1) D^{F}_{\mu\nu} (x_{2} - x_{1}) \overline{\psi}^{+} (x_2) \gamma^{\nu} \psi^{-} (x_2):$\\
            $:\overline{\psi}^{-} (x_1) \gamma^{\mu} \psi^{+} (x_1) D^{F}_{\mu\nu} (x_{2} - x_{1}) \psi^{+} (x_2) \gamma^{\nu} \psi^{-} (x_2):$ & $:\overline{\psi}^{-} (x_1) \gamma^{\mu} \overline{\psi}^{-} (x_1) D^{F}_{\mu\nu} (x_{2} - x_{1}) \overline{\psi}^{-} (x_2) \gamma^{\nu} \psi^{+} (x_2):$ \\
            0 Particles in, 4 Particles out & 4 Particles in, 0 Particles out \\
            $:\overline{\psi} (x_1) \gamma^{\mu} \psi^{+} (x_1) D^{F}_{\mu\nu} (x_{2} - x_{1}) \psi^{-} (x_2) \gamma^{\nu} \psi (x_2):$ & $:\overline{\psi} (x_1) \gamma^{\mu} \overline{\psi}^{+} (x_1) D^{F}_{\mu\nu} (x_{2} - x_{1}) \overline{\psi}^{-} (x_2) \gamma^{\nu} \psi (x_2):$
        \end{tabular}
    \end{framed}

    The same type of inspection analysis show that no other S-operator term to $2^{nd}$ order term yields a nonzero matrix element contribution for Moller
    scattering or Bhabha scattering. The reasons for this are the same as when we considered the contributions of Eq. A.

    At this point we have identified all of the second order matrix element contributions for all of the processes of interest except four. We have not yet
    considered, electron, positron, and photon self-energy at the second order, and we have not yet considered vacuum energy.

    Because a self-energy is a self-interaction, we are dealing with a proccess where one of the particle enters or leaves unchanged, but under some sort of
    self-interaction in-between. So because we have one incoming particle and one outgoing particle, second order contributions should come from the two
    contraction terms. They are only terms that have the possibility of containing just one creation operator and annihilation operator to create one outgoing
    particle, and annihilate one incoming particle.

    We can work out the specific contributions exactly as we have done it in twice now. We find that $S^{(2)}_{2,1} + S^{(2)}_{2,2}$ gives the only nonzero
    contribution to the electron and positron self-energy matrix elements at the second-order level.
    
    \begin{framed}
        \begin{equation}
            S^{(2)}_{2,1} + S^{(2)}_{2,2} = -e^{2} \int :\overline{\psi} (x_1) \gamma^{\mu} i S_{F} (x_2 - x_1) \gamma^{\nu} i D_{\mu \nu}^{F} (x_2 - x_1) \psi (x_2): d^{4} x_{1} d^{4} x_{2}
        \end{equation}
    \end{framed}

    Of course, there are numerous terms that don't contribute to any of the processes at the beginning. The list is actually comprehensive, so these extraneous
    terms don't actually contribute to anything. This goes for the extraneous terms in the tables aboe as well. This results from the violations of conservation
    laws. Think about it carefully. $S^{(2)}_{2,3}$ gives only nonvanishing contriubtion to the photon self-energy matrix element at the second order level:


    \begin{framed}
        \begin{tabular}{c c c c}
            PROCESS & REACTION & \shortstack{General \\ S-Matrix \\ Element} & \shortstack{Contributing Eq. A \\ Integrand \\ Terms} \\
            Electron Self-Energy & $e^{-} \rightarrow e^{-}$ & $\langle e^{-} | S | e^{-} \rangle$ & $:\overline{\psi}^{-} (x_1) \gamma^{\mu} i S_{F} (x_2 - x_1) \gamma^{\nu} i D_{\mu \nu}^{F} (x_2 - x_1) \psi^{-} (x_2):$ \\
            Positron Self-Energy & $e^{+} \rightarrow e^{+}$ & $\langle e^{+} | S | e^{+} \rangle$ & $:\overline{\psi}^{+} (x_1) \gamma^{\mu} i S_{F} (x_2 - x_1) \gamma^{\nu} i D_{\mu \nu}^{F} (x_2 - x_1) \psi^{+} (x_2):$
        \end{tabular}
    \end{framed}

    $S^{(2)}_{2,3}$ gives the only nonvanishing contribution to the photon self-energy matrix element at the second order level:

    \begin{framed}
        \begin{equation}
            S^{(2)}_{2,3} = \frac{(-ie)^2}{2!} \int :Tr[i S_{F} (x_1 - x_2) \gamma^{\mu} A_{\mu} (x_1) i S_{F} (x_2 - x_1) \gamma^{\nu} A_{\nu} (x_2)]: d^{4} x_{1} d^{4} x_{2} d^{4} x_{1} d^{4} x_{2}
        \end{equation}
    \end{framed}

    Specifically, the contributing integrand terms are as follows:

    \begin{framed}
        \begin{tabular}{c c c c}
            PROCESS & REACTION & \shortstack{General \\ S-Matrix \\ Element} & \shortstack{Contributing Eq. A \\ Integrand \\ Terms} \\
            Photon Self-Energy & $\gamma \rightarrow \gamma$ & $\langle \gamma | S | \gamma \rangle$ & $:Tr[i S_{F} (x_1 - x_2) \gamma^{\mu} A^{+}_{\mu} (x_1) i S_{F} (x_2 - x_1) \gamma^{\nu} A^{-}_{\nu} (x_2)]:$ \\
            & & & $:Tr[i S_{F} (x_1 - x_2) \gamma^{\mu} A^{-}_{\mu} (x_1) i S_{F} (x_2 - x_1) \gamma^{\nu} A^{+}_{\nu} (x_2)]:$
        \end{tabular}
    \end{framed}

    These two terms are equivalent under integration and can be combined. Just like the Bhabha scattering, this gives:

    \begin{framed}
        \begin{tabular}{c c c c}
            PROCESS & REACTION & \shortstack{General \\ S-Matrix \\ Element} & \shortstack{Contributing Eq. A \\ Integrand \\ Terms} \\
            Photon Self-Energy & $\gamma \rightarrow \gamma$ & $\langle \gamma | S | \gamma \rangle$ & $2 :Tr[i S_{F} (x_1 - x_2) \gamma^{\mu} A^{+}_{\mu} (x_1) i S_{F} (x_2 - x_1) \gamma^{\nu} A^{-}_{\nu} (x_2)]:$
        \end{tabular}
    \end{framed}

    Now, for the final proccess, the vacuum energy. $S^{(2)}_{3}$ is the only term that contains no field operators, and therefore won't annihilate the
    vacuum states in this matrix element. It is therefore the only term that can yield a non-zero contribution to the vacuum enery matrix element, so we
    have the following result:

    \begin{framed}
        \begin{tabular}{c c c c}
            PROCESS & REACTION & \shortstack{General \\ S-Matrix \\ Element} & \shortstack{Contributing Eq. A \\ Integrand \\ Terms} \\
            Vacuum Energy & vacuum $\rightarrow$ vacuum & $\langle 0 | S | 0 \rangle$ & $S^{(2)}_{3}$
        \end{tabular}
    \end{framed}

    Now on to associating these S-matrix terms with Feynman Diagrams!

    \section{Feynman Diagram Interpretation of Contributing Scattering Operator Terms}

    For every creation operator, we know that a particle must exit from the spacetime point that is the argument of the operator. For every annihilation
    operator, we know that a particle must enter to be annihilated at the spacetime location that is the argument of the annihilation operator, and we
    know that propagators propagate particles from one spacetime point to another. As is clear from this description, the spacetime arguments tell us how
    to connect the lines to form vertices. So, we should be able to use the facts to associate each S-matrix term with a Feynman diagram that schematically
    displays the reaction corresponding term describes. Solid lines represent fermions (with forward arrows, they represent electrons, with backward ones,
    positrons), and wavy lines represents photons: 

    \hspace{0.25cm}

    \begingroup
        \begin{longtable}{| p{.20\textwidth} | p{.80\textwidth} |}
        \hline

        $e^{-}$ Compton Scattering &
            \begin{equation}
                \langle e^{-}, \gamma | S^{(2)} | e^{-}, \gamma \rangle = \langle e^{-}, \gamma | S_{A} | e^{-}, \gamma \rangle = \langle e^{-}, \gamma | S_{B} | e^{-}, \gamma \rangle
            \end{equation}

            \begin{equation}
                \begin{aligned}
                    \langle e^{-}, \gamma | S_{a} | e^{-}, \gamma \rangle = - e^{2} \int :\overline{\psi}^{-} (x_1) \gamma^{\mu} A_{\mu}^{+} (x_1) i S_{F} (x_2 - x_1) \gamma^{\nu} A_{\nu}^{-} (x_2) \psi^{+} (x_2): d^{4} x_{1} d^{4} x_{2}
                \end{aligned}
            \end{equation}

            \begin{equation}
                \begin{aligned}
                    \langle e^{-}, \gamma | S_{b} | e^{-}, \gamma \rangle = - e^{2} \int :\overline{\psi}^{-} (x_1) \gamma^{\mu} A_{\mu}^{-} (x_1) i S_{F} (x_2 - x_1) \gamma^{\nu} A_{\nu}^{+} (x_2) \psi^{+} (x_2): d^{4} x_{1} d^{4} x_{2}
                \end{aligned}
            \end{equation}

            \begin{center}
                \begin{tabular}{|c|c|}
                    \hline
                    $\langle e^{-}, \gamma | S_{a} | e^{-}, \gamma \rangle$ & $\langle e^{-}, \gamma | S_{b} | e^{-}, \gamma \rangle$ \\
                    \hline
                    \begin{tikzpicture}
                        \begin{feynman}
                            \vertex [label = right: $x_1$] (a);
                            \vertex [below = of a, label = left: $x_2$] (b);
                            \vertex [above right = of a, label = $\gamma$] (c);
                            \vertex [above left = of a, label = $e^{-}$] (d);
                            \vertex [below left = of b, label = $\gamma$] (e);
                            \vertex [below right = of b, label = $e^{+}$] (f);
            
                            \diagram{
                                (b) -- [fermion] (a);
                                (c) -- [boson] (b);
                                (a) -- [fermion] (d);
                                (e) -- [boson] (a);
                                (b) -- [fermion] (f);
                            };
                        \end{feynman}
                    \end{tikzpicture} & \begin{tikzpicture}
                        \begin{feynman}
                            \vertex [label = right: $x_1$] (a);
                            \vertex [below = of a, label = left: $x_2$] (b);
                            \vertex [above right = of a, label = $e^{+}$] (c);
                            \vertex [above left = of a, label = $e^{-}$] (d);
                            \vertex [below left = of b, label = $e^{-}$] (e);
                            \vertex [below right = of b, label = $e^{+}$] (f);
            
                            \diagram{
                                (b) -- [fermion] (a);
                                (c) -- [fermion] (a);
                                (a) -- [boson] (d);
                                (e) -- [boson] (b);
                                (b) -- [fermion] (f);
                            };
                        \end{feynman}
                    \end{tikzpicture} \\
                    \hline
                \end{tabular} \\
            \end{center} \\

            & The integral just accounts for the fact that the vertices could be located anywhere, as we noted with the first order term. \\

        \hline

        $e^{+}$ Compton Scattering &
            \begin{equation}
                \langle e^{-}, \gamma | S^{(2)} | e^{-}, \gamma \rangle = \langle e^{-}, \gamma | S_{A} | e^{-}, \gamma \rangle = \langle e^{-}, \gamma | S_{B} | e^{-}, \gamma \rangle
            \end{equation}

            \begin{equation}
                \begin{aligned}
                    \langle e^{-}, \gamma | S_{a} | e^{-}, \gamma \rangle = - e^{2} \int :\overline{\psi}^{+} (x_1) \gamma^{\mu} A_{\mu}^{+} (x_1) i S_{F} (x_2 - x_1) \gamma^{\nu} A_{\nu}^{-} (x_2) \psi^{-} (x_2): d^{4} x_{1} d^{4} x_{2}
                \end{aligned}
            \end{equation}

            \begin{equation}
                \begin{aligned}
                    \langle e^{-}, \gamma | S_{b} | e^{-}, \gamma \rangle = - e^{2} \int :\overline{\psi}^{+} (x_1) \gamma^{\mu} A_{\mu}^{-} (x_1) i S_{F} (x_2 - x_1) \gamma^{\nu} A_{\nu}^{+} (x_2) \psi^{-} (x_2): d^{4} x_{1} d^{4} x_{2}
                \end{aligned}
            \end{equation}

            \begin{center}
                \begin{tabular}{|c|c|}
                    \hline
                    $\langle e^{-}, \gamma | S_{a} | e^{-}, \gamma \rangle$ & $\langle e^{-}, \gamma | S_{b} | e^{-}, \gamma \rangle$ \\
                    \hline
                    \begin{tikzpicture}
                        \begin{feynman}
                            \vertex [label = right: $x_2$] (a);
                            \vertex [below = of a, label = left: $x_1$] (b);
                            \vertex [above right = of a, label = $e^{+}$] (c);
                            \vertex [above left = of a, label = $e^{-}$] (d);
                            \vertex [below left = of b, label = $e^{-}$] (e);
                            \vertex [below right = of b, label = $e^{+}$] (f);
            
                            \diagram{
                                (a) -- [fermion] (b);
                                (c) -- [fermion] (a);
                                (a) -- [boson] (d);
                                (e) -- [boson] (b);
                                (b) -- [fermion] (f);
                            };
                        \end{feynman}
                    \end{tikzpicture} & \begin{tikzpicture}
                        \begin{feynman}
                            \vertex [label = right: $x_2$] (a);
                            \vertex [below = of a, label = left: $x_1$] (b);
                            \vertex [above right = of a, label = $\gamma$] (c);
                            \vertex [above left = of a, label = $e^{-}$] (d);
                            \vertex [below left = of b, label = $\gamma$] (e);
                            \vertex [below right = of b, label = $e^{+}$] (f);
            
                            \diagram{
                                (a) -- [fermion] (b);
                                (c) -- [boson] (b);
                                (a) -- [fermion] (d);
                                (e) -- [boson] (a);
                                (b) -- [fermion] (f);
                            };
                        \end{feynman}
                    \end{tikzpicture} \\
                    \hline
                \end{tabular} \\
            \end{center} \\

        \hline

        Pair Annihilation &
            \begin{equation}
                \langle \gamma, \gamma | S^{(2)} | e^{-}, e^{+} \rangle = \langle \gamma, \gamma | S^{(2)}_{PA} | e^{-}, e^{+} \rangle
            \end{equation}

            \begin{equation}
                S^(2)_{PA} = -e^{2} \int :\overline{\psi}^{+} (x_1) \gamma^{\mu} A_{\mu}^{-} (x_1) i S_{F} (x_2 - x_1) \gamma^{\nu} A_{\nu}^{-} (x_2) \psi^{+} (x_2): d^4 x_1 d^4 x_2
            \end{equation}

            Looking at this amplitude term, we notice something interesting. Because the outgoing particles are identical (both photons), there are two
            possible Feynman diagrams that we could associate with this term, which differ by an interchange of outgoing photons. When we evaluate these
            amplitudes further, later on in this video, we will find that $\langle \gamma, \gamma | S^{(2)}_{PA} | e^{-}, e^{+} \rangle$ actually produces
            two terms. These terms will only differ by an interchange of the photon polarization vectors, and wll correspond to the two different possible
            Feynman diagrams we have noticed here, which differ by exactly that outgoing photon interchange.

            \begin{center}
                \begin{tabular}{|c|c|}
                    \hline
                    \multicolumn{2}{|c|}{$\langle e^{-}, e^{+} | S^{(2)}_{PA} | \gamma, \gamma \rangle$} \\
                    \hline
                    \begin{tikzpicture}
                        \begin{feynman}
                            \vertex [label = below: $x_1$] (a);
                            \vertex [right = of a,label = above: $x_2$] (b);
                            \vertex [above left = of a, label = $\gamma_2$] (c);
                            \vertex [below left = of a, label = $e^{-}$] (d);
                            \vertex [above right = of b, label = $\gamma_1$] (e);
                            \vertex [below right = of b, label = $e^{+}$] (f);
            
                            \diagram{
                                (b) -- [fermion] (a);
                                (a) -- [boson] (c);
                                (d) -- [fermion] (a);
                                (e) -- [boson] (b);
                                (b) -- [fermion] (f);
                            };
                        \end{feynman}
                    \end{tikzpicture} & \begin{tikzpicture}
                        \begin{feynman}
                            \vertex [label = below: $x_1$] (a);
                            \vertex [right = of a,label = above: $x_2$] (b);
                            \vertex [above left = of a, label = $\gamma_1$] (c);
                            \vertex [below left = of a, label = $e^{-}$] (d);
                            \vertex [above right = of b, label = $\gamma_2$] (e);
                            \vertex [below right = of b, label = $e^{+}$] (f);
            
                            \diagram{
                                (b) -- [fermion] (a);
                                (b) -- [boson] (c);
                                (d) -- [fermion] (a);
                                (e) -- [boson] (a);
                                (b) -- [fermion] (f);
                            };
                        \end{feynman}
                    \end{tikzpicture} \\
                    \hline
                \end{tabular} \\
            \end{center} \\

        \hline

        Pair Production &
            \begin{equation}
                \langle e^{-}, e^{+} | S^{(2)} | \gamma, \gamma \rangle = \langle e^{-}, e^{+} | S^{(2)}_{PP} | \gamma, \gamma \rangle
            \end{equation}

            \begin{equation}
                S^(2)_{PP} = -e^{2} \int :\overline{\psi}^{-} (x_1) \gamma^{\mu} A_{\mu}^{+} (x_1) i S_{F} (x_2 - x_1) \gamma^{\nu} A_{\nu}^{+} (x_2) \psi^{-} (x_2): d^4 x_1 d^4 x_2
            \end{equation}

            Here, we have the same situation we saw with pair annihilation, There are two possible Feynman diagrams that could be associated with this
            term that, again just differ by the interchange of the identical photons. This happens anytime the incoming or outgoing particles are identical
            pairs. Just as with pair annihilation, when we evaluate this amplitude further, we will find two terms that differ only by an interchange of
            the photon polarization vectors.

            \begin{center}
                \begin{tabular}{|c|c|}
                    \hline
                    \multicolumn{2}{|c|}{$\langle \gamma, \gamma | S^{(2)}_{PP} | e^{-}, e^{+} \rangle$} \\
                    \hline
                    \begin{tikzpicture}
                        \begin{feynman}
                            \vertex [label = below: $x_1$] (a);
                            \vertex [right = of a,label = above: $x_2$] (b);
                            \vertex [above left = of a, label = $e^{-}$] (c);
                            \vertex [below left = of a, label = $\gamma_1$] (d);
                            \vertex [above right = of b, label = $e^{+}$] (e);
                            \vertex [below right = of b, label = $\gamma_2$] (f);
            
                            \diagram{
                                (b) -- [fermion] (a);
                                (a) -- [fermion] (c);
                                (d) -- [boson] (a);
                                (e) -- [fermion] (b);
                                (b) -- [boson] (f);
                            };
                        \end{feynman}
                    \end{tikzpicture} & \begin{tikzpicture}
                        \begin{feynman}
                            \vertex [label = below: $x_1$] (a);
                            \vertex [right = of a,label = above: $x_2$] (b);
                            \vertex [above left = of a, label = $e^{-}$] (c);
                            \vertex [below left = of a, label = $\gamma_2$] (d);
                            \vertex [above right = of b, label = $e^{+}$] (e);
                            \vertex [below right = of b, label = $\gamma_1$] (f);
            
                            \diagram{
                                (b) -- [fermion] (a);
                                (a) -- [fermion] (c);
                                (d) -- [boson] (b);
                                (e) -- [fermion] (b);
                                (a) -- [boson] (f);
                            };
                        \end{feynman}
                    \end{tikzpicture} \\
                    \hline
                \end{tabular} \\
            \end{center} \\

        \hline

        $e^{-}$ Moller Scattering &
            \begin{equation}
                \langle e^{-}, e^{-} | S^{(2)} | e^{-}, e^{-} \rangle = \langle e^{-}, e^{-} | S_{EM} | e^{-}, e^{-} \rangle
            \end{equation}

            \begin{equation}
                S_{EM} = \frac{-e^2}{2} \int :\overline{\psi}^{-} (x_1) \gamma^{\mu} \psi^{+} (x_1) i D^{F}_{\mu \nu} (x_2 - x_1) \overline{\psi}^{-} (x_2) \gamma^{\nu} \psi^{+} (x_2): d^4 x_1 d^4 x_2
            \end{equation}

            With Moller scattering, we again have a similar situation to what we saw in the last two entries in this table, only this time, it is more extreme.
            Both the incoming particles and the outgoing particles are identical pairs. Therefore, there are four different Feynman diagrams that we could
            associate with this term, and when we evaluate the amplitude further, we will find that it does contain four terms. We will also find that two pairs
            of them are actually identical to the incoming particles this corresponds to the fact that there are only two physically distinct diagrams that could
            be associated with this matrix. Just like the previous ones, these diagrams differby an interchange of two identical particles. The usual selection
            for the two physically distinct diagrams is as follows:

            \begin{center}
                \begin{tabular}{|c|c|}
                    \hline
                    \multicolumn{2}{|c|}{$\langle e^{-}, e^{-} | S_{EM} | e^{-}, e^{-} \rangle$} \\
                    \hline
                    \begin{tikzpicture}
                            \begin{feynman}
                                \vertex [label = below: $x_1$] (a);
                                \vertex [right = of a,label = above: $x_2$] (b);
                                \vertex [above left = of a, label = $e^{-}$] (c);
                                \vertex [below left = of a, label = $e^{-}$] (d);
                                \vertex [above right = of b, label = $e^{-}$] (e);
                                \vertex [below right = of b, label = $e^{-}$] (f);
                
                                \diagram{
                                    (a) -- [boson] (b);
                                    (a) -- [fermion] (c);
                                    (d) -- [fermion] (a);
                                    (e) -- [fermion] (b);
                                    (b) -- [fermion] (f);
                                };
                            \end{feynman}
                        \end{tikzpicture} & \begin{tikzpicture}
                        \begin{feynman}
                            \vertex [label = below: $x_1$] (a);
                            \vertex [right = of a,label = above: $x_2$] (b);
                            \vertex [above left = of a, label = $e^{-}$] (c);
                            \vertex [below left = of a, label = $e^{-}$] (d);
                            \vertex [above right = of b, label = $e^{-}$] (e);
                            \vertex [below right = of b, label = $e^{-}$] (f);
            
                            \diagram{
                                (a) -- [boson] (b);
                                (e) -- [fermion] (a);
                                (d) -- [fermion] (a);
                                (c) -- [fermion] (b);
                                (b) -- [fermion] (f);
                            };
                        \end{feynman}
                    \end{tikzpicture} \\
                    \hline
                \end{tabular} \\
            \end{center} \\

            & When we do evaluate this amplitude further, we will find one other thing. The two terms that we do ultimately end up with have opposite
            signs in addition to the interchanged outgoing electron momenta. This results from the anticommuting property of fermionic creation and
            annihilation operators. This won't happen for the case of bosons becase their associated operators have commuting properties. \\ 

        \hline

        $e^{+}$ Moller Scattering &
            \begin{equation}
                \langle e^{+}, e^{+} | S^{(2)} | e^{+}, e^{+} \rangle = \langle e^{+}, e^{+} | S_{EM} | e^{+}, e^{+} \rangle
            \end{equation}

            \begin{equation}
                S_{PM} = \frac{-e^2}{2} \int :\overline{\psi}^{+} (x_1) \gamma^{\mu} \psi^{-} (x_1) i D^{F}_{\mu \nu} (x_2 - x_1) \overline{\psi}^{+} (x_2) \gamma^{\nu} \psi^{-} (x_2): d^4 x_1 d^4 x_2
            \end{equation}

            The multiplicity of the graphs follows exactly the same description as in the $e^{-}$ Moller Scattering case.

            \begin{center}
                \begin{tabular}{|c|c|}
                    \hline
                    \multicolumn{2}{|c|}{$\langle e^{+}, e^{+} | S_{EM} | e^{+}, e^{+} \rangle$} \\
                    \hline
                    \begin{tikzpicture}
                        \begin{feynman}
                            \vertex [label = below: $x_1$] (a);
                            \vertex [right = of a,label = above: $x_2$] (b);
                            \vertex [above left = of a, label = $e^{+}$] (c);
                            \vertex [below left = of a, label = $e^{+}$] (d);
                            \vertex [above right = of b, label = $e^{+}$] (e);
                            \vertex [below right = of b, label = $e^{+}$] (f);
            
                            \diagram{
                                (a) -- [boson] (b);
                                (a) -- [fermion] (c);
                                (d) -- [fermion] (a);
                                (e) -- [fermion] (b);
                                (b) -- [fermion] (f);
                            };
                        \end{feynman}
                    \end{tikzpicture} & \begin{tikzpicture}
                        \begin{feynman}
                            \vertex [label = below: $x_1$] (a);
                            \vertex [right = of a,label = above: $x_2$] (b);
                            \vertex [above left = of a, label = $e^{+}$] (c);
                            \vertex [below left = of a, label = $e^{+}$] (d);
                            \vertex [above right = of b, label = $e^{+}$] (e);
                            \vertex [below right = of b, label = $e^{+}$] (f);
            
                            \diagram{
                                (a) -- [boson] (b);
                                (e) -- [fermion] (a);
                                (d) -- [fermion] (a);
                                (c) -- [fermion] (b);
                                (b) -- [fermion] (f);
                            };
                        \end{feynman}
                    \end{tikzpicture} \\
                    \hline
                \end{tabular} \\
            \end{center} \\

        \hline

        Bhabha Scattering &
            \begin{equation}
                \langle e^{-}, e^{+} | S^{(2)} | e^{-}, e^{+}\rangle = \langle e^{-}, e^{+} | S_{\alpha} | e^{-}, e^{+} \rangle = \langle e^{-}, \gamma | S_{\beta} | e^{-}, e^{+} \rangle
            \end{equation}

            \begin{equation}
                S_{\alpha} = - e^{2} \int :\overline{\psi}^{-} (x_1) \gamma^{\mu} \psi^{-} (x_1) i D^{F}_{\mu \nu} (x_2 - x_1) \overline{\psi}^{+} (x_2) \gamma^{\nu} \psi^{+} (x_2): d^4 x_1 d^4 x_2
            \end{equation}

            \begin{equation}
                S_{\beta} = - e^{2} \int :\overline{\psi}^{-} (x_1) \gamma^{\mu} \psi^{+} (x_1) i D^{F}_{\mu \nu} (x_2 - x_1) \overline{\psi}^{+} (x_2) \gamma^{\nu} \psi^{-} (x_2): d^4 x_1 d^4 x_2
            \end{equation}

            \begin{center}
                \begin{tabular}{|c|c|}
                    \hline
                    $\langle e^{-}, e^{+} | S_{\alpha} | e^{-}, e^{+} \rangle$ & $\langle e^{-}, \gamma | S_{\beta} | e^{-}, e^{+} \rangle$ \\
                    \hline
                    \begin{tikzpicture}
                        \begin{feynman}
                            \vertex [label = right: $x_1$] (a);
                            \vertex [below = of a, label = left: $x_2$] (b);
                            \vertex [above right = of a, label = $e^{+}$] (c);
                            \vertex [above left = of a, label = $e^{-}$] (d);
                            \vertex [below left = of b, label = $e^{-}$] (e);
                            \vertex [below right = of b, label = $e^{+}$] (f);
            
                            \diagram{
                                (a) -- [boson] (b);
                                (c) -- [fermion] (a);
                                (a) -- [fermion] (d);
                                (e) -- [fermion] (b);
                                (b) -- [fermion] (f);
                            };
                        \end{feynman}
                    \end{tikzpicture} & \begin{tikzpicture}
                        \begin{feynman}
                            \vertex [label = below: $x_1$] (a);
                            \vertex [right = of a,label = above: $x_2$] (b);
                            \vertex [above left = of a, label = $e^{-}$] (c);
                            \vertex [below left = of a, label = $e^{-}$] (d);
                            \vertex [above right = of b, label = $e^{+}$] (e);
                            \vertex [below right = of b, label = $e^{+}$] (f);
            
                            \diagram{
                                (a) -- [boson] (b);
                                (a) -- [fermion] (c);
                                (d) -- [fermion] (a);
                                (e) -- [fermion] (b);
                                (b) -- [fermion] (f);
                            };
                        \end{feynman}
                    \end{tikzpicture} \\
                    \hline
                \end{tabular} \\
            \end{center} \\

        \hline

        Electron Self Energy &
            \begin{equation}
                \langle e^{-} | S^{(2)} | e^{-} \rangle = \langle e^{-} | S_{ESE} | e^{-} \rangle
            \end{equation}

            \begin{equation}
                S_{ESE} = -e^{2} \int :\overline{\psi}^{-} (x_1) \gamma^{\mu} i S_{F} (x_2 - x_1) \gamma^{\nu} i D^{F}_{\mu \nu} (x_2 - x_1) \psi^{+} (x_2): d^4 x_1 d^4 x_2
            \end{equation}

            \begin{center}

                \begin{tikzpicture}
                    \begin{feynman}
                        \vertex [label = right: $e^{-}$] (a);
                        \vertex [below = of a, label = below: $x_1$] (b);
                        \vertex [below = of b, label = above: $x_2$] (c);
                        \vertex [below = of c, label = right: $e^{-}$] (d);
        
                        \diagram{
                            (a) -- [fermion] (b);
                            (b) -- [fermion] (c);
                            (c) -- [boson, half left] (b);
                            (c) -- [fermion] (d);
                        };
                    \end{feynman}
                \end{tikzpicture}
    
            \end{center} \\

        \hline

        Positron Self Energy &
            \begin{equation}
                \langle e^{+} | S^{(2)} | e^{+} \rangle = \langle e^{+} | S_{PSE} | e^{+} \rangle
            \end{equation}

            \begin{equation}
                S_{PSE} = -e^{2} \int :\overline{\psi}^{+} (x_1) \gamma^{\mu} i S_{F} (x_2 - x_1) \gamma^{\nu} i D^{F}_{\mu \nu} (x_2 - x_1) \psi^{-} (x_2): d^4 x_1 d^4 x_2
            \end{equation}

            \begin{center}

                \begin{tikzpicture}
                    \begin{feynman}
                        \vertex [label = right: $e^{+}$] (a);
                        \vertex [below = of a, label = below: $x_2$] (b);
                        \vertex [below = of b, label = above: $x_1$] (c);
                        \vertex [below = of c, label = right: $e^{+}$] (d);
        
                        \diagram{
                            (a) -- [fermion] (b);
                            (b) -- [fermion] (c);
                            (c) -- [boson, half left] (b);
                            (c) -- [fermion] (d);
                        };
                    \end{feynman}
                \end{tikzpicture}
    
            \end{center} \\

        \hline

        Photon Self Energy &
            \begin{equation}
                \langle \gamma | S^{(2)} | \gamma \rangle = \langle \gamma | S_{PhSE} | \gamma \rangle
            \end{equation}

            \begin{equation}
                S_{PhSE} = -e^{2} \int :Tr[i S_{F} (x_1 - x_2) \gamma^{\mu} A_{\mu}^{+} (x_1) S_{F} (x_2 - x_1) \gamma^{\nu} A_{\nu}^{-} (x_2)]: d^4 x_1 d^4 x_2
            \end{equation}

            \begin{center}

                \begin{tikzpicture}
                    \begin{feynman}
                        \vertex [label = right: $\gamma$] (a);
                        \vertex [below = of a, label = below: $x_2$] (b);
                        \vertex [below = of b, label = above: $x_1$] (c);
                        \vertex [below = of c, label = right: $\gamma$] (d);
        
                        \diagram{
                            (a) -- [boson] (b);
                            (b) -- [fermion, half left] (c);
                            (c) -- [fermion, half left] (b);
                            (c) -- [boson] (d);
                        };
                    \end{feynman}
                \end{tikzpicture}
    
            \end{center} \\

        \hline

        Vacuum Energy &
            \begin{equation}
                \langle 0 | S | 0 \rangle = \langle 0 | S^{(2)}_{3} | 0 \rangle
            \end{equation}

            \begin{equation}
                S^{(2)}_{3} = \frac{(-ie)^2}{2!} \int :Tr[i S_{F} (x_1 - x_2) \gamma^{\mu} i S_{F} (x_2 - x_1) \gamma^{\nu} i D^{F}_{\mu \nu} (x_2 - x_1)]: d^4 x_1 d^4 x_2
            \end{equation}

            \begin{center}

                \begin{tikzpicture}
                    \begin{feynman}
                        \vertex [label = above: $x_2$] (a);
                        \vertex [below = of a, label = below: $x_1$] (b);
        
                        \diagram{
                            (a) -- [boson] (b);
                            (a) -- [fermion, half right] (b);
                            (b) -- [fermion, half right] (a);
                        };
                    \end{feynman}
                \end{tikzpicture}
    
            \end{center} \\

        \hline

        \end{longtable}
    \endgroup

    \hspace{0.25cm}

    Now that we have this Feynman diagram interpretation in place, we can notice another interesting thing or two about the two particle scattering processes
    (Compton Scattering, Pair Annihilation, Pair Production, Moller Scattering, Bhabha Scattering). All of the diagrams to second order are loop free. Also,
    with orders any higher than second, the additional vertices would be would force loops into the diagram. Therefore, for these processes, we have not just
    worked out the complete $2^{nd}$ order S-matrix contribution, but with the complete "tree level" contribution.

    Let us now work on computing these amplitudes further. The goal is to simplify them to the point where their Feynman amplitudes can be extracted. In the
    last section, we will compare these Feynman amplitudes to the Feynman diagrams that we have just associated to their corresponding matrix elements. This
    comparison will reveal the Feynman rules.

    \section{Direct Computation of Second Order Amplitudes}

        This is where the calculation starts to get fun, we will see the familiar expressions yielded by Feynman's Rules emerge from direct matrix element calculations.

        \subsection{$e^{-}$ Compton Scattering}

        \begin{framed}
            \begin{equation}
                \langle e^{-}, \gamma | S^{(2)} | e^{-}, \gamma \rangle = \langle e^{-}, \gamma | S_{A} | e^{-}, \gamma \rangle = \langle e^{-}, \gamma | S_{B} | e^{-}, \gamma \rangle
            \end{equation}

            \begin{equation}
                \begin{aligned}
                    \langle e^{-}, \gamma | S_{A} | e^{-}, \gamma \rangle = - e^{2} \int \langle e^{-}, \gamma |:\overline{\psi} (x_1) \gamma^{\mu} A_{\mu} (x_1) i S_{F} (x_2 - x_1) \gamma^{\nu} A_{\nu} (x_2) \psi (x_2):| e^{-}, \gamma \rangle d^{4} x_{1} d^{4} x_{2}
                \end{aligned}
            \end{equation}

            \begin{equation}
                \begin{aligned}
                    \langle e^{-}, \gamma | S_{A} | e^{-}, \gamma \rangle = - e^{2} \int \langle e^{-}, \gamma |:\overline{\psi} (x_1) \gamma^{\mu} A_{\mu} (x_1) i S_{F} (x_2 - x_1) \gamma^{\nu} A_{\nu} (x_2) \psi (x_2):| e^{-}, \gamma \rangle d^{4} x_{1} d^{4} x_{2}
                \end{aligned}
            \end{equation}

            We can write this without the normal ordering signs as long as the operators are properly normal ordered. From there we can bring it out of strict normal
            ordered form without destroying the equality as long as only mutually commuting factors are moved past each other: 

            \begin{equation}
                \begin{aligned}
                    \langle e^{-}, \gamma | S_{A} | e^{-}, \gamma \rangle = - e^{2} \int \langle e^{-}, \gamma |\overline{\psi} (x_1) \gamma^{\mu} A_{\mu} (x_1) i S_{F} (x_2 - x_1) \gamma^{\nu} A_{\nu} (x_2) \psi (x_2)| e^{-}, \gamma \rangle d^{4} x_{1} d^{4} x_{2}
                \end{aligned}
            \end{equation}

            \begin{equation}
                \begin{aligned}
                    \langle e^{-}, \gamma | S_{A} | e^{-}, \gamma \rangle = - e^{2} \int \langle e^{-}, \gamma |\overline{\psi} (x_1) \gamma^{\mu} A_{\mu} (x_1) i S_{F} (x_2 - x_1) \gamma^{\nu} A_{\nu} (x_2) \psi (x_2)| e^{-}, \gamma \rangle d^{4} x_{1} d^{4} x_{2}
                \end{aligned}
            \end{equation}

            Let's start by computing $\langle e^{-}, \gamma | S_{A} | e^{-}, \gamma \rangle$:

            \begin{equation}
                \begin{aligned}
                    \langle f | S_{A} | i \rangle = - e^{2} \int \langle e^{-}, \gamma |\overline{\psi}^{-} (x_1) \gamma^{\mu} A_{\nu}^{-} (x_2) i S_{F} (x_2 - x_1) \gamma^{\nu}  \psi^{+} (x_2)| e^{-}, \gamma A_{\mu}^{+} (x_1) \rangle d^{4} x_{1} d^{4} x_{2}
                \end{aligned}
            \end{equation}

            \begin{equation}
                \begin{aligned}
                    A_{\mu}^{+} (x) | \gamma \rangle = \epsilon_{\mu}^{\lambda} e^{- i k \cdot x} | 0 \rangle \qquad & \qquad \langle \gamma | A_{\mu}^{-} (x) = \langle 0 | \epsilon_{\mu}^{\lambda} e^{i k \cdot x} \\
                    \psi^{+} (x) | e^{-} \rangle = u_{s} (\vec{p}) e^{- i p \cdot x}  | 0 \rangle \qquad & \qquad \langle e^{-} | \overline{\psi}^{-} (x) = \langle 0 | e^{i p \cdot x} \overline{u}_{s} (\vec{p})
                \end{aligned}
            \end{equation}

            \begin{equation}
                \begin{aligned}
                    \langle f | S_{A} | i \rangle = - e^{2} \int \langle 0 | e^{i k \cdot x_2} e^{i p \cdot x_1} \overline{u}_{s} (\vec{p}) \gamma^{\mu} \epsilon_{\mu}^{\lambda} i S_{F} (x_2 - x_1) \gamma^{\nu} \epsilon_{\nu}^{\lambda'} u_{s} (\vec{p}) e^{- i p \cdot x_2} e^{- i k \cdot x_1} | 0 \rangle d^{4} x_{1} d^{4} x_{2}
                \end{aligned}
            \end{equation}

            There are no operators left, only c-numbers, and we will take the vacuum state to be normalized:

            \begin{equation}
                \begin{aligned}
                    \langle f | S_{A} | i \rangle = - e^{2} \int e^{i k \cdot x_2} e^{i p \cdot x_1} \overline{u}_{s} (\vec{p}) \gamma^{\mu} \epsilon_{\mu}^{\lambda} i S_{F} (x_2 - x_1) \gamma^{\nu} \epsilon_{\nu}^{\lambda'} u_{s} (\vec{p}) e^{- i p \cdot x_2} e^{- i k \cdot x_1} d^{4} x_{1} d^{4} x_{2}
                \end{aligned}
            \end{equation}

            We can make further simplifications and also switch to Feynman slash notation. Also, remember that we are only dealing with transverse polarized photons,
            so the polarization index can be placed up or down:

            \begin{equation}
                \begin{aligned}
                    \langle f | S_{A} | i \rangle = - e^{2} \int \overline{u}_{s} (\vec{p}) \slashed{\epsilon}_{\mu}^{\lambda} i S_{F} (x_2 - x_1) \slashed{\epsilon}_{\nu}^{\lambda'} u_{s} (\vec{p}) e^{- i (p - k') \cdot x_2} e^{- i (k- p') \cdot x_1} d^{4} x_{1} d^{4} x_{2}
                \end{aligned}
            \end{equation}

            Now we can write the propagator in terms of its Fourier transform:

            \begin{equation}
                \begin{aligned}
                    \langle f | S_{A} | i \rangle = - e^{2} \int \langle e^{-}, \gamma |\overline{\psi} (x_1) \gamma^{\mu} A_{\mu} (x_1) i S_{F} (x_2 - x_1) \gamma^{\nu} A_{\nu} (x_2) \psi (x_2)| e^{-}, \gamma \rangle d^{4} x_{1} d^{4} x_{2}
                \end{aligned}
            \end{equation}

            \begin{equation}
                \begin{aligned}
                    \langle f | S_{A} | i \rangle = - e^{2} \int \langle e^{-}, \gamma |\overline{\psi} (x_1) \gamma^{\mu} A_{\mu} (x_1) i S_{F} (x_2 - x_1) \gamma^{\nu} A_{\nu} (x_2) \psi (x_2)| e^{-}, \gamma \rangle d^{4} x_{1} d^{4} x_{2}
                \end{aligned}
            \end{equation}

            Next we do the $d^{4} x_{1}$ and $d^{4} x_{2}$ integrations. They yield the delta functions:

            \begin{equation}
                \begin{aligned}
                    \langle f | S_{A} | i \rangle = - e^{2} \int \langle e^{-}, \gamma |\overline{\psi} (x_1) \gamma^{\mu} A_{\mu} (x_1) i S_{F} (x_2 - x_1) \gamma^{\nu} A_{\nu} (x_2) \psi (x_2)| e^{-}, \gamma \rangle d^{4} x_{1} d^{4} x_{2}
                \end{aligned}
            \end{equation}

            \begin{equation}
                \begin{aligned}
                    \langle f | S_{A} | i \rangle = - e^{2} \int \langle e^{-}, \gamma |\overline{\psi} (x_1) \gamma^{\mu} A_{\mu} (x_1) i S_{F} (x_2 - x_1) \gamma^{\nu} A_{\nu} (x_2) \psi (x_2)| e^{-}, \gamma \rangle d^{4} x_{1} d^{4} x_{2}
                \end{aligned}
            \end{equation}

            Now recall the following property of the delta function:

            \begin{equation}
                \delta (x - a) f (x) = \delta (x - a) f(a)
            \end{equation}

            Applying that to the product of delta functions in the integral gives:

            \begin{equation}
                \delta^{4} (k - p' + q) \delta^{4} (p - k' - q) = \delta^{4} (p + k - p' - k') \delta^{4} (p - k' - q)
            \end{equation}

            Inserting this gives:

            \begin{equation}
                \langle f | S_{A} | i \rangle = -e^{2} (2 \pi)^4 \int \delta^{4} (p + k - p' - k') \delta^{4} (p - k' - q) \overline{u}_{s'} (\vec{p}) \slashed{\epsilon}_{\lambda} i S_{F} (q) \slashed{\epsilon}_{\lambda'} u_{s} (\vec{p}) d^{4} q
            \end{equation}

            Now that only one of the delta functions has q dependence, we can easily do the q integration:

            \begin{equation}
                \langle f | S_{A} | i \rangle = -e^{2} (2 \pi)^4 \delta^{4} (p + k - p' - k') \overline{u}_{s'} (\vec{p}) \slashed{\epsilon}_{\lambda} i S_{F} (p - k') \slashed{\epsilon}_{\lambda'} u_{s} (\vec{p})
            \end{equation}

            So we can finally write: 

            \begin{equation}
                \langle f | S_{A} | i \rangle = (2 \pi)^4 \delta^{4} (p + k - p' - k') \mathcal{M}_{fi}^{!}
            \end{equation}

            \begin{equation}
                \mathcal{M}_{fi} = -e^{2} \overline{u}_{s'} (\vec{p}) \slashed{\epsilon}_{\lambda} i S_{F} (p- k') \slashed{\epsilon}_{\lambda'} u_{s} (\vec{p})
            \end{equation}

            An extremely calculation gives the follwing result for $\langle f | S_{B} | i \rangle$:

            \begin{equation}
                \langle f | S_{B} | i \rangle = (2 \pi)^4 \delta^{4} (p + k - p' - k') \mathcal{M}_{fi}^{B}
            \end{equation}

            \begin{equation}
                \mathcal{M}_{fi} = -e^{2} \overline{u}_{s'} (\vec{p}) \slashed{\epsilon}_{\lambda} i S_{F} (p + k) \slashed{\epsilon}_{\lambda'} u_{s} (\vec{p})
            \end{equation}

        \end{framed}

        \subsection{$e^{+}$ Compton Scattering}

        \begin{framed}

            \begin{equation}
                \langle e^{-}, \gamma | S^{(2)} | e^{-}, \gamma \rangle = \langle e^{-}, \gamma | S_{A} | e^{-}, \gamma \rangle = \langle e^{-}, \gamma | S_{B} | e^{-}, \gamma \rangle
            \end{equation}

            \begin{equation}
                \begin{aligned}
                    \langle e^{-}, \gamma | S_{B} | e^{-}, \gamma \rangle = - e^{2} \int \langle e^{-}, \gamma |:\overline{\psi} (x_1) \gamma^{\mu} A_{\mu} (x_1) i S_{F} (x_2 - x_1) \gamma^{\nu} A_{\nu} (x_2) \psi (x_2):| e^{-}, \gamma \rangle d^{4} x_{1} d^{4} x_{2}
                \end{aligned}
            \end{equation}

            \begin{equation}
                \begin{aligned}
                    \langle e^{-}, \gamma | S_{B} | e^{-}, \gamma \rangle = - e^{2} \int \langle e^{-}, \gamma |:\overline{\psi} (x_1) \gamma^{\mu} A_{\mu} (x_1) i S_{F} (x_2 - x_1) \gamma^{\nu} A_{\nu} (x_2) \psi (x_2):| e^{-}, \gamma \rangle d^{4} x_{1} d^{4} x_{2}
                \end{aligned}
            \end{equation}

            A calculation  essentially identical to $e^{-}$ Compton scattering yields the same results for $e^{+}$ Compton Scattering:

            \begin{equation}
                \mathcal{M}_{fi} = \mathcal{M}_{fi}^{A} + \mathcal{M}_{fi}^{B}
            \end{equation}
        \end{framed}

        \subsection{Pair Annihilation}

        \begin{framed}

            \begin{equation}
                \langle \gamma, \gamma | S^{(2)} | e^{-}, e^{+} \rangle = -e^{2} \int \langle \gamma, \gamma |:\overline{\psi}^{+} (x_1) \gamma^{\mu} A_{\mu}^{-} (x_1) i S_{F} (x_2 - x_1) \gamma^{\nu} A_{\nu}^{-} (x_2) \psi^{+} (x_2):| e^{-}, e^{+} \rangle d^4 x_1 d^4 x_2
            \end{equation}

            \begin{equation}
                \psi^{+} (x) = \sum_{s} \frac{m}{\omega} \int c_{s} (\vec{p}) u_{s} (\vec{p}) e^{- i p \cdot x} \frac{d^3 k}{(2 \pi)^{3}} \qquad \overline{\psi}^{+} (x) = \sum_{s} \frac{m}{\omega} \int d_{s} (\vec{p}) \overline{v}_{s} (\vec{p}) e^{- i p \cdot x} \frac{d^3 k}{(2 \pi)^{3}}
            \end{equation}

            \begin{equation}
                A_{\mu}^{-} (x) = \int \epsilon^{\lambda}_{\mu} a^{\dagger}_{\lambda} e^{i k \cdot x} \frac{d^3 k}{(2 \pi)^{3} 2 \omega}
            \end{equation}

            \begin{equation}
                \begin{aligned}
                    \langle \gamma, \gamma | S^{(2)} | e^{-}, e^{+} \rangle = & - e^{2} \sum_{r s} \int \langle \gamma, \gamma |:d_{s} (\vec{k_1}) a^{\dagger}_{h} (\vec{k'_1}) a^{\dagger}_{h'} (\vec{k'_2}) c_{r} (\vec{k_2}):| e^{-}, e^{+} \rangle \\
                    & \overline{v}_{s} (\vec{k}_2) \slashed{\epsilon}^{h'} i S_{F} (x_2 - x_1) \slashed{\epsilon}^{h} u_{r} (\vec{k}_1) e^{i k_1 \cdot x} e^{- i k'_2 \cdot x} e^{i k_2 \cdot x} e^{- i k'_1 \cdot x} \\
                    & \frac{d^3 k_1}{(2 \pi)^{3}} \frac{m}{\omega_1} \frac{d^3 k_2}{(2 \pi)^{3}} \frac{m}{\omega_2} \frac{d^3 k'_1}{(2 \pi)^{3} 2 \omega'_1} \frac{d^3 k'_2}{(2 \pi)^{3} 2 \omega'_2} d^4 x_1 d^4 x_2
                \end{aligned}
            \end{equation}

            Now we can do normal ordering:

            \begin{equation}
                \begin{aligned}
                    \langle \gamma, \gamma | S^{(2)} | e^{-}, e^{+} \rangle = & - e^{2} \sum_{r s} \int \langle \gamma, \gamma |a^{\dagger}_{h} (\vec{k'_1}) a^{\dagger}_{h'} (\vec{k'_2}) d_{s} (\vec{k_1}) c_{r} (\vec{k_2})| e^{-}, e^{+} \rangle \\
                    & \overline{v}_{s} (\vec{k}_2) \slashed{\epsilon}^{h'} i S_{F} (x_2 - x_1) \slashed{\epsilon}^{h} u_{r} (\vec{k}_1) e^{- i (k_1 - k'_2) \cdot x} e^{- i (k_2 - k'_1) \cdot x} \\
                    & \frac{d^3 k_1}{(2 \pi)^{3}} \frac{m}{\omega_1} \frac{d^3 k_2}{(2 \pi)^{3}} \frac{m}{\omega_2} \frac{d^3 k'_1}{(2 \pi)^{3} 2 \omega'_1} \frac{d^3 k'_2}{(2 \pi)^{3} 2 \omega'_2} d^4 x_1 d^4 x_2
                \end{aligned}
            \end{equation}

            We can now pull the operators out of the state: 

            \begin{equation}
                \begin{aligned}
                    \langle \gamma, \gamma | S^{(2)} | e^{-}, e^{+} \rangle = & - e^{2} \sum_{r s} \int \langle 0 |a_{\lambda} (\vec{p'_2}) a_{\lambda'} (\vec{p'_1}) a^{\dagger}_{h} (\vec{k'_1}) a^{\dagger}_{h'} (\vec{k'_2}) d_{s} (\vec{k_2}) c_{r} (\vec{k_1}) c^{\dagger}_{r} (\vec{p_1}) d^{\dagger}_{s} (\vec{p_2})| 0 \rangle \\
                    & \overline{v}_{s} (\vec{k}_2) \slashed{\epsilon}^{h'} i S_{F} (x_2 - x_1) \slashed{\epsilon}^{h} u_{r} (\vec{k}_1) e^{- i (k_1 - k'_2) \cdot x} e^{- i (k_2 - k'_1) \cdot x} \\
                    & \frac{d^3 k_1}{(2 \pi)^{3}} \frac{m}{\omega_1} \frac{d^3 k_2}{(2 \pi)^{3}} \frac{m}{\omega_2} \frac{d^3 k'_1}{(2 \pi)^{3} 2 \omega'_1} \frac{d^3 k'_2}{(2 \pi)^{3} 2 \omega'_2} d^4 x_1 d^4 x_2
                \end{aligned}
            \end{equation}

            \begin{framed}
                We expect the k integrations and the spin/helicity sums to yield two terms because there are two different ways of assigning the k
                values and spin/helicity values that don't result in the states being annihilated
            \end{framed}

            \begin{framed}
                Just like when we proved that the first order term contributes to nothing, the photons are transverse, so the longitudinal and temporal
                annihilation operator terms h and h' sums to two term sums where $\eta_{\lambda \lambda'}$ is replaced with $\delta_{\lambda \lambda'}$,
                where the indices on the Kronecker delta are therefore now two dimensional. As with the first order calculation, this also changes the
                commutation relations to:
                
                \begin{equation}
                    [a_{\lambda} (\vec{k}), a^{\dag}_{\lambda'} (\vec{k'})] = (2 \pi)^{3} 2 \omega \delta_{\lambda \lambda'} \delta^{3} (\vec{k} - \vec{k'})
                \end{equation}

                The relation will be used next
            \end{framed}

            We now know that for angular momentum to be conserved, the exciting photons must have opposite helicity, so both photons cannot be exciting in
            the same quantum state. Therefore, given the commutation relation, we can see that the photon creation operators (in the above matrix element)
            can fail to commute with at most one of the photon annihilation operators, so we can pick which ones definitely commute in two diffrent ways.
            Given the particular selection made, one can then pair each creation operator with the annihilation operator that it will not necessary commute
            with, and replace the pair with its commutator. We can do this because the added term annihilates on the vacuum. The delta functions and the
            Kronecker deltas in the commutator then force a momentum and spin assignment that represents one of the contributions described in Box A. Because
            there are two such pairings, there are two contributions to the integral, as mentioned in Box A. The fermion operators that can also be paired by
            type and replaced by anticommutators. The pairing by type is the only pairing of fermionic operators that has the capacity to allow a nonvanishing
            contributions to the integral. Doing all these gives:

            \begin{equation}
                \begin{aligned}
                    \langle \gamma, \gamma | S^{(2)} | e^{-}, e^{+} \rangle = & - e^{2} \sum_{r s} \int \langle 0 |[a_{\lambda} (\vec{p'_2}),a^{\dagger}_{h} (\vec{k'_1}] [a_{\lambda'} (\vec{p'_1}), a^{\dagger}_{h'} (\vec{k'_2})] \{c_{r} (\vec{k_1}), c^{\dagger}_{r} (\vec{p_1}) \} \{d_{s} (\vec{k_2}), d^{\dagger}_{s} (\vec{p_2}) \}| 0 \rangle \\
                    & \overline{v}_{s} (\vec{k}_2) \slashed{\epsilon}^{h'} i S_{F} (x_2 - x_1) \slashed{\epsilon}^{h} u_{r} (\vec{k}_1) e^{- i (k_1 - k'_2) \cdot x} e^{- i (k_2 - k'_1) \cdot x} \\
                    & \frac{d^3 k_1}{(2 \pi)^{3}} \frac{m}{\omega_1} \frac{d^3 k_2}{(2 \pi)^{3}} \frac{m}{\omega_2} \frac{d^3 k'_1}{(2 \pi)^{3} 2 \omega'_1} \frac{d^3 k'_2}{(2 \pi)^{3} 2 \omega'_2} d^4 x_1 d^4 x_2 \\
                    & - e^{2} \sum_{r s} \int \langle 0 |[a_{\lambda} (\vec{p'_2}),a^{\dagger}_{h} (\vec{k'_1}] [a_{\lambda'} (\vec{p'_1}), a^{\dagger}_{h'} (\vec{k'_2})] \{c_{r} (\vec{k_1}), c^{\dagger}_{r} (\vec{p_1}) \} \{d_{s} (\vec{k_2}), d^{\dagger}_{s} (\vec{p_2}) \}| 0 \rangle \\
                    & \overline{v}_{s} (\vec{k}_2) \slashed{\epsilon}^{h'} i S_{F} (x_2 - x_1) \slashed{\epsilon}^{h} u_{r} (\vec{k}_1) e^{- i (k_1 - k'_2) \cdot x} e^{- i (k_2 - k'_1) \cdot x} \\
                    & \frac{d^3 k_1}{(2 \pi)^{3}} \frac{m}{\omega_1} \frac{d^3 k_2}{(2 \pi)^{3}} \frac{m}{\omega_2} \frac{d^3 k'_1}{(2 \pi)^{3} 2 \omega'_1} \frac{d^3 k'_2}{(2 \pi)^{3} 2 \omega'_2} d^4 x_1 d^4 x_2
                \end{aligned}
            \end{equation}

            Now, we can insert the values of these commutators, and anticommutators:

            \begin{equation}
                \begin{aligned}
                    \quad [a_{\lambda} (\vec{k}), a^{\dag}_{\lambda'} (\vec{k'})] = (2 \pi)^{3} 2 \omega \delta_{\lambda \lambda'} \delta^{3} (\vec{k} - \vec{k'}) \\
                    {c_{r} (\vec{p}), c^{\dag}_{s} (\vec{p'})} = (2 \pi)^{3} (2 \omega)^{3} \frac{\omega}{m} \delta_{\lambda \lambda'} \delta^{3} (\vec{k} - \vec{k'}) \\
                    {d_{r} (\vec{p}), d^{\dag}_{s} (\vec{p'})} = (2 \pi)^{3} (2 \omega)^{3} \frac{\omega}{m} \delta_{\lambda \lambda'} \delta^{3} (\vec{k} - \vec{k'})
                \end{aligned}
            \end{equation}

            This gives:

            \begin{equation}
                \begin{aligned}
                    \langle \gamma, \gamma | S^{(2)} | e^{-}, e^{+} \rangle = & \\
                    - e^{2} \sum_{r s} \int \langle 0 |[a_{\lambda} (\vec{p'_2}),a^{\dagger}_{h} (\vec{k'_1}] (2 \pi)^{3} 2 \omega \delta_{\lambda \lambda'} \delta^{3} (\vec{k} - \vec{k'}) (2 \pi)^{3} (2 \omega)^{3} & \frac{\omega}{m} \delta_{\lambda \lambda'} \delta^{3} (\vec{k} - \vec{k'}) (2 \pi)^{3} (2 \omega)^{3} \frac{\omega}{m} \delta_{\lambda \lambda'} \delta^{3} (\vec{k} - \vec{k'})| 0 \rangle \\
                    \overline{v}_{s} (\vec{k}_2) \slashed{\epsilon}^{h'} i S_{F} (x_2 - x_1) \slashed{\epsilon}^{h} & u_{r} (\vec{k}_1) e^{- i (k_1 - k'_2) \cdot x} e^{- i (k_2 - k'_1) \cdot x} \\
                    \frac{d^3 k_1}{(2 \pi)^{3}} \frac{m}{\omega_1} \frac{d^3 k_2}{(2 \pi)^{3}} \frac{m}{\omega_2} & \frac{d^3 k'_1}{(2 \pi)^{3} 2 \omega'_1} \frac{d^3 k'_2}{(2 \pi)^{3} 2 \omega'_2} d^4 x_1 d^4 x_2 \\
                    - e^{2} \sum_{r s} \int \langle 0 |[a_{\lambda} (\vec{p'_2}),a^{\dagger}_{h} (\vec{k'_1}] (2 \pi)^{3} 2 \omega \delta_{\lambda \lambda'} \delta^{3} (\vec{k} - \vec{k'}) (2 \pi)^{3} (2 \omega)^{3} & \frac{\omega}{m} \delta_{\lambda \lambda'} \delta^{3} (\vec{k} - \vec{k'}) (2 \pi)^{3} (2 \omega)^{3} \frac{\omega}{m} \delta_{\lambda \lambda'} \delta^{3} (\vec{k} - \vec{k'})| 0 \rangle \\
                    \overline{v}_{s} (\vec{k}_2) \slashed{\epsilon}^{h'} i S_{F} (x_2 - x_1) \slashed{\epsilon}^{h} & u_{r} (\vec{k}_1) e^{- i (k_1 - k'_2) \cdot x} e^{- i (k_2 - k'_1) \cdot x} \\
                    \frac{d^3 k_1}{(2 \pi)^{3}} \frac{m}{\omega_1} \frac{d^3 k_2}{(2 \pi)^{3}} \frac{m}{\omega_2} & \frac{d^3 k'_1}{(2 \pi)^{3} 2 \omega'_1} \frac{d^3 k'_2}{(2 \pi)^{3} 2 \omega'_2} d^4 x_1 d^4 x_2
                \end{aligned}
            \end{equation}

            We can now do all of the k-integrations, spin-sums, and helicity sums:

            \begin{equation}
                \begin{aligned}
                    \langle \gamma, \gamma | S^{(2)} | e^{-}, e^{+} \rangle & = \\
                    -e^{2} \int & \overline{v}_{s'_1} (\vec{p}_1) \slashed{\epsilon}_{\lambda} i S_{F} (p + k) \slashed{\epsilon}_{\lambda'} u_{s'_2} (\vec{p'}_2) e^{- i (p_1 - p'_1) \cdot x_2} e^{- i (p_2 - p'_2) \cdot x_1} d^{4} q d^4 x_1 d^4 x_2 \\
                    -e^{2} \int & \overline{v}_{s'_1} (\vec{p}_1) \slashed{\epsilon}_{\lambda} i S_{F} (p + k) \slashed{\epsilon}_{\lambda'} u_{s'_2} (\vec{p'}_2) e^{- i (p_1 - p'_2) \cdot x_2} e^{- i (p_2 - p'_1) \cdot x_1} d^{4} q d^4 x_1 d^4 x_2
                \end{aligned}
            \end{equation}

            We can rewrite this in terms of Fourier transform of the Fermion Propagator:

            \begin{equation}
                \begin{aligned}
                    \langle \gamma, \gamma | S^{(2)} | e^{-}, e^{+} \rangle & = \\
                    -e^{2} \int & \overline{v}_{s'_1} (\vec{p}_1) \slashed{\epsilon}_{\lambda} i S_{F} (p + k) \slashed{\epsilon}_{\lambda'} u_{s'_2} (\vec{p'}_2) e^{- i (p_2 - p'_2 + q) \cdot x_1} e^{- i (p_1 - p'_1 - q) \cdot x_2} d^4 x_1 d^4 x_2 \\
                    -e^{2} \int & \overline{v}_{s'_1} (\vec{p}_1) \slashed{\epsilon}_{\lambda} i S_{F} (p + k) \slashed{\epsilon}_{\lambda'} u_{s'_2} (\vec{p'}_2) e^{- i (p_2 - p'_1 + q) \cdot x_1} e^{- i (p_1 - p'_2 - q) \cdot x_2} d^4 x_1 d^4 x_2
                \end{aligned}
            \end{equation}

            Now, we can do the x integrations:

            \begin{equation}
                \begin{aligned}
                    \langle \gamma, \gamma | S^{(2)} | e^{-}, e^{+} \rangle & = \\
                    -e^{2} \int & \overline{v}_{s'_1} (\vec{p}_1) \slashed{\epsilon}_{\lambda} i S_{F} (p + k) \slashed{\epsilon}_{\lambda'} u_{s'_2} (\vec{p'}_2) (2 \pi)^4 \delta^{4} (p_1 + p_2 - q) \delta^{4} (p_1 - p'_1 - q) d^{4} q \\
                    -e^{2} \int & \overline{v}_{s'_1} (\vec{p}_1) \slashed{\epsilon}_{\lambda} i S_{F} (p + k) \slashed{\epsilon}_{\lambda'} u_{s'_2} (\vec{p'}_2) (2 \pi)^4 \delta^{4} (p_1 + p_2 - q) \delta^{4} (p_1 - p'_2 - q) d^{4} q
                \end{aligned}
            \end{equation}

            We can then use that same famous delta function identity to get it ready for the q integration:

            \begin{equation}
                \delta^{4} (k - p' + q) \delta^{4} (p - k' - q) = \delta^{4} (p + k - p' - k') \delta^{4} (p - k' - q)
            \end{equation}

            Applying it gives:

            \begin{equation}
                \begin{aligned}
                    \langle \gamma, \gamma | S^{(2)} | e^{-}, e^{+} \rangle & = \\
                    -e^{2} \int & \overline{v}_{s'_1} (\vec{p}_1) \slashed{\epsilon}_{\lambda} i S_{F} (p + k) \slashed{\epsilon}_{\lambda'} u_{s'_2} (\vec{p'}_2) (2 \pi)^4 \delta^{4} (p_1 + p_2 - p'_1 - p'_2) \delta^{4} (p_1 - p'_1 - q) d^{4} q \\
                    -e^{2} \int & \overline{v}_{s'_1} (\vec{p}_1) \slashed{\epsilon}_{\lambda} i S_{F} (p + k) \slashed{\epsilon}_{\lambda'} u_{s'_2} (\vec{p'}_2) (2 \pi)^4 \delta^{4} (p_1 + p_2 - p'_1 - p'_2) \delta^{4} (p_1 - p'_2 - q) d^{4} q
                \end{aligned}
            \end{equation}

            \begin{equation}
                \begin{aligned}
                    \langle \gamma, \gamma | S^{(2)} | e^{-}, e^{+} \rangle & = (2 \pi)^{4} \delta^{4} (p_1 + p_2 - p'_1 - p'_2) \\
                    [ -e^{2} \int & \overline{v}_{s'_1} (\vec{p}_1) \slashed{\epsilon}_{\lambda} i S_{F} (p + k) \slashed{\epsilon}_{\lambda'} u_{s'_2} (\vec{p'}_2) \delta^{4} (p_1 - p'_1 - q) d^{4} q \\
                    -e^{2} \int & \overline{v}_{s'_1} (\vec{p}_1) \slashed{\epsilon}_{\lambda} i S_{F} (p + k) \slashed{\epsilon}_{\lambda'} u_{s'_2} (\vec{p'}_2) \delta^{4} (p_1 - p'_2 - q) d^{4} q]
                \end{aligned}
            \end{equation}

            Now we can finish this off by doing the q integration using the delta function:

            \begin{equation}
                \langle \gamma, \gamma | S^{(2)} | e^{-}, e^{+} \rangle = (2 \pi)^{4} \delta^{4} (p_1 + p_2 - p'_1 - p'_2) [-e^{2} \overline{u}_{s'} (\vec{p}) \slashed{\epsilon}_{\lambda} i S_{F} (p + k) \slashed{\epsilon}_{\lambda'} u_{s} (\vec{p}) - e^{2} \overline{u}_{s'} (\vec{p}) \slashed{\epsilon}_{\lambda} i S_{F} (p + k) \slashed{\epsilon}_{\lambda'} u_{s} (\vec{p})]
            \end{equation}

        \end{framed}

        \begin{framed}

            \begin{equation}
                \langle \gamma, \gamma | S^{(2)} | e^{-}, e^{+} \rangle = (2 \pi)^{4} \delta^{4} (p_1 + p_2 - p'_1 - p'_2) \mathcal{M}_{fi}
            \end{equation}

            \begin{equation}
                \mathcal{M}_{fi} = \mathcal{M}_{fi}^{1} + \mathcal{M}_{fi}^{2}
            \end{equation}

            \begin{equation}
                \mathcal{M}_{fi}^{1} = -e^{2} \overline{v}_{s'_1} (\vec{p}_1) \slashed{\epsilon}_{\lambda} i S_{F} (p + k) \slashed{\epsilon}_{\lambda'} u_{s'_2} (\vec{p'}_2)
            \end{equation}

            \begin{equation}
                \mathcal{M}_{fi}^{2} = -e^{2} \overline{v}_{s'_1} (\vec{p}_1) \slashed{\epsilon}_{\lambda} i S_{F} (p + k) \slashed{\epsilon}_{\lambda'} u_{s'_2} (\vec{p'}_2)
            \end{equation}

        \end{framed}

        \subsection{Pair Production}

        \begin{framed}

            % current position Timestamp: 50:46

            \begin{equation}
                \langle \gamma, \gamma | S^{(2)} | e^{-}, e^{+} \rangle = -e^{2} \int \langle \gamma, \gamma |:\overline{\psi}^{+} (x_1) \gamma^{\mu} A_{\mu}^{-} (x_1) i S_{F} (x_2 - x_1) \gamma^{\nu} A_{\nu}^{-} (x_2) \psi^{+} (x_2):| e^{-}, e^{+} \rangle d^4 x_1 d^4 x_2
            \end{equation}

            \begin{equation}
                \psi^{+} (x) = \sum_{s} \frac{m}{\omega} \int c_{s} (\vec{p}) u_{s} (\vec{p}) e^{- i p \cdot x} \frac{d^3 k}{(2 \pi)^{3}} \qquad \overline{\psi}^{+} (x) = \sum_{s} \frac{m}{\omega} \int d_{s} (\vec{p}) \overline{v}_{s} (\vec{p}) e^{- i p \cdot x} \frac{d^3 k}{(2 \pi)^{3}}
            \end{equation}

            \begin{equation}
                A_{\mu}^{-} (x) = \int \epsilon^{\lambda}_{\mu} a^{\dagger}_{\lambda} e^{i k \cdot x} \frac{d^3 k}{(2 \pi)^{3} 2 \omega}
            \end{equation}

            \begin{equation}
                \begin{aligned}
                    \langle \gamma, \gamma | S^{(2)} | e^{-}, e^{+} \rangle = & - e^{2} \sum_{r s} \int \langle \gamma, \gamma |:d_{s} (\vec{k_1}) a^{\dagger}_{h} (\vec{k'_1}) a^{\dagger}_{h'} (\vec{k'_2}) c_{r} (\vec{k_2}):| e^{-}, e^{+} \rangle \\
                    & \overline{v}_{s} (\vec{k}_2) \slashed{\epsilon}^{h'} i S_{F} (x_2 - x_1) \slashed{\epsilon}^{h} u_{r} (\vec{k}_1) e^{i k_1 \cdot x} e^{- i k'_2 \cdot x} e^{i k_2 \cdot x} e^{- i k'_1 \cdot x} \\
                    & \frac{d^3 k_1}{(2 \pi)^{3}} \frac{m}{\omega_1} \frac{d^3 k_2}{(2 \pi)^{3}} \frac{m}{\omega_2} \frac{d^3 k'_1}{(2 \pi)^{3} 2 \omega'_1} \frac{d^3 k'_2}{(2 \pi)^{3} 2 \omega'_2} d^4 x_1 d^4 x_2
                \end{aligned}
            \end{equation}

            Now we can do the normal ordering:

            \begin{equation}
                \begin{aligned}
                    \langle \gamma, \gamma | S^{(2)} | e^{-}, e^{+} \rangle = & - e^{2} \sum_{r s} \int \langle \gamma, \gamma |a^{\dagger}_{h} (\vec{k'_1}) a^{\dagger}_{h'} (\vec{k'_2}) d_{s} (\vec{k_1}) c_{r} (\vec{k_2})| e^{-}, e^{+} \rangle \\
                    & \overline{v}_{s} (\vec{k}_2) \slashed{\epsilon}^{h'} i S_{F} (x_2 - x_1) \slashed{\epsilon}^{h} u_{r} (\vec{k}_1) e^{- i (k_1 - k'_2) \cdot x} e^{- i (k_2 - k'_1) \cdot x} \\
                    & \frac{d^3 k_1}{(2 \pi)^{3}} \frac{m}{\omega_1} \frac{d^3 k_2}{(2 \pi)^{3}} \frac{m}{\omega_2} \frac{d^3 k'_1}{(2 \pi)^{3} 2 \omega'_1} \frac{d^3 k'_2}{(2 \pi)^{3} 2 \omega'_2} d^4 x_1 d^4 x_2
                \end{aligned}
            \end{equation}

            Now we can pull the operators out of the states:

            \begin{equation}
                \begin{aligned}
                    \langle \gamma, \gamma | S^{(2)} | e^{-}, e^{+} \rangle = & - e^{2} \sum_{r s} \int \langle 0 |a_{\lambda} (\vec{p'_2}) a_{\lambda'} (\vec{p'_1}) a^{\dagger}_{h} (\vec{k'_1}) a^{\dagger}_{h'} (\vec{k'_2}) d_{s} (\vec{k_2}) c_{r} (\vec{k_1}) c^{\dagger}_{r} (\vec{p_1}) d^{\dagger}_{s} (\vec{p_2})| 0 \rangle \\
                    & \overline{v}_{s} (\vec{k}_2) \slashed{\epsilon}^{h'} i S_{F} (x_2 - x_1) \slashed{\epsilon}^{h} u_{r} (\vec{k}_1) e^{- i (k_1 - k'_2) \cdot x} e^{- i (k_2 - k'_1) \cdot x} \\
                    & \frac{d^3 k_1}{(2 \pi)^{3}} \frac{m}{\omega_1} \frac{d^3 k_2}{(2 \pi)^{3}} \frac{m}{\omega_2} \frac{d^3 k'_1}{(2 \pi)^{3} 2 \omega'_1} \frac{d^3 k'_2}{(2 \pi)^{3} 2 \omega'_2} d^4 x_1 d^4 x_2
                \end{aligned}
            \end{equation}

            In the exact same way and for all of the same reasons as with pair annihilation case, the k integration and spin/helicity sums have two
            contributing terms, once the operators have been paired and replaced with commutators/anticommutators. Also just like in the pair
            annihilation case, the photons must have opposite helicity to conserve angular momentum. This means that each photon annihilation operator
            can fail to commute with at most one photon creation operator. This makes the necessary operator reordering as possible. So, we have:

            \begin{equation}
                \begin{aligned}
                    \langle \gamma, \gamma | S^{(2)} | e^{-}, e^{+} \rangle = & - e^{2} \sum_{r s} \int \langle 0 |[a_{\lambda} (\vec{p'_2}),a^{\dagger}_{h} (\vec{k'_1}] [a_{\lambda'} (\vec{p'_1}), a^{\dagger}_{h'} (\vec{k'_2})] \{c_{r} (\vec{k_1}), c^{\dagger}_{r} (\vec{p_1}) \} \{d_{s} (\vec{k_2}), d^{\dagger}_{s} (\vec{p_2}) \}| 0 \rangle \\
                    & \overline{v}_{s} (\vec{k}_2) \slashed{\epsilon}^{h'} i S_{F} (x_2 - x_1) \slashed{\epsilon}^{h} u_{r} (\vec{k}_1) e^{- i (k_1 - k'_2) \cdot x} e^{- i (k_2 - k'_1) \cdot x} \\
                    & \frac{d^3 k_1}{(2 \pi)^{3}} \frac{m}{\omega_1} \frac{d^3 k_2}{(2 \pi)^{3}} \frac{m}{\omega_2} \frac{d^3 k'_1}{(2 \pi)^{3} 2 \omega'_1} \frac{d^3 k'_2}{(2 \pi)^{3} 2 \omega'_2} d^4 x_1 d^4 x_2 \\
                    & - e^{2} \sum_{r s} \int \langle 0 |[a_{\lambda} (\vec{p'_2}),a^{\dagger}_{h} (\vec{k'_1}] [a_{\lambda'} (\vec{p'_1}), a^{\dagger}_{h'} (\vec{k'_2})] \{c_{r} (\vec{k_1}), c^{\dagger}_{r} (\vec{p_1}) \} \{d_{s} (\vec{k_2}), d^{\dagger}_{s} (\vec{p_2}) \}| 0 \rangle \\
                    & \overline{v}_{s} (\vec{k}_2) \slashed{\epsilon}^{h'} i S_{F} (x_2 - x_1) \slashed{\epsilon}^{h} u_{r} (\vec{k}_1) e^{- i (k_1 - k'_2) \cdot x} e^{- i (k_2 - k'_1) \cdot x} \\
                    & \frac{d^3 k_1}{(2 \pi)^{3}} \frac{m}{\omega_1} \frac{d^3 k_2}{(2 \pi)^{3}} \frac{m}{\omega_2} \frac{d^3 k'_1}{(2 \pi)^{3} 2 \omega'_1} \frac{d^3 k'_2}{(2 \pi)^{3} 2 \omega'_2} d^4 x_1 d^4 x_2
                \end{aligned}
            \end{equation}

            Next, we can insert values of all of these commutators and anticommutators:

            \begin{equation}
                \begin{aligned} []
                    [a_{\lambda} (\vec{k}), a^{\dag}_{\lambda'} (\vec{k'})] = (2 \pi)^{3} 2 \omega \delta_{\lambda \lambda'} \delta^{3} (\vec{k} - \vec{k'}) \\
                    {c_{r} (\vec{p}), c^{\dag}_{s} (\vec{p'})} = (2 \pi)^{3} (2 \omega)^{3} \frac{\omega}{m} \delta_{\lambda \lambda'} \delta^{3} (\vec{k} - \vec{k'}) \\
                    {d_{r} (\vec{p}), d^{\dag}_{s} (\vec{p'})} = (2 \pi)^{3} (2 \omega)^{3} \frac{\omega}{m} \delta_{\lambda \lambda'} \delta^{3} (\vec{k} - \vec{k'})
                \end{aligned}
            \end{equation}

            Inserting these gives:

            \begin{equation}
                \langle e^{-}, e^{+} | S^{(2)} | \gamma, \gamma \rangle
            \end{equation}

            Now, for some simplifications, the spin and helicity sums, and the k integration:

            \begin{equation}
                \langle e^{-}, e^{+} | S^{(2)} | \gamma, \gamma \rangle
            \end{equation}

            We can then rewrite it this in terms of the momentum space propagator:

            \begin{equation}
                \langle e^{-}, e^{+} | S^{(2)} | \gamma, \gamma \rangle
            \end{equation}

            Now, the x integration can be done to yield delta functions:

            \begin{equation}
                \langle e^{-}, e^{+} | S^{(2)} | \gamma, \gamma \rangle
            \end{equation}

            \begin{equation}
                \mathcal{M}_{fi} = \mathcal{M}_{fi}^{A} + \mathcal{M}_{fi}^{B}
            \end{equation}

            We can use that delta function identity again:

            \begin{equation}
                \delta (x - a) f (x) = \delta (x - a) f(a)
            \end{equation}

            This gives:

            \begin{equation}
                \langle e^{-}, e^{+} | S^{(2)} | \gamma, \gamma \rangle
            \end{equation}

            The q integrations can now be done:

        \end{framed}

        \begin{framed}

            \begin{equation}
                \langle \gamma, \gamma | S^{(2)} | e^{-}, e^{+} \rangle = (2 \pi)^{4} \delta^{4} (p_1 + p_2 - p'_1 - p'_2) \mathcal{M}_{fi}
            \end{equation}

            \begin{equation}
                \mathcal{M}_{fi} = \mathcal{M}_{fi}^{1} + \mathcal{M}_{fi}^{2}
            \end{equation}

            \begin{equation}
                \mathcal{M}_{fi}^{1} = -e^{2} \overline{v}_{s'_1} (\vec{p}_1) \slashed{\epsilon}_{\lambda} i S_{F} (p + k) \slashed{\epsilon}_{\lambda'} u_{s'_2} (\vec{p'}_2)
            \end{equation}

            \begin{equation}
                \mathcal{M}_{fi}^{2} = -e^{2} \overline{v}_{s'_1} (\vec{p}_1) \slashed{\epsilon}_{\lambda} i S_{F} (p + k) \slashed{\epsilon}_{\lambda'} u_{s'_2} (\vec{p'}_2)
            \end{equation}

        \end{framed}

        \subsection{$e^{-}$ Moller Scattering}

        \begin{framed}

            \begin{equation}
                \langle \gamma, \gamma | S^{(2)} | e^{-}, e^{+} \rangle = -e^{2} \int \langle \gamma, \gamma |:\overline{\psi}^{+} (x_1) \gamma^{\mu} A_{\mu}^{-} (x_1) i S_{F} (x_2 - x_1) \gamma^{\nu} A_{\nu}^{-} (x_2) \psi^{+} (x_2):| e^{-}, e^{+} \rangle d^4 x_1 d^4 x_2
            \end{equation}

            \begin{equation}
                \psi^{+} (x) = \sum_{s} \frac{m}{\omega} \int c_{s} (\vec{p}) u_{s} (\vec{p}) e^{- i p \cdot x} \frac{d^3 k}{(2 \pi)^{3}} \qquad \overline{\psi}^{+} (x) = \sum_{s} \frac{m}{\omega} \int d_{s} (\vec{p}) \overline{v}_{s} (\vec{p}) e^{- i p \cdot x} \frac{d^3 k}{(2 \pi)^{3}}
            \end{equation}

            \begin{equation}
                A_{\mu}^{-} (x) = \int \epsilon^{\lambda}_{\mu} a^{\dagger}_{\lambda} e^{i k \cdot x} \frac{d^3 k}{(2 \pi)^{3} 2 \omega}
            \end{equation}

            \begin{equation}
                \begin{aligned}
                    \langle \gamma, \gamma | S^{(2)} | e^{-}, e^{+} \rangle = & - e^{2} \sum_{r s} \int \langle \gamma, \gamma |a^{\dagger}_{h} (\vec{k'_1}) a^{\dagger}_{h'} (\vec{k'_2}) d_{s} (\vec{k_1}) c_{r} (\vec{k_2})| e^{-}, e^{+} \rangle \\
                    & \overline{v}_{s} (\vec{k}_2) \slashed{\epsilon}^{h'} i S_{F} (x_2 - x_1) \slashed{\epsilon}^{h} u_{r} (\vec{k}_1) e^{- i (k_1 - k'_2) \cdot x} e^{- i (k_2 - k'_1) \cdot x} \\
                    & \frac{d^3 k_1}{(2 \pi)^{3}} \frac{m}{\omega_1} \frac{d^3 k_2}{(2 \pi)^{3}} \frac{m}{\omega_2} \frac{d^3 k'_1}{(2 \pi)^{3} 2 \omega'_1} \frac{d^3 k'_2}{(2 \pi)^{3} 2 \omega'_2} d^4 x_1 d^4 x_2
                \end{aligned}
            \end{equation}

            Now let's insert the momentum state propagator, and do the normal ordering:

            \begin{equation}
                \begin{aligned}
                    \langle \gamma, \gamma | S^{(2)} | e^{-}, e^{+} \rangle = & - e^{2} \sum_{r s} \int \langle \gamma, \gamma |a^{\dagger}_{h} (\vec{k'_1}) a^{\dagger}_{h'} (\vec{k'_2}) d_{s} (\vec{k_1}) c_{r} (\vec{k_2})| e^{-}, e^{+} \rangle \\
                    & \overline{v}_{s} (\vec{k}_2) \slashed{\epsilon}^{h'} i S_{F} (x_2 - x_1) \slashed{\epsilon}^{h} u_{r} (\vec{k}_1) e^{- i (k_1 - k'_2) \cdot x} e^{- i (k_2 - k'_1) \cdot x} \\
                    & \frac{d^3 k_1}{(2 \pi)^{3}} \frac{m}{\omega_1} \frac{d^3 k_2}{(2 \pi)^{3}} \frac{m}{\omega_2} \frac{d^3 k'_1}{(2 \pi)^{3} 2 \omega'_1} \frac{d^3 k'_2}{(2 \pi)^{3} 2 \omega'_2} d^4 x_1 d^4 x_2
                \end{aligned}
            \end{equation}

            \begin{framed}
                We expect the k integrations to yield four terms because there are four different ways of assigning the k values and spin/helicity
                values that don't result in the states being annihilated.
            \end{framed}

            We can pull the creation operators out of the states to get:

            \begin{equation}
                \begin{aligned}
                    \langle \gamma, \gamma | S^{(2)} | e^{-}, e^{+} \rangle = & - e^{2} \sum_{r s} \int \langle 0 |a_{\lambda} (\vec{p'_2}) a_{\lambda'} (\vec{p'_1}) a^{\dagger}_{h} (\vec{k'_1}) a^{\dagger}_{h'} (\vec{k'_2}) d_{s} (\vec{k_2}) c_{r} (\vec{k_1}) c^{\dagger}_{r} (\vec{p_1}) d^{\dagger}_{s} (\vec{p_2})| 0 \rangle \\
                    & \overline{v}_{s} (\vec{k}_2) \slashed{\epsilon}^{h'} i S_{F} (x_2 - x_1) \slashed{\epsilon}^{h} u_{r} (\vec{k}_1) e^{- i (k_1 - k'_2) \cdot x} e^{- i (k_2 - k'_1) \cdot x} \\
                    & \frac{d^3 k_1}{(2 \pi)^{3}} \frac{m}{\omega_1} \frac{d^3 k_2}{(2 \pi)^{3}} \frac{m}{\omega_2} \frac{d^3 k'_1}{(2 \pi)^{3} 2 \omega'_1} \frac{d^3 k'_2}{(2 \pi)^{3} 2 \omega'_2} d^4 x_1 d^4 x_2
                \end{aligned}
            \end{equation}

            Given that we are dealing with fermions which can't be in the same quantum state, we can do some anticommuting that is useful, because
            any one annihilation operator can't possibly correspond to the quantum state of more than one of the creation operators. We can use this
            to pair up creation and annihilation operators. Once creation and annihilation operators are paired up, each of the operators can be 
            replaced by its anticommutator. The anticommutation relations satisfied by the operators tell us that these anticommutators contain delta
            functions and Kronecker deltas that will cause the k integrations and spin sums to fix momentum and spin assignments of all dummy variables
            according to the particular pairing. Because there are four pairings, the integration of four terms, As mentioned in Box A:

            \begin{equation}
                \begin{aligned}
                    \langle e^{-}, e^{-} | S_{EM} | e^{-}, e^{-} \rangle = & - e^{2} \sum_{r s} \int \langle 0 |[a_{\lambda} (\vec{p'_2}),a^{\dagger}_{h} (\vec{k'_1}] [a_{\lambda'} (\vec{p'_1}), a^{\dagger}_{h'} (\vec{k'_2})] \{c_{r} (\vec{k_1}), c^{\dagger}_{r} (\vec{p_1}) \} \{d_{s} (\vec{k_2}), d^{\dagger}_{s} (\vec{p_2}) \}| 0 \rangle \\
                    & \overline{v}_{s} (\vec{k}_2) \slashed{\epsilon}^{h'} i S_{F} (x_2 - x_1) \slashed{\epsilon}^{h} u_{r} (\vec{k}_1) e^{- i (k_1 - k'_2) \cdot x} e^{- i (k_2 - k'_1) \cdot x} \\
                    & \frac{d^3 k_1}{(2 \pi)^{3}} \frac{m}{\omega_1} \frac{d^3 k_2}{(2 \pi)^{3}} \frac{m}{\omega_2} \frac{d^3 k'_1}{(2 \pi)^{3} 2 \omega'_1} \frac{d^3 k'_2}{(2 \pi)^{3} 2 \omega'_2} d^4 x_1 d^4 x_2 \\
                    & - e^{2} \sum_{r s} \int \langle 0 |[a_{\lambda} (\vec{p'_2}),a^{\dagger}_{h} (\vec{k'_1}] [a_{\lambda'} (\vec{p'_1}), a^{\dagger}_{h'} (\vec{k'_2})] \{c_{r} (\vec{k_1}), c^{\dagger}_{r} (\vec{p_1}) \} \{d_{s} (\vec{k_2}), d^{\dagger}_{s} (\vec{p_2}) \}| 0 \rangle \\
                    & \overline{v}_{s} (\vec{k}_2) \slashed{\epsilon}^{h'} i S_{F} (x_2 - x_1) \slashed{\epsilon}^{h} u_{r} (\vec{k}_1) e^{- i (k_1 - k'_2) \cdot x} e^{- i (k_2 - k'_1) \cdot x} \\
                    & \frac{d^3 k_1}{(2 \pi)^{3}} \frac{m}{\omega_1} \frac{d^3 k_2}{(2 \pi)^{3}} \frac{m}{\omega_2} \frac{d^3 k'_1}{(2 \pi)^{3} 2 \omega'_1} \frac{d^3 k'_2}{(2 \pi)^{3} 2 \omega'_2} d^4 x_1 d^4 x_2 \\
                    & - e^{2} \sum_{r s} \int \langle 0 |[a_{\lambda} (\vec{p'_2}),a^{\dagger}_{h} (\vec{k'_1}] [a_{\lambda'} (\vec{p'_1}), a^{\dagger}_{h'} (\vec{k'_2})] \{c_{r} (\vec{k_1}), c^{\dagger}_{r} (\vec{p_1}) \} \{d_{s} (\vec{k_2}), d^{\dagger}_{s} (\vec{p_2}) \}| 0 \rangle \\
                    & \overline{v}_{s} (\vec{k}_2) \slashed{\epsilon}^{h'} i S_{F} (x_2 - x_1) \slashed{\epsilon}^{h} u_{r} (\vec{k}_1) e^{- i (k_1 - k'_2) \cdot x} e^{- i (k_2 - k'_1) \cdot x} \\
                    & \frac{d^3 k_1}{(2 \pi)^{3}} \frac{m}{\omega_1} \frac{d^3 k_2}{(2 \pi)^{3}} \frac{m}{\omega_2} \frac{d^3 k'_1}{(2 \pi)^{3} 2 \omega'_1} \frac{d^3 k'_2}{(2 \pi)^{3} 2 \omega'_2} d^4 x_1 d^4 x_2 \\
                    & - e^{2} \sum_{r s} \int \langle 0 |[a_{\lambda} (\vec{p'_2}),a^{\dagger}_{h} (\vec{k'_1}] [a_{\lambda'} (\vec{p'_1}), a^{\dagger}_{h'} (\vec{k'_2})] \{c_{r} (\vec{k_1}), c^{\dagger}_{r} (\vec{p_1}) \} \{d_{s} (\vec{k_2}), d^{\dagger}_{s} (\vec{p_2}) \}| 0 \rangle \\
                    & \overline{v}_{s} (\vec{k}_2) \slashed{\epsilon}^{h'} i S_{F} (x_2 - x_1) \slashed{\epsilon}^{h} u_{r} (\vec{k}_1) e^{- i (k_1 - k'_2) \cdot x} e^{- i (k_2 - k'_1) \cdot x} \\
                    & \frac{d^3 k_1}{(2 \pi)^{3}} \frac{m}{\omega_1} \frac{d^3 k_2}{(2 \pi)^{3}} \frac{m}{\omega_2} \frac{d^3 k'_1}{(2 \pi)^{3} 2 \omega'_1} \frac{d^3 k'_2}{(2 \pi)^{3} 2 \omega'_2} d^4 x_1 d^4 x_2
                \end{aligned}
            \end{equation}

            Now we can insert the values of the commutators:

            \begin{equation}
                \{c_{r} (\vec{p}), c^{\dag}_{s} (\vec{p'}) \} = (2 \pi)^{3} \frac{\omega}{m} \delta^{3} (\vec{p} - \vec{p'}) \delta_{rs}
            \end{equation}

            \pagebreak 

            Inserting it gives:

            \begin{equation}
                \begin{aligned}
                    \langle e^{-}, e^{-} | S_{EM} | e^{-}, e^{-} \rangle = \\
                    - e^{2} \sum_{r s} \int \langle 0 |[a_{\lambda} (\vec{p'_2}),a^{\dagger}_{h} (\vec{k'_1}] (2 \pi)^{3} 2 \omega \delta_{\lambda \lambda'} \delta^{3} (\vec{k} - \vec{k'}) (2 \pi)^{3} (2 \omega)^{3} & \frac{\omega}{m} \delta_{\lambda \lambda'} \delta^{3} (\vec{k} - \vec{k'}) (2 \pi)^{3} (2 \omega)^{3} \frac{\omega}{m} \delta_{\lambda \lambda'} \delta^{3} (\vec{k} - \vec{k'})| 0 \rangle \\
                    \overline{v}_{s} (\vec{k}_2) \slashed{\epsilon}^{h'} i S_{F} (x_2 - x_1) \slashed{\epsilon}^{h} & u_{r} (\vec{k}_1) e^{- i (k_1 - k'_2) \cdot x} e^{- i (k_2 - k'_1) \cdot x} \\
                    \frac{d^3 k_1}{(2 \pi)^{3}} \frac{m}{\omega_1} \frac{d^3 k_2}{(2 \pi)^{3}} \frac{m}{\omega_2} & \frac{d^3 k'_1}{(2 \pi)^{3} 2 \omega'_1} \frac{d^3 k'_2}{(2 \pi)^{3} 2 \omega'_2} d^4 x_1 d^4 x_2 \\
                    - e^{2} \sum_{r s} \int \langle 0 |[a_{\lambda} (\vec{p'_2}),a^{\dagger}_{h} (\vec{k'_1}] (2 \pi)^{3} 2 \omega \delta_{\lambda \lambda'} \delta^{3} (\vec{k} - \vec{k'}) (2 \pi)^{3} (2 \omega)^{3} & \frac{\omega}{m} \delta_{\lambda \lambda'} \delta^{3} (\vec{k} - \vec{k'}) (2 \pi)^{3} (2 \omega)^{3} \frac{\omega}{m} \delta_{\lambda \lambda'} \delta^{3} (\vec{k} - \vec{k'})| 0 \rangle \\
                    \overline{v}_{s} (\vec{k}_2) \slashed{\epsilon}^{h'} i S_{F} (x_2 - x_1) \slashed{\epsilon}^{h} & u_{r} (\vec{k}_1) e^{- i (k_1 - k'_2) \cdot x} e^{- i (k_2 - k'_1) \cdot x} \\
                    \frac{d^3 k_1}{(2 \pi)^{3}} \frac{m}{\omega_1} \frac{d^3 k_2}{(2 \pi)^{3}} \frac{m}{\omega_2} & \frac{d^3 k'_1}{(2 \pi)^{3} 2 \omega'_1} \frac{d^3 k'_2}{(2 \pi)^{3} 2 \omega'_2} \\
                    - e^{2} \sum_{r s} \int \langle 0 |[a_{\lambda} (\vec{p'_2}),a^{\dagger}_{h} (\vec{k'_1}] (2 \pi)^{3} 2 \omega \delta_{\lambda \lambda'} \delta^{3} (\vec{k} - \vec{k'}) (2 \pi)^{3} (2 \omega)^{3} & \frac{\omega}{m} \delta_{\lambda \lambda'} \delta^{3} (\vec{k} - \vec{k'}) (2 \pi)^{3} (2 \omega)^{3} \frac{\omega}{m} \delta_{\lambda \lambda'} \delta^{3} (\vec{k} - \vec{k'})| 0 \rangle \\
                    \overline{v}_{s} (\vec{k}_2) \slashed{\epsilon}^{h'} i S_{F} (x_2 - x_1) \slashed{\epsilon}^{h} & u_{r} (\vec{k}_1) e^{- i (k_1 - k'_2) \cdot x} e^{- i (k_2 - k'_1) \cdot x} \\
                    \frac{d^3 k_1}{(2 \pi)^{3}} \frac{m}{\omega_1} \frac{d^3 k_2}{(2 \pi)^{3}} \frac{m}{\omega_2} & \frac{d^3 k'_1}{(2 \pi)^{3} 2 \omega'_1} \frac{d^3 k'_2}{(2 \pi)^{3} 2 \omega'_2} d^4 x_1 d^4 x_2 \\
                    - e^{2} \sum_{r s} \int \langle 0 |[a_{\lambda} (\vec{p'_2}),a^{\dagger}_{h} (\vec{k'_1}] (2 \pi)^{3} 2 \omega \delta_{\lambda \lambda'} \delta^{3} (\vec{k} - \vec{k'}) (2 \pi)^{3} (2 \omega)^{3} & \frac{\omega}{m} \delta_{\lambda \lambda'} \delta^{3} (\vec{k} - \vec{k'}) (2 \pi)^{3} (2 \omega)^{3} \frac{\omega}{m} \delta_{\lambda \lambda'} \delta^{3} (\vec{k} - \vec{k'})| 0 \rangle \\
                    \overline{v}_{s} (\vec{k}_2) \slashed{\epsilon}^{h'} i S_{F} (x_2 - x_1) \slashed{\epsilon}^{h} & u_{r} (\vec{k}_1) e^{- i (k_1 - k'_2) \cdot x} e^{- i (k_2 - k'_1) \cdot x} \\
                    \frac{d^3 k_1}{(2 \pi)^{3}} \frac{m}{\omega_1} \frac{d^3 k_2}{(2 \pi)^{3}} \frac{m}{\omega_2} & \frac{d^3 k'_1}{(2 \pi)^{3} 2 \omega'_1} \frac{d^3 k'_2}{(2 \pi)^{3} 2 \omega'_2}
                \end{aligned}
            \end{equation}

            All of the k-integrations can finally be done now, as well as spin sums. Also, all that is within the vacuum expectation value is not an
            operator, and can be pulled out. We can then assume the vacuum state to be normalized. This yields:

            \begin{equation}
                \begin{aligned}
                    \langle e^{-}, e^{-} | S_{EM} | e^{-}, e^{-} \rangle = & \\
                    & \frac{e^{2}}{2} \int \overline{u}_{s'_1} (\vec{p}'_1) \gamma^{\mu} u_{s'_2} (\vec{p}) i D_{\mu \nu}^{F} (q) \overline{u}_{s'_2} (\vec{p}_2) \gamma^{\nu} u_{s_1} (\vec{p}_1) e^{-i \cdot (p_1 - p'_1 + q) \cdot x_1} e{-i \cdot (p_2 - p'_2 - q) \cdot x_2} d^{4} q \\
                    & + \frac{e^{2}}{2} \int \overline{u}_{s'_1} (\vec{p}'_1) \gamma^{\mu} u_{s'_2} (\vec{p}) i D_{\mu \nu}^{F} (q) \overline{u}_{s'_2} (\vec{p}_2) \gamma^{\nu} u_{s_1} (\vec{p}_1) e^{-i \cdot (- p'_2 + p_2 + q) \cdot x_1} e{-i \cdot (p_1 - p'_1 - q) \cdot x_2} d^{4} q \\
                    & \frac{e^{2}}{2} \int \overline{u}_{s'_2} (\vec{p}'_2) \gamma^{\mu} u_{s'_2} (\vec{p}) i D_{\mu \nu}^{F} (q) \overline{u}_{s'_1} (\vec{p}'_1) \gamma^{\nu} u_{s_1} (\vec{p}_1) e^{-i \cdot (p_1 + p_2 + q) \cdot x_1} e{-i \cdot (p_1 - p'2 - q) \cdot x_2} d^{4} q \\
                    & + \frac{e^{2}}{2} \int \overline{u}_{s'_2} (\vec{p}'_2) \gamma^{\mu} u_{s'_2} (\vec{p}) i D_{\mu \nu}^{F} (q) \overline{u}_{s'_1} (\vec{p}'_1) \gamma^{\nu} u_{s_1} (\vec{p}_1) e^{-i \cdot (p_1 - p'_2 + q) \cdot x_1} e{-i \cdot (p_2 - p'_1 - q) \cdot x_2} d^{4} q \\
                \end{aligned}
            \end{equation}

            Now the x integrations can be done to yield the delta functions:

            \pagebreak

            \begin{equation}
                \begin{aligned}
                    \langle e^{-}, e^{-} | S_{EM} | e^{-}, e^{-} \rangle = & \\
                    & \frac{e^{2}}{2} \int \overline{u}_{s'_1} (\vec{p}'_1) \gamma^{\mu} u_{s'_2} (\vec{p}) i D_{\mu \nu}^{F} (q) \overline{u}_{s'_2} (\vec{p}_2) \gamma^{\nu} u_{s_1} (\vec{p}_1) \delta^4 (p_1 - p'_1 + q) \delta^4 (p_2 - p'_2 - q) d^{4} q \\
                    & + \frac{e^{2}}{2} \int \overline{u}_{s'_1} (\vec{p}'_1) \gamma^{\mu} u_{s'_2} (\vec{p}) i D_{\mu \nu}^{F} (q) \overline{u}_{s'_2} (\vec{p}_2) \gamma^{\nu} u_{s_1} (\vec{p}_1) \delta^4 (- p'_2 + p_2 + q) \delta^4 (p_1 - p'_1 - q) d^{4} q \\
                    & \frac{e^{2}}{2} \int \overline{u}_{s'_2} (\vec{p}'_2) \gamma^{\mu} u_{s'_2} (\vec{p}) i D_{\mu \nu}^{F} (q) \overline{u}_{s'_1} (\vec{p}'_1) \gamma^{\nu} u_{s_1} (\vec{p}_1) \delta^4 (p_1 + p_2 + q) \delta^4 (p_1 - p'2 - q) d^{4} q \\
                    & + \frac{e^{2}}{2} \int \overline{u}_{s'_2} (\vec{p}'_2) \gamma^{\mu} u_{s'_2} (\vec{p}) i D_{\mu \nu}^{F} (q) \overline{u}_{s'_1} (\vec{p}'_1) \gamma^{\nu} u_{s_1} (\vec{p}_1) \delta^4 (p_1 - p'_2 + q) \delta^4 (p_2 - p'_1 - q) d^{4} q \\
                \end{aligned}
            \end{equation}

            Now, just like above, we can apply the following identity repeatedly on the products of delta functions:

            \begin{equation}
                \delta (x - a) f (x) = \delta (x - a) f(a)
            \end{equation}

            Doing this gives:

            \begin{equation}
                \begin{aligned}
                    \langle e^{-}, e^{-} | S_{EM} | e^{-}, e^{-} \rangle = & \\
                    & \frac{e^{2}}{2} \int \overline{u}_{s'_1} (\vec{p}'_1) \gamma^{\mu} u_{s'_2} (\vec{p}) i D_{\mu \nu}^{F} (q) \overline{u}_{s'_2} (\vec{p}_2) \gamma^{\nu} u_{s_1} (\vec{p}_1) \delta^4 (p_1 - p'_1 + p_2 - p'_2) \delta^4 (p_2 - p'_2 - q) d^{4} q \\
                    & + \frac{e^{2}}{2} \int \overline{u}_{s'_1} (\vec{p}'_1) \gamma^{\mu} u_{s'_2} (\vec{p}) i D_{\mu \nu}^{F} (q) \overline{u}_{s'_2} (\vec{p}_2) \gamma^{\nu} u_{s_1} (\vec{p}_1) \delta^4 (p_1 - p'_2 + p_2 - p'_1) \delta^4 (p_1 - p'_1 - q) d^{4} q \\
                    & \frac{e^{2}}{2} \int \overline{u}_{s'_2} (\vec{p}'_2) \gamma^{\mu} u_{s'_2} (\vec{p}) i D_{\mu \nu}^{F} (q) \overline{u}_{s'_1} (\vec{p}'_1) \gamma^{\nu} u_{s_1} (\vec{p}_1) \delta^4 (p_1 - p'_1 + p_2 - p'_2) \delta^4 (p_1 - p'2 - q) d^{4} q \\
                    & + \frac{e^{2}}{2} \int \overline{u}_{s'_2} (\vec{p}'_2) \gamma^{\mu} u_{s'_2} (\vec{p}) i D_{\mu \nu}^{F} (q) \overline{u}_{s'_1} (\vec{p}'_1) \gamma^{\nu} u_{s_1} (\vec{p}_1) \delta^4 (p_1 - p'_2 + p_2 - p'_1) \delta^4 (p_2 - p'_1 - q) d^{4} q \\
                \end{aligned}
            \end{equation}

            We can do some factoring:

            \begin{equation}
                \begin{aligned}
                    \langle e^{-}, e^{-} | S_{EM} | e^{-}, e^{-} \rangle = & \\
                    \frac{e^{2}}{2} [\int \overline{u}_{s'_1} (\vec{p}'_1) \gamma^{\mu} u_{s'_2} (\vec{p}) i D_{\mu \nu}^{F} (q) \overline{u}_{s'_2} (\vec{p}_2) \gamma^{\nu} u_{s_1} (\vec{p}_1) d^{4} q & + \int \overline{u}_{s'_1} (\vec{p}'_1) \gamma^{\mu} u_{s'_2} (\vec{p}) i D_{\mu \nu}^{F} (q) \overline{u}_{s'_2} (\vec{p}_2) \gamma^{\nu} u_{s_1} (\vec{p}_1) d^{4} q \\
                    \int \overline{u}_{s'_2} (\vec{p}'_2) \gamma^{\mu} u_{s'_2} (\vec{p}) i D_{\mu \nu}^{F} (q) \overline{u}_{s'_1} (\vec{p}'_1) \gamma^{\nu} u_{s_1} (\vec{p}_1) d^{4} q & + \int \overline{u}_{s'_2} (\vec{p}'_2) \gamma^{\mu} u_{s'_2} (\vec{p}) i D_{\mu \nu}^{F} (q) \overline{u}_{s'_1} (\vec{p}'_1) \gamma^{\nu} u_{s_1} (\vec{p}_1) d^{4} q] \\
                    & (2 \pi)^4 \delta^4 (p_1 + p_2 - p'_1 - p'_2)
                \end{aligned}
            \end{equation}

            Then we can do the q integrations:

            \begin{equation}
                \begin{aligned}
                    \langle e^{-}, e^{-} | S_{EM} | e^{-}, e^{-} \rangle = & \\
                    \frac{e^{2}}{2} [\overline{u}_{s'_1} (\vec{p}'_1) \gamma^{\mu} u_{s'_2} (\vec{p}) i D_{\mu \nu}^{F} (p_1 - p_2) \overline{u}_{s'_2} (\vec{p}_2) \gamma^{\nu} u_{s_1} (\vec{p}_1) & + \overline{u}_{s'_1} (\vec{p}'_1) \gamma^{\mu} u_{s'_2} (\vec{p}) i D_{\mu \nu}^{F} (p_1 - p_2) \overline{u}_{s'_2} (\vec{p}_2) \gamma^{\nu} u_{s_1} (\vec{p}_1) \\
                    \overline{u}_{s'_2} (\vec{p}'_2) \gamma^{\mu} u_{s'_2} (\vec{p}) i D_{\mu \nu}^{F} (p'_1 - p_1) \overline{u}_{s'_1} (\vec{p}'_1) \gamma^{\nu} u_{s_1} (\vec{p}_1) & + \overline{u}_{s'_2} (\vec{p}'_2) \gamma^{\mu} u_{s'_2} (\vec{p}) i D_{\mu \nu}^{F} (p'_1 - p_1) \overline{u}_{s'_1} (\vec{p}'_1) \gamma^{\nu} u_{s_1} (\vec{p}_1)] \\
                    & (2 \pi)^4 \delta^4 (p_1 + p_2 - p'_1 - p'_2)
                \end{aligned}
            \end{equation}

            The remaining delta function enforces some momentum conservation:

            \begin{equation}
                p_{1} + p_{2} = p'_{1} + p'_{2}
            \end{equation}

            We can see that the first two terms are identical and that the last two terms are identical. We can therefore combine them:

            \begin{equation}
                \begin{aligned}
                    \langle e^{-}, e^{-} | S_{EM} | e^{-}, e^{-} \rangle = & \\
                    e^{2} [\overline{u}_{s'_1} (\vec{p}'_1) \gamma^{\mu} u_{s'_2} (\vec{p}) i D_{\mu \nu}^{F} (p_1 - p_2) \overline{u}_{s'_2} (\vec{p}_2) \gamma^{\nu} u_{s_1} (\vec{p}_1) & + \overline{u}_{s'_2} (\vec{p}'_2) \gamma^{\mu} u_{s'_2} (\vec{p}) i D_{\mu \nu}^{F} (p'_1 - p_1) \overline{u}_{s'_1} (\vec{p}'_1) \gamma^{\nu} u_{s_1} (\vec{p}_1)] \\
                    & (2 \pi)^4 \delta^4 (p_1 + p_2 - p'_1 - p'_2)
                \end{aligned}
            \end{equation}

            We can therefore identify the Feynman Amplitude, which I will denote with the usual $\mathcal{M}_{fi}$:
            
            \begin{equation}
                \mathcal{M}_{fi} = e^{2} [\overline{u}_{s'_1} (\vec{p}'_1) \gamma^{\mu} u_{s'_2} (\vec{p}) i D_{\mu \nu}^{F} (p_1 - p_2) \overline{u}_{s'_2} (\vec{p}_2) \gamma^{\nu} u_{s_1} (\vec{p}_1) + \overline{u}_{s'_2} (\vec{p}'_2) \gamma^{\mu} u_{s'_2} (\vec{p}) i D_{\mu \nu}^{F} (p'_1 - p_1) \overline{u}_{s'_1} (\vec{p}'_1) \gamma^{\nu} u_{s_1} (\vec{p}_1)]
            \end{equation}  

        \end{framed}

        \begin{framed}

            \begin{equation}
                \langle e^{-}, e^{-} | S^{(2)} | e^{-}, e^{-} \rangle = (2 \pi)^4 \delta^4 (p_1 + p_2 - p'_1 - p'_2) \mathcal{M}_{fi}
            \end{equation}

            \begin{equation}
                \mathcal{M}_{fi} = \mathcal{M}_{fi}^{1} + \mathcal{M}_{fi}^{2}
            \end{equation}

            \begin{equation}
                \begin{aligned}
                    \mathcal{M}_{fi}^{1} = e^2 \overline{u}_{s'_1} (\vec{p}'_1) \gamma^{\mu} u_{s'_2} (\vec{p}) i D_{\mu \nu}^{F} (p_1 - p_2) \overline{u}_{s'_2} (\vec{p}_2) \gamma^{\nu} u_{s_1} (\vec{p}_1) \\
                    \mathcal{M}_{fi}^{2} = e^2 \overline{u}_{s'_2} (\vec{p}'_2) \gamma^{\mu} u_{s'_2} (\vec{p}) i D_{\mu \nu}^{F} (p'_1 - p_1) \overline{u}_{s'_1} (\vec{p}'_1) \gamma^{\nu} u_{s_1} (\vec{p}_1)
                \end{aligned}
            \end{equation}

        \end{framed}

        \subsection{$e^{+}$ Moller Scattering}

        \begin{framed}

            \begin{equation}
                \langle e^{-}, e^{+} | S^{(2)} | e^{-}, e^{+} \rangle = - \frac{e^2}{2} \langle e^{-}, e^{+} | :\overline{\psi}^{-} (x_1) \gamma^{\mu} \psi^{+} (x_1) i D^{F}_{\mu \nu} (x_2 - x_1) \overline{\psi}^{-} (x_2) \gamma^{\nu} \psi^{+} (x_2): | e^{-}, e^{+} \rangle 
            \end{equation}

            An essentially identical calculation to that of $e^{-}$ Moller scattering yields the following results for $e^{+}$ Moller scattering:
            
            \begin{equation}
                \langle e^{-}, e^{+} | S^{(2)} | e^{-}, e^{+} \rangle = (2 \pi)^4 \delta^4 (p_1 + p_2 - p'_1 - p'_2) \mathcal{M}_{fi}
            \end{equation}

            \begin{equation}
                \mathcal{M}_{fi} = \mathcal{M}_{fi}^{1} + \mathcal{M}_{fi}^{2}
            \end{equation}

            \begin{equation}
                \begin{aligned}
                    \mathcal{M}_{fi}^{1} = e^2 \overline{v}_{s'_1} (\vec{p}'_1) \gamma^{\mu} v_{s'_2} (\vec{p}) i D_{\mu \nu}^{F} (p_1 - p_2) \overline{v}_{s_2} (\vec{p}_2) \gamma^{\nu} v_{s_1} (\vec{p}_1) \\
                    \mathcal{M}_{fi}^{2} = e^2 \overline{v}_{s_2} (\vec{p}_2) \gamma^{\mu} v_{s'_2} (\vec{p}) i D_{\mu \nu}^{F} (p'_1 - p_1) \overline{v}_{s'_1} (\vec{p}'_1) \gamma^{\nu} v_{s_1} (\vec{p}_1)
                \end{aligned}
            \end{equation}

        \end{framed}

    \section{Direct Order of Second Order Amplitudes Continued}

        \subsection{Bhabha Scattering}

        \begin{framed}
            \begin{equation}
                \langle e^{-}, e^{+} | S^{(2)} | e^{-}, e^{+}\rangle = \langle e^{-}, e^{+} | S_{A} | e^{-}, e^{+} \rangle = \langle e^{-}, \gamma | S_{B} | e^{-}, e^{+} \rangle
            \end{equation}

            \begin{equation}
                \begin{aligned}
                    \langle e^{-}, e^{+} | S_{\alpha} | e^{-}, e^{+} \rangle = - e^{2} \int \langle e^{-}, e^{+} |:\overline{\psi}^{-} (x_1) \gamma^{\mu} \psi^{-} (x_1) i S_{F} (x_2 - x_1) \overline{\psi}^{+} (x_2) \gamma^{\nu} \psi^{+} (x_2):| e^{-}, e^{+} \rangle d^{4} x_{1} d^{4} x_{2}
                \end{aligned}
            \end{equation}

            \begin{equation}
                \begin{aligned}
                    \langle e^{-}, e^{+} | S_{\beta} | e^{-}, e^{+} \rangle = - e^{2} \int \langle e^{-}, e^{+} |:\overline{\psi}^{-} (x_1) \gamma^{\mu} \psi^{+} (x_1) i S_{F} (x_2 - x_1) \overline{\psi}^{+} (x_2) \gamma^{\nu} \psi^{-} (x_2):| e^{-}, e^{+} \rangle d^{4} x_{1} d^{4} x_{2}
                \end{aligned}
            \end{equation}

            Let's consider $\langle e^{-}, e^{+} | S^{(2)} | e^{-}, e^{+}\rangle$ first:

            \begin{equation}
                \begin{aligned}
                    \langle e^{-}, e^{+} | S_{\alpha} | e^{-}, e^{+} \rangle = - e^{2} \int \langle e^{-}, e^{+} |:\overline{\psi}^{-} (x_1) \gamma^{\mu} \psi^{-} (x_1) i S_{F} (x_2 - x_1) \overline{\psi}^{+} (x_2) \gamma^{\nu} \psi^{+} (x_2):| e^{-}, e^{+} \rangle d^{4} x_{1} d^{4} x_{2}
                \end{aligned}
            \end{equation}

            \begin{equation}
                \begin{aligned}
                    \psi^{+} (x) = \sum_{s} \int \frac{m}{\omega} c_{s} (\vec{p}) u_{s} (\vec{p}) e^{i p \cdot x} \frac{d^3 p}{(2 \pi)^3} \qquad \overline{\psi}^{+} (x) = \sum_{s} \int \frac{m}{\omega} d_{s} (\vec{p}) \overline{v}_{s} (\vec{p}) e^{i p \cdot x} \frac{d^3 p}{(2 \pi)^3} \\
                    \psi^{-} (x) = \sum_{s} \int \frac{m}{\omega} d_{s}^{\dag} (\vec{p}) v_{s} (\vec{p}) e^{i p \cdot x} \frac{d^3 p}{(2 \pi)^3} \qquad \overline{\psi}^{-} (x) = \sum_{s} \int \frac{m}{\omega} c_{s}^{\dag} (\vec{p}) \overline{u}_{s} (\vec{p}) e^{i p \cdot x} \frac{d^3 p}{(2 \pi)^3}
                \end{aligned}
            \end{equation}

            The operators already happen to be normal ordering can be dropped:

            \begin{equation}
                \begin{aligned}
                    \langle \gamma, \gamma | S^{(2)} | e^{-}, e^{+} \rangle = & - e^{2} \sum_{m n r s} \int \langle 0 | c_{r}^{\dagger} (\vec{k}'_1) d_{s}^{\dagger} (\vec{k}'_2) d_{m} (\vec{k}'_2) c_{n} (\vec{k}'_1) | 0 \rangle \\
                    & e^2 \int \overline{u}_{r} (\vec{p}'_1) \gamma^{\mu} v_{s} (\vec{p}) i D_{\mu \nu}^{F} (x_2 - x_1) \overline{v}_{m} (\vec{p}_2) \gamma^{\nu} u_{n} (\vec{p}_1) (2 \pi)^4 e^{- i k'_1 \cdot x_1} e^{- i k'_2 \cdot x_1} e^{- i k_2 \cdot x_2} e^{- i k_1 \cdot x_2} \\
                    & \frac{d^3 k_1}{(2 \pi)^{3}} \frac{m}{\omega_1} \frac{d^3 k_2}{(2 \pi)^{3}} \frac{m}{\omega_2} \frac{d^3 k'_1}{(2 \pi)^{3}} \frac{m}{\omega_1}  \frac{d^3 k'_2}{(2 \pi)^{3}} \frac{m}{\omega_2} d^4 x_1 d^4 x_2
                \end{aligned}
            \end{equation}

            Since the incoming particles are not identical, and the outgoing particles are not identical, there is only one contribution to all the k integrations and spin sums, and that is where the
            momenta and spins of particles in the initial and final states match annihilation operators of the corresponding type of the matrix element (remember that creation
            operators acting to the left are annihilation operators). This is in contrast with Moller Scattering, where the identical nature of particles causes there to be four such momentum assignments,
            and therefore four contributions to the integrals.

            We can now pull out the creation operators from the states:

            \begin{equation}
                \begin{aligned}
                    \langle \gamma, \gamma | S^{(2)} | e^{-}, e^{+} \rangle = & - e^{2} \sum_{m n r s} \int \langle 0 | c_{s'_1} (\vec{p}'_1) d_{s'_2} (\vec{p}'_2) c_{r}^{\dagger} (\vec{k}'_1) d_{s}^{\dagger} (\vec{k}'_2) d_{m} (\vec{k}'_2) c_{n} (\vec{k}'_1) c_{s_1}^{\dagger} (\vec{p}'_1) d_{s_2}^{\dagger} (\vec{p}'_2) | 0 \rangle \\
                    & e^2 \int \overline{u}_{r} (\vec{p}'_1) \gamma^{\mu} v_{s} (\vec{p}) i D_{\mu \nu}^{F} (x_2 - x_1) \overline{v}_{m} (\vec{p}_2) \gamma^{\nu} u_{n} (\vec{p}_1) (2 \pi)^4 e^{- i k'_1 \cdot x_1} e^{- i k'_2 \cdot x_1} e^{- i k_2 \cdot x_2} e^{- i k_1 \cdot x_2} \\
                    & \frac{d^3 k_1}{(2 \pi)^{3}} \frac{m}{\omega_1} \frac{d^3 k_2}{(2 \pi)^{3}} \frac{m}{\omega_2} \frac{d^3 k'_1}{(2 \pi)^{3}} \frac{m}{\omega_1}  \frac{d^3 k'_2}{(2 \pi)^{3}} \frac{m}{\omega_2} d^4 x_1 d^4 x_2
                \end{aligned}
            \end{equation}

            Because there is only one dummy variable assignment that contributes to the k integrals and spin sums, there is only one way to pair up the creation and
            annihilation operators to yield a nonzero contribution once the operator pairs have been replaced by anticommutators. Otherwise, the process is similar
            to the Moller Scattering Case:

            \begin{equation}
                \begin{aligned}
                    \langle \gamma, \gamma | S^{(2)} | e^{-}, e^{+} \rangle = & - e^{2} \sum_{m n r s} \int \langle 0 | \{ c_{s'_1} (\vec{p}'_1) c_{r}^{\dagger} (\vec{k}'_1) \} \{ d_{s'_2} (\vec{p}'_2) d_{s}^{\dagger} (\vec{k}'_2) \} \{ c_{n} (\vec{k}'_1) c_{s_1}^{\dagger} (\vec{p}'_1) \} \{ d_{m} (\vec{k}'_2) d_{s_2}^{\dagger} (\vec{p}'_2) \} | 0 \rangle \\
                    & e^2 \int \overline{u}_{r} (\vec{p}'_1) \gamma^{\mu} v_{s} (\vec{p}) i D_{\mu \nu}^{F} (x_2 - x_1) \overline{v}_{m} (\vec{p}_2) \gamma^{\nu} u_{n} (\vec{p}_1) (2 \pi)^4 e^{- i k'_1 \cdot x_1} e^{- i k'_2 \cdot x_1} e^{- i k_2 \cdot x_2} e^{- i k_1 \cdot x_2} \\
                    & \frac{d^3 k_1}{(2 \pi)^{3}} \frac{m}{\omega_1} \frac{d^3 k_2}{(2 \pi)^{3}} \frac{m}{\omega_2} \frac{d^3 k'_1}{(2 \pi)^{3}} \frac{m}{\omega_1}  \frac{d^3 k'_2}{(2 \pi)^{3}} \frac{m}{\omega_2} d^4 x_1 d^4 x_2
                \end{aligned}
            \end{equation}
            
            Now we can insert the anticommutator values:

            \begin{equation}
                \begin{aligned}
                    \{c_{r} (\vec{p}), c^{\dag}_{s} (\vec{p'}) \} = (2 \pi)^{3} \frac{\omega}{m} \delta^{3} (\vec{p} - \vec{p'}) \delta_{rs} \\
                \{d_{r} (\vec{p}), d^{\dag}_{s} (\vec{p'}) \} = (2 \pi)^{3} \frac{\omega}{m} \delta^{3} (\vec{p} - \vec{p'}) \delta_{rs}
                \end{aligned}
            \end{equation}

            Doing this gives:

            \begin{equation}
                \begin{aligned}
                    \langle \gamma, \gamma | S^{(2)} | e^{-}, e^{+} \rangle = & - e^{2} \sum_{m n r s} \int \\
                    & \langle 0 | (2 \pi)^3 \frac{\omega'_1}{m} \delta^{3} (\vec{p}'_1 - \vec{k}'_1) \delta_{s'_1 r} (2 \pi)^3 \frac{\omega'_2}{m} \delta^{3} (\vec{p}'_2 - \vec{k}'_2) \delta_{s'_2 s} (2 \pi)^3 \frac{\omega_1}{m} \delta^{3} (\vec{k}_1 - \vec{p}_1) \delta_{n s_1} (2 \pi)^3 \frac{\omega_2}{m} \delta^{3} (\vec{p}_2 - \vec{k}_2) \delta_{m s_2} | 0 \rangle \\
                    & e^2 \int \overline{u}_{r} (\vec{p}'_1) \gamma^{\mu} v_{s} (\vec{p}) i D_{\mu \nu}^{F} (x_2 - x_1) \overline{v}_{m} (\vec{p}_2) \gamma^{\nu} u_{n} (\vec{p}_1) (2 \pi)^4 e^{- i k'_1 \cdot x_1} e^{- i k'_2 \cdot x_1} e^{- i k_2 \cdot x_2} e^{- i k_1 \cdot x_2} \\
                    & \frac{d^3 k_1}{(2 \pi)^{3}} \frac{m}{\omega_1} \frac{d^3 k_2}{(2 \pi)^{3}} \frac{m}{\omega_2} \frac{d^3 k'_1}{(2 \pi)^{3}} \frac{m}{\omega_1}  \frac{d^3 k'_2}{(2 \pi)^{3}} \frac{m}{\omega_2} d^4 x_1 d^4 x_2
                \end{aligned}
            \end{equation}

            \begin{equation}
                \langle e^{-}, e^{+} | S^{(2)}_{\alpha} | e^{-}, e^{+} \rangle = e^2 \int \overline{u}_{s'_1} (\vec{p}'_1) \gamma^{\mu} v_{s'_2} (\vec{p}) i D_{\mu \nu}^{F} (x_2 - x_1) \overline{v}_{s_2} (\vec{p}_2) \gamma^{\nu} u_{s_1} (\vec{p}_1) (2 \pi)^4 e^{- i (- p'_1 - p'_2) \cdot x_1} e^{- i (p_1 + p_2) \cdot x_2}
            \end{equation}

            Now let's insert the momentum space propagators:

            \begin{equation}
                \langle e^{-}, e^{+} | S^{(2)}_{\alpha} | e^{-}, e^{+} \rangle = e^2 \int \overline{u}_{s'_1} (\vec{p}'_1) \gamma^{\mu} v_{s'_2} (\vec{p}) i D_{\mu \nu}^{F} (p_1 + p_2) \overline{v}_{s_2} (\vec{p}_2) \gamma^{\nu} u_{s_1} (\vec{p}_1) (2 \pi)^4 e^{- i (- p'_1 - p'_2 - q) \cdot x_1} e^{- i (p_1 + p_2 - q) \cdot x_2}
            \end{equation}

            We can do the x integrations to yield delta functions:

            \begin{equation}
                \langle e^{-}, e^{+} | S^{(2)}_{\alpha} | e^{-}, e^{+} \rangle = e^2 \int \overline{u}_{s'_1} (\vec{p}'_1) \gamma^{\mu} v_{s'_2} (\vec{p}) i D_{\mu \nu}^{F} (p_1 + p_2) \overline{v}_{s_2} (\vec{p}_2) \gamma^{\nu} u_{s_1} (\vec{p}_1) (2 \pi)^4 \delta^4 (- p'_1 - p'_2 - q) \delta^4 (p_1 + p_2 - q)
            \end{equation}

            We can now apply the usual identity to the delta functions:

            \begin{equation}
                \delta (x - a) f (x) = \delta (x - a) f(a)
            \end{equation}

            Doing this gives:

            \begin{equation}
                \langle e^{-}, e^{+} | S^{(2)}_{\alpha} | e^{-}, e^{+} \rangle = e^2 \int \overline{u}_{s'_1} (\vec{p}'_1) \gamma^{\mu} v_{s'_2} (\vec{p}) i D_{\mu \nu}^{F} (p_1 + p_2) \overline{v}_{s_2} (\vec{p}_2) \gamma^{\nu} u_{s_1} (\vec{p}_1) (2 \pi)^4 \delta^4 (p_1 + p_2 - p'_1 - p'_2) \delta^4 (p_1 + p_2 - q)
            \end{equation}

            \begin{equation}
                \langle e^{-}, e^{+} | S^{(2)}_{\alpha} | e^{-}, e^{+} \rangle = (2 \pi)^4 \delta^4 (p_1 + p_2 - p'_1 - p'_2) \int e^2 \overline{u}_{s'_1} (\vec{p}'_1) \gamma^{\mu} v_{s'_2} (\vec{p}) i D_{\mu \nu}^{F} (p_1 + p_2) \overline{v}_{s_2} (\vec{p}_2) \gamma^{\nu} u_{s_1} (\vec{p}_1) \delta^4 (p_1 + p_2 - q)
            \end{equation}

            Now the q integration can be done:

            \begin{equation}
                \langle e^{-}, e^{+} | S^{(2)}_{\alpha} | e^{-}, e^{+} \rangle = (2 \pi)^4 \delta^4 (p_1 + p_2 - p'_1 - p'_2) e^2 \overline{u}_{s'_1} (\vec{p}'_1) \gamma^{\mu} v_{s'_2} (\vec{p}) i D_{\mu \nu}^{F} (p_1 + p_2) \overline{v}_{s_2} (\vec{p}_2) \gamma^{\nu} u_{s_1} (\vec{p}_1)
            \end{equation}

            A similar calculation can be used to yield the other term:

            \begin{equation}
                \langle e^{-}, e^{+} | S^{(2)}_{\beta} | e^{-}, e^{+} \rangle = (2 \pi)^4 \delta^4 (p_1 + p_2 - p'_1 - p'_2) e^2 \overline{v}_{s_2} (\vec{p}_2) \gamma^{\mu} v_{s'_2} (\vec{p}) i D_{\mu \nu}^{F} (p'_1 - p_1) \overline{u}_{s'_1} (\vec{p}'_1) \gamma^{\nu} u_{s_1} (\vec{p}_1)
            \end{equation}

        \end{framed}

        \begin{framed}

            \begin{equation}
                \langle e^{-}, e^{+} | S^{(2)} | e^{-}, e^{+} \rangle = (2 \pi)^4 \delta^4 (p_1 + p_2 - p'_1 - p'_2) \mathcal{M}_{fi}
            \end{equation}

            \begin{equation}
                \mathcal{M}_{fi} = \mathcal{M}_{fi}^{A} + \mathcal{M}_{fi}^{B}
            \end{equation}

            \begin{equation}
                \begin{aligned}
                    \mathcal{M}_{fi}^{A} = e^2 \overline{u}_{s'_1} (\vec{p}'_1) \gamma^{\mu} v_{s'_2} (\vec{p}) i D_{\mu \nu}^{F} (p_1 + p_2) \overline{v}_{s_2} (\vec{p}_2) \gamma^{\nu} u_{s_1} (\vec{p}_1) \\
                    \mathcal{M}_{fi}^{B} = e^2 \overline{v}_{s_2} (\vec{p}_2) \gamma^{\mu} v_{s'_2} (\vec{p}) i D_{\mu \nu}^{F} (p'_1 - p_1) \overline{u}_{s'_1} (\vec{p}'_1) \gamma^{\nu} u_{s_1} (\vec{p}_1)
                \end{aligned}
            \end{equation}
            
        \end{framed}

        \subsection{Electron Self-Energy}

        \begin{framed}

            \begin{equation}
                \langle e^{-} | S^{(2)} |: e^{-} \rangle = - e^2 \int \langle e^{-} |: \overline{\psi}^{-} (x) \gamma^{\mu} i S_{F} (q) \gamma^{\nu} i D_{\mu \nu}^{F} (p' - q) \psi^{+} (x) :| e^{-} \rangle d^4 x_1 d^4 x_2
            \end{equation}
            
            The operators in the matrix are already normal ordered. We can therefore drop the normal ordering notation:

            \begin{equation}
                \langle e^{-} | S^{(2)} | e^{-} \rangle = - e^2 \int \langle e^{-} | \overline{\psi}^{-} (x) \gamma^{\mu} i S_{F} (q) \gamma^{\nu} i D_{\mu \nu}^{F} (p' - q) \psi^{+} (x) | e^{-} \rangle d^4 x_1 d^4 x_2
            \end{equation}

            Remember from previously:

            \begin{equation}
                \begin{aligned}
                    \psi^{+} (x) | e^{-} \rangle = u_{s} (\vec{p}) e^{- i p \cdot x} | 0 \rangle \\
                    \langle e^{-} | \overline{\psi}^{-} (x) = \langle 0 | e^{i p \cdot x} \overline{u}_{s} (\vec{p})
                \end{aligned}
            \end{equation}

            \begin{equation}
                \langle e^{-} | S^{(2)} | e^{-} \rangle = - e^2 \int \langle 0 | e^{ i p' \cdot x_1} \vec{u}_{s} (\vec{p}) \gamma^{\mu} i S_{F} (q) \gamma^{\nu} i D_{\mu \nu}^{F} (p' - q) u_{s} (\vec{p}) e^{- i p \cdot x_2} | 0 \rangle d^4 x_1 d^4 x_2
            \end{equation}

            \begin{equation}
                \langle e^{-} | S^{(2)} | e^{-} \rangle = - e^2 \int \vec{u}_{s} (\vec{p}) \gamma^{\mu} i S_{F} (q) \gamma^{\nu} i D_{\mu \nu}^{F} (p' - q) u_{s} (\vec{p}) e^{ i p' \cdot x_1} e^{- i p \cdot x_2} d^4 x_1 d^4 x_2
            \end{equation}

            We can now write this in terms of momentum space propagators:

            \begin{equation}
                \langle e^{-} | S^{(2)} | e^{-} \rangle = - e^2 \int \vec{u}_{s} (\vec{p}) \gamma^{\mu} i S_{F} (q) \gamma^{\nu} i D_{\mu \nu}^{F} (p' - q) u_{s} (\vec{p}) e^{- i (- p' + q + k) \cdot x_1} e^{- i (p - q - k) \cdot x_2} d^4 x_1 d^4 x_2 \frac{d^{4} q}{(2 \pi)^4} \frac{d^{4} k}{(2 \pi)^4}
            \end{equation}

            Now we can do the x integration to yield the delta functions:


            \begin{equation}
                \langle e^{-} | S^{(2)} | e^{-} \rangle = - e^2 \int \vec{u}_{s} (\vec{p}) \gamma^{\mu} i S_{F} (q) \gamma^{\nu} i D_{\mu \nu}^{F} (p' - q) u_{s} (\vec{p}) \delta^4 (p' + q + p - q) \delta^4 (p - q - k) d^4 q d^4 k 
            \end{equation}

            We can now use the same delta function identity that we have been using:

            \begin{equation}
                \delta (x - a) f (x) = \delta (x - a) f(a)
            \end{equation}

            This yields:

            \begin{equation}
                \langle e^{-} | S^{(2)} | e^{-} \rangle = - e^2 \int k \vec{u}_{s} (\vec{p}) \gamma^{\mu} i S_{F} (q) \gamma^{\nu} i D_{\mu \nu}^{F} (p' - q) u_{s} (\vec{p}) \delta^4 (p' + q + p - q) \delta^4 (p - q - k) d^4 q d^4 k 
            \end{equation}

            \begin{equation}
                \langle e^{-} | S^{(2)} | e^{-} \rangle = - \delta^4 (p - p') e^2 \int \vec{u}_{s} (\vec{p}) \gamma^{\mu} i S_{F} (q) \gamma^{\nu} i D_{\mu \nu}^{F} (p' - q) u_{s} (\vec{p}) \delta^4 (p - q - k) d^4 q d^4 k 
            \end{equation}

            Now we can do the k-integration:

            \begin{equation}
                \langle e^{-} | S^{(2)} | e^{-} \rangle = - e^2 \int \vec{u}_{s} (\vec{p}) \gamma^{\mu} i S_{F} (q) \gamma^{\nu} i D_{\mu \nu}^{F} (p' - q) v_{s} (\vec{p}) \frac{d^{4} q}{(2 \pi)^4}
            \end{equation}
            
        \end{framed}

        \begin{framed}

            \begin{equation}
                \langle e^{-} | S^{(2)} | e^{-} \rangle = (2 \pi)^4 \delta^{4} (p - p') \mathcal{M}_{fi}
            \end{equation}

            \begin{equation}
                \mathcal{M}_{fi} = - e^2 \int \vec{v}_{s} (\vec{p}) \gamma^{\mu} i S_{F} (q) \gamma^{\nu} i D_{\mu \nu}^{F} (p' - q) v_{s} (\vec{p}) \frac{d^{4} q}{(2 \pi)^4}
            \end{equation}

        \end{framed}

        \subsection{Positron Self-Energy}

        \begin{framed}

            \begin{equation}
                \langle e^{+} | S^{(2)} | e^{+} \rangle = -e^{2} \int :\overline{\psi}^{+} (x_1) \gamma^{\mu} i S_{F} (x_2 - x_1) \gamma^{\nu} i D^{F}_{\mu \nu} (x_2 - x_1) \psi^{-} (x_2): d^4 x_1 d^4 x_2
            \end{equation}

            A calculation essentially identical to the one above for the electron self energy:

            \begin{equation}
                \langle e^{+} | S^{(2)} | e^{+} \rangle = (2 \pi)^4 \delta^{4} (p - p') \mathcal{M}_{fi}
            \end{equation}

            \begin{equation}
                \mathcal{M}_{fi} = - e^2 \int \vec{v}_{s} (\vec{p}) \gamma^{\mu} i S_{F} (q) \gamma^{\nu} i D_{\mu \nu}^{F} (p' - q) v_{s} (\vec{p}) \frac{d^{4} q}{(2 \pi)^4}
            \end{equation}

        \end{framed}

        \subsection{Photon Self-Energy}

        \begin{framed}

            \begin{equation}
                \langle \gamma | S^{(2)} | \gamma \rangle = - e^{2} \int \langle 0 |: Tr [i S_{F} (x_2; x_1) \gamma^{\mu} A_{\mu}^{-} (x) i S_{F} (x_2 - x_1) \gamma^{\nu} A_{\nu}^{+} (x)] :| 0 \rangle d^4 x_1 d^4 x_2
            \end{equation}

            We can pull off a trace of non-operators from the vacuum expectation value:

            \begin{equation}
                \langle \gamma | S^{(2)} | \gamma \rangle = - e^{2} \int Tr [i S_{F} (x_2; x_1) \gamma^{\mu} i S_{F} (x_2 - x_1) \gamma^{\nu}] \langle 0 |: A_{\nu}^{-} (x) A_{\mu}^{+} (x) :| 0 \rangle d^4 x_1 d^4 x_2
            \end{equation}

            Now, we can take care of normal ordering:]

            \begin{equation}
                \langle \gamma | S^{(2)} | \gamma \rangle = - e^{2} \int Tr [i S_{F} (x_2; x_1) \gamma^{\mu} i S_{F} (x_2 - x_1) \gamma^{\nu}] \langle 0 | A_{\nu}^{-} (x) A_{\mu}^{+} (x) | 0 \rangle d^4 x_1 d^4 x_2
            \end{equation}

            \begin{equation}
                A_{\mu}^{+} (x) | \gamma \rangle = \epsilon^{\lambda}_{\mu} e^{- i k \cdot x} | 0 \rangle \quad \langle \gamma | A_{\mu}^{-} (x) = \langle 0 | \epsilon^{\lambda}_{\mu} e^{i k' \cdot x}
            \end{equation}

            \begin{equation}
                \langle \gamma | S^{(2)} | \gamma \rangle = - e^{2} \int Tr [i S_{F} (x_2; x_1) \gamma^{\mu} i S_{F} (x_2 - x_1) \gamma^{\nu}] \langle 0 | \epsilon^{\lambda}_{\nu} e^{- i k \cdot x_1} \epsilon^{\lambda}_{\mu} e^{i k' \cdot x_2} | 0 \rangle d^4 x_1 d^4 x_2
            \end{equation}

            \begin{equation}
                \langle \gamma | S^{(2)} | \gamma \rangle = - e^{2} \int Tr [i S_{F} (x_2; x_1) \slashed{\epsilon}_{\lambda} i S_{F} (x_2 - x_1) \slashed{\epsilon}_{\lambda}] e^{- i k \cdot x_1} e^{i k' \cdot x_2} d^4 x_1 d^4 x_2
            \end{equation}

            Next, we can write this in terms of momentum space propagators:

            \begin{equation}
                \langle \gamma | S^{(2)} | \gamma \rangle = - e^{2} \int Tr [i S_{F} (q) \slashed{\epsilon}_{\lambda} i S_{F} (q) \slashed{\epsilon}_{\lambda}] e^{- i (k + q - p) \cdot x_1} e^{- i (- k' - q + p) \cdot x_2} d^4 x_1 d^4 x_2 \frac{d^4 p}{(2 \pi)^4} \frac{d^4 q}{(2 \pi)^4}
            \end{equation}

            Then we can do the x-integrations to yield the delta functions:

            \begin{equation}
                \langle \gamma | S^{(2)} | \gamma \rangle = - e^{2} \int Tr [i S_{F} (q) \slashed{\epsilon}_{\lambda} i S_{F} (q) \slashed{\epsilon}_{\lambda}] \delta^4 (k + q - p) \delta^4 (- k' - q + q) d^4 p d^4 q
            \end{equation}

            We can now use the same delta function property:

            \begin{equation}
                \delta (x - a) f (x) = \delta (x - a) f(a)
            \end{equation}

            Doing this yields:

            \begin{equation}
                \langle \gamma | S^{(2)} | \gamma \rangle = - e^{2} \int Tr [i S_{F} (q) \slashed{\epsilon}_{\lambda} i S_{F} (q) \slashed{\epsilon}_{\lambda}] \delta^4 (k + q - p) \delta^4 (- k' - q + k + q) d^4 p d^4 q
            \end{equation}

            \begin{equation}
                \langle \gamma | S^{(2)} | \gamma \rangle = - e^{2} (2 \pi)^{(4)} \delta^4 (k - k') \int Tr [i S_{F} (q) \slashed{\epsilon}_{\lambda} i S_{F} (q) \slashed{\epsilon}_{\lambda}] \delta^4 (k + q - p) d^4 p \frac{d^4 q}{(2 \pi)^4}
            \end{equation}

            Now we are ready to do the p integration:

            \begin{equation}
                \langle \gamma | S^{(2)} | \gamma \rangle = - e^{2} (2 \pi)^{(4)} \delta^4 (k - k') \int Tr [i S_{F} (k + q) \slashed{\epsilon}_{\lambda} i S_{F} (q) \slashed{\epsilon}_{\lambda}] \frac{d^4 q}{(2 \pi)^4}
            \end{equation}

        \end{framed}

        \begin{framed}

            \begin{equation}
                \langle \gamma | S^{(2)} | \gamma \rangle = (2 \pi)^{(4)} \delta^4 (k - k') \mathcal{M}_{fi}
            \end{equation}

            \begin{equation}
                M_{fi} = - e^{2} \int Tr [i S_{F} (k + q) \slashed{\epsilon}_{\lambda} i S_{F} (q) \slashed{\epsilon}_{\lambda}] \frac{d^4 q}{(2 \pi)^4}
            \end{equation}

        \end{framed}

        \subsection{Vacuum Self-Energy}

        \begin{framed}

            There are no operators, so the normal ordering notation can be dropped, and the vacuum is normalized, so the inner product of vacuum states yields unity:

            \begin{equation}
                \langle 0 | S^{(2)} | 0 \rangle = \frac{- e^2}{2!} \int \langle 0 |: Tr [i S_{F} (x_1 - x_2) \gamma^{\mu} i S_{F} (x_2 - x_1) \gamma^{\nu} i D_{\mu \nu}^{F} (x_2 - x_1)] :| 0 \rangle d^{4} x_1 d^{4} x_2
            \end{equation}

            Now we can write this in terms of momentum space propagators:

            \begin{equation}
                S^{(2)}_3 = \frac{- e^2}{2!} \int Tr [i S_{F} (x_1 - x_2) \gamma^{\mu} i S_{F} (x_2 - x_1) \gamma^{\nu} i D_{\mu \nu}^{F} (x_2 - x_1)] d^{4} x_1 d^{4} x_2 
            \end{equation}

            Now we can do the x integration to yield delta functions:

            \begin{equation}
                S^{(2)}_3 = \frac{- e^2}{2!} \int Tr [i S_{F} (k) \gamma^{\mu} i S_{F} (p) \gamma^{\nu} i D_{\mu \nu}^{F} (q)] \delta^{4} (- k + p + k - p) \delta^{4} (k - p - q) d^{4} p d^{4} q \frac{d^4 k}{(2 \pi)^4} \frac{d^4 q}{(2 \pi)^4}
            \end{equation}

            We can now apply the usual delta function identity:

            \begin{equation}
                \delta (x - a) f (x) = \delta (x - a) f(a)
            \end{equation}

            This gives:

            \begin{equation}
                S^{(2)}_3 = \frac{- e^2}{2!} \delta^{4} (0) \int Tr [i S_{F} (k) \gamma^{\mu} i S_{F} (p) \gamma^{\nu} i D_{\mu \nu}^{F} (q)] \delta^{4} (- k + p + k - p) \delta^{4} (k - p - q) d^{4} p d^{4} q \frac{d^4 k}{(2 \pi)^4} \frac{d^4 q}{(2 \pi)^4}
            \end{equation}

            \begin{equation}
                S^{(2)}_3 = \frac{- e^2}{2!} \delta^{4} (0) \int Tr [i S_{F} (k) \gamma^{\mu} i S_{F} (p) \gamma^{\nu} i D_{\mu \nu}^{F} (q)] \delta^{4} (k - p - q) d^{4} p \frac{d^4 k}{(2 \pi)^4} \frac{d^4 q}{(2 \pi)^4}
            \end{equation}

            \begin{equation}
                S^{(2)}_3 = \frac{- e^2}{2!} \delta^{4} (0) \int Tr [i S_{F} (k) \gamma^{\mu} i S_{F} (k - q) \gamma^{\nu} i D_{\mu \nu}^{F} (q)] \frac{d^4 k}{(2 \pi)^4} \frac{d^4 q}{(2 \pi)^4}
            \end{equation}

        \end{framed}

        \begin{framed}

            \begin{equation}
                \langle 0 | S^{(2)} | 0 \rangle = (2 \pi)^4 \delta^{4} (0) \mathcal{M}_{fi}
            \end{equation}

            \begin{equation}
                \mathcal{M}_{fi} = \frac{- e^2}{2!} \int Tr [i S_{F} (k) \gamma^{\mu} i S_{F} (k - q) \gamma^{\nu} i D_{\mu \nu}^{F} (q)] \frac{d^4 k}{(2 \pi)^4} \frac{d^4 q}{(2 \pi)^4}
            \end{equation}
        \end{framed}

        Now that we have computed these Feynman amplitudes to their Feynman diagrams, to yield the Feynman rules.

    \section{Derivation of Feynman Rules by Inspection of Feynman Diagrams and Calculated Amplitudes}

    A little review: In Section, we associated the various second order terms in the S-matrix expansion to Feynman
    Diagrams. Then in section 7 and 8, we further evaluated these amplitudes to a simplified form where we could
    identify the Feynman Rules. We will now insert these further evaluated S-matrix terms back into the table from
    Section 6, and then compare the two and read off the Feynman rules for QED. The updated table is as follows:

    \hspace{0.25cm}

    \begingroup
        \begin{longtable}{| p{.20\textwidth} | p{.80\textwidth} |}
        \hline

        $e^{-}$ Compton Scattering &
            \begin{equation}
                \langle e^{-}, \gamma | S^{(2)} | e^{-}, \gamma \rangle = \langle e^{-}, \gamma | S_{A} | e^{-}, \gamma \rangle = \langle e^{-}, \gamma | S_{B} | e^{-}, \gamma \rangle
            \end{equation}

            \begin{equation}
                \begin{aligned}
                    \langle e^{-}, \gamma | S_{a} | e^{-}, \gamma \rangle = - e^{2} \int :\overline{\psi}^{-} (x_1) \gamma^{\mu} A_{\mu}^{+} (x_1) i S_{F} (x_2 - x_1) \gamma^{\nu} A_{\nu}^{-} (x_2) \psi^{+} (x_2): d^{4} x_{1} d^{4} x_{2}
                \end{aligned}
            \end{equation}

            \begin{equation}
                \begin{aligned}
                    \langle e^{-}, \gamma | S_{b} | e^{-}, \gamma \rangle = - e^{2} \int :\overline{\psi}^{-} (x_1) \gamma^{\mu} A_{\mu}^{-} (x_1) i S_{F} (x_2 - x_1) \gamma^{\nu} A_{\nu}^{+} (x_2) \psi^{+} (x_2): d^{4} x_{1} d^{4} x_{2}
                \end{aligned}
            \end{equation}

            \begin{center}
                \begin{tabular}{|c|c|}
                    \hline
                    $\langle e^{-}, \gamma | S_{a} | e^{-}, \gamma \rangle$ & $\langle e^{-}, \gamma | S_{b} | e^{-}, \gamma \rangle$ \\
                    \hline
                    \begin{tikzpicture}
                        \begin{feynman}
                            \vertex [label = right: $x_1$] (a);
                            \vertex [below = of a, label = left: $x_2$] (b);
                            \vertex [above right = of a, label = $\gamma$] (c);
                            \vertex [above left = of a, label = $e^{-}$] (d);
                            \vertex [below left = of b, label = $\gamma$] (e);
                            \vertex [below right = of b, label = $e^{+}$] (f);
            
                            \diagram{
                                (b) -- [fermion] (a);
                                (c) -- [boson] (b);
                                (a) -- [fermion] (d);
                                (e) -- [boson] (a);
                                (b) -- [fermion] (f);
                            };
                        \end{feynman}
                    \end{tikzpicture} & \begin{tikzpicture}
                        \begin{feynman}
                            \vertex [label = right: $x_1$] (a);
                            \vertex [below = of a, label = left: $x_2$] (b);
                            \vertex [above right = of a, label = $e^{+}$] (c);
                            \vertex [above left = of a, label = $e^{-}$] (d);
                            \vertex [below left = of b, label = $e^{-}$] (e);
                            \vertex [below right = of b, label = $e^{+}$] (f);
            
                            \diagram{
                                (b) -- [fermion] (a);
                                (c) -- [fermion] (a);
                                (a) -- [boson] (d);
                                (e) -- [boson] (b);
                                (b) -- [fermion] (f);
                            };
                        \end{feynman}
                    \end{tikzpicture} \\
                    \hline
                \end{tabular} \\
            \end{center} \\

            & The integral just accounts for the fact that the vertices could be located anywhere, as we noted with the first order term. \\

        \hline

        $e^{+}$ Compton Scattering &
            \begin{equation}
                \langle e^{-}, \gamma | S^{(2)} | e^{-}, \gamma \rangle = \langle e^{-}, \gamma | S_{A} | e^{-}, \gamma \rangle = \langle e^{-}, \gamma | S_{B} | e^{-}, \gamma \rangle
            \end{equation}

            \begin{equation}
                \begin{aligned}
                    \langle e^{-}, \gamma | S_{a} | e^{-}, \gamma \rangle = - e^{2} \int :\overline{\psi}^{+} (x_1) \gamma^{\mu} A_{\mu}^{+} (x_1) i S_{F} (x_2 - x_1) \gamma^{\nu} A_{\nu}^{-} (x_2) \psi^{-} (x_2): d^{4} x_{1} d^{4} x_{2}
                \end{aligned}
            \end{equation}

            \begin{equation}
                \begin{aligned}
                    \langle e^{-}, \gamma | S_{b} | e^{-}, \gamma \rangle = - e^{2} \int :\overline{\psi}^{+} (x_1) \gamma^{\mu} A_{\mu}^{-} (x_1) i S_{F} (x_2 - x_1) \gamma^{\nu} A_{\nu}^{+} (x_2) \psi^{-} (x_2): d^{4} x_{1} d^{4} x_{2}
                \end{aligned}
            \end{equation}

            \begin{center}
                \begin{tabular}{|c|c|}
                    \hline
                    $\langle e^{-}, \gamma | S_{a} | e^{-}, \gamma \rangle$ & $\langle e^{-}, \gamma | S_{b} | e^{-}, \gamma \rangle$ \\
                    \hline
                    \begin{tikzpicture}
                        \begin{feynman}
                            \vertex [label = right: $x_2$] (a);
                            \vertex [below = of a, label = left: $x_1$] (b);
                            \vertex [above right = of a, label = $e^{+}$] (c);
                            \vertex [above left = of a, label = $e^{-}$] (d);
                            \vertex [below left = of b, label = $e^{-}$] (e);
                            \vertex [below right = of b, label = $e^{+}$] (f);
            
                            \diagram{
                                (a) -- [fermion] (b);
                                (c) -- [fermion] (a);
                                (a) -- [boson] (d);
                                (e) -- [boson] (b);
                                (b) -- [fermion] (f);
                            };
                        \end{feynman}
                    \end{tikzpicture} & \begin{tikzpicture}
                        \begin{feynman}
                            \vertex [label = right: $x_2$] (a);
                            \vertex [below = of a, label = left: $x_1$] (b);
                            \vertex [above right = of a, label = $\gamma$] (c);
                            \vertex [above left = of a, label = $e^{-}$] (d);
                            \vertex [below left = of b, label = $\gamma$] (e);
                            \vertex [below right = of b, label = $e^{+}$] (f);
            
                            \diagram{
                                (a) -- [fermion] (b);
                                (c) -- [boson] (b);
                                (a) -- [fermion] (d);
                                (e) -- [boson] (a);
                                (b) -- [fermion] (f);
                            };
                        \end{feynman}
                    \end{tikzpicture} \\
                    \hline
                \end{tabular} \\
            \end{center} \\

        \hline

        Pair Annihilation &
            \begin{equation}
                \langle \gamma, \gamma | S^{(2)} | e^{-}, e^{+} \rangle = \langle \gamma, \gamma | S^{(2)}_{PA} | e^{-}, e^{+} \rangle
            \end{equation}

            \begin{equation}
                S^(2)_{PA} = -e^{2} \int :\overline{\psi}^{+} (x_1) \gamma^{\mu} A_{\mu}^{-} (x_1) i S_{F} (x_2 - x_1) \gamma^{\nu} A_{\nu}^{-} (x_2) \psi^{+} (x_2): d^4 x_1 d^4 x_2
            \end{equation}
 
            Looking at this amplitude term, we notice something interesting. Because the outgoing particles are identical (both photons), there are two
            possible Feynman diagrams that we could associate with this term, which differ by an interchange of outgoing photons. When we evaluate these
            amplitudes further, later on in this video, we will find that $\langle \gamma, \gamma | S^{(2)}_{PA} | e^{-}, e^{+} \rangle$ actually produces
            two terms. These terms will only differ by an interchange of the photon polarization vectors, and wll correspond to the two different possible
            Feynman diagrams we have noticed here, which differ by exactly that outgoing photon interchange.

            \begin{center}
                \begin{tabular}{|c|c|}
                    \hline
                    \multicolumn{2}{|c|}{$\langle e^{-}, e^{+} | S^{(2)}_{PA} | \gamma, \gamma \rangle$} \\
                    \hline
                    \begin{tikzpicture}
                        \begin{feynman}
                            \vertex [label = below: $x_1$] (a);
                            \vertex [right = of a,label = above: $x_2$] (b);
                            \vertex [above left = of a, label = $\gamma_2$] (c);
                            \vertex [below left = of a, label = $e^{-}$] (d);
                            \vertex [above right = of b, label = $\gamma_1$] (e);
                            \vertex [below right = of b, label = $e^{+}$] (f);
            
                            \diagram{
                                (b) -- [fermion] (a);
                                (a) -- [boson] (c);
                                (d) -- [fermion] (a);
                                (e) -- [boson] (b);
                                (b) -- [fermion] (f);
                            };
                        \end{feynman}
                    \end{tikzpicture} & \begin{tikzpicture}
                        \begin{feynman}
                            \vertex [label = below: $x_1$] (a);
                            \vertex [right = of a,label = above: $x_2$] (b);
                            \vertex [above left = of a, label = $\gamma_1$] (c);
                            \vertex [below left = of a, label = $e^{-}$] (d);
                            \vertex [above right = of b, label = $\gamma_2$] (e);
                            \vertex [below right = of b, label = $e^{+}$] (f);
            
                            \diagram{
                                (b) -- [fermion] (a);
                                (b) -- [boson] (c);
                                (d) -- [fermion] (a);
                                (e) -- [boson] (a);
                                (b) -- [fermion] (f);
                            };
                        \end{feynman}
                    \end{tikzpicture} \\
                    \hline
                \end{tabular} \\
            \end{center} \\

        \hline

        Pair Production &
            \begin{equation}
                \langle e^{-}, e^{+} | S^{(2)} | \gamma, \gamma \rangle = \langle e^{-}, e^{+} | S^{(2)}_{PP} | \gamma, \gamma \rangle
            \end{equation}

            \begin{equation}
                S^{(2)}_{PP} = -e^{2} \int :\overline{\psi}^{-} (x_1) \gamma^{\mu} A_{\mu}^{+} (x_1) i S_{F} (x_2 - x_1) \gamma^{\nu} A_{\nu}^{+} (x_2) \psi^{-} (x_2): d^4 x_1 d^4 x_2
            \end{equation}

            Here, we have the same situation we saw with pair annihilation, There are two possible Feynman diagrams that could be associated with this
            term that, again just differ by the interchange of the identical photons. This happens anytime the incoming or outgoing particles are identical
            pairs. Just as with pair annihilation, when we evaluate this amplitude further, we will find two terms that differ only by an interchange of
            the photon polarization vectors.

            \begin{center}
                \begin{tabular}{|c|c|}
                    \hline
                    \multicolumn{2}{|c|}{$\langle \gamma, \gamma | S^{(2)}_{PP} | e^{-}, e^{+} \rangle$} \\
                    \hline
                    \begin{tikzpicture}
                        \begin{feynman}
                            \vertex [label = below: $x_1$] (a);
                            \vertex [right = of a,label = above: $x_2$] (b);
                            \vertex [above left = of a, label = $e^{-}$] (c);
                            \vertex [below left = of a, label = $\gamma_1$] (d);
                            \vertex [above right = of b, label = $e^{+}$] (e);
                            \vertex [below right = of b, label = $\gamma_2$] (f);
            
                            \diagram{
                                (b) -- [fermion] (a);
                                (a) -- [fermion] (c);
                                (d) -- [boson] (a);
                                (e) -- [fermion] (b);
                                (b) -- [boson] (f);
                            };
                        \end{feynman}
                    \end{tikzpicture} & \begin{tikzpicture}
                        \begin{feynman}
                            \vertex [label = below: $x_1$] (a);
                            \vertex [right = of a,label = above: $x_2$] (b);
                            \vertex [above left = of a, label = $e^{-}$] (c);
                            \vertex [below left = of a, label = $\gamma_2$] (d);
                            \vertex [above right = of b, label = $e^{+}$] (e);
                            \vertex [below right = of b, label = $\gamma_1$] (f);
            
                            \diagram{
                                (b) -- [fermion] (a);
                                (a) -- [fermion] (c);
                                (d) -- [boson] (b);
                                (e) -- [fermion] (b);
                                (a) -- [boson] (f);
                            };
                        \end{feynman}
                    \end{tikzpicture} \\
                    \hline
                \end{tabular} \\
            \end{center} \\

        \hline

        $e^{-}$ Moller Scattering &
            \begin{equation}
                \langle e^{-}, e^{-} | S^{(2)} | e^{-}, e^{-} \rangle = \langle e^{-}, e^{-} | S_{EM} | e^{-}, e^{-} \rangle
            \end{equation}

            \begin{equation}
                S_{EM} = \frac{-e^2}{2} \int :\overline{\psi}^{-} (x_1) \gamma^{\mu} \psi^{+} (x_1) i D^{F}_{\mu \nu} (x_2 - x_1) \overline{\psi}^{-} (x_2) \gamma^{\nu} \psi^{+} (x_2): d^4 x_1 d^4 x_2
            \end{equation}

            With Moller scattering, we again have a similar situation to what we saw in the last two entries in this table, only this time, it is more extreme.
            Both the incoming particles and the outgoing particles are identical pairs. Therefore, there are four different Feynman diagrams that we could
            associate with this term, and when we evaluate the amplitude further, we will find that it does contain four terms. We will also find that two pairs
            of them are actually identical to the incoming particles this corresponds to the fact that there are only two physically distinct diagrams that could
            be associated with this matrix. Just like the previous ones, these diagrams differby an interchange of two identical particles. The usual selection
            for the two physically distinct diagrams is as follows:

            \begin{center}
                \begin{tabular}{|c|c|}
                    \hline
                    \multicolumn{2}{|c|}{$\langle e^{-}, e^{-} | S_{EM} | e^{-}, e^{-} \rangle$} \\
                    \hline
                    \begin{tikzpicture}
                            \begin{feynman}
                                \vertex [label = below: $x_1$] (a);
                                \vertex [right = of a,label = above: $x_2$] (b);
                                \vertex [above left = of a, label = $e^{-}$] (c);
                                \vertex [below left = of a, label = $e^{-}$] (d);
                                \vertex [above right = of b, label = $e^{-}$] (e);
                                \vertex [below right = of b, label = $e^{-}$] (f);
                
                                \diagram{
                                    (a) -- [boson] (b);
                                    (a) -- [fermion] (c);
                                    (d) -- [fermion] (a);
                                    (e) -- [fermion] (b);
                                    (b) -- [fermion] (f);
                                };
                            \end{feynman}
                        \end{tikzpicture} & \begin{tikzpicture}
                        \begin{feynman}
                            \vertex [label = below: $x_1$] (a);
                            \vertex [right = of a,label = above: $x_2$] (b);
                            \vertex [above left = of a, label = $e^{-}$] (c);
                            \vertex [below left = of a, label = $e^{-}$] (d);
                            \vertex [above right = of b, label = $e^{-}$] (e);
                            \vertex [below right = of b, label = $e^{-}$] (f);
            
                            \diagram{
                                (a) -- [boson] (b);
                                (e) -- [fermion] (a);
                                (d) -- [fermion] (a);
                                (c) -- [fermion] (b);
                                (b) -- [fermion] (f);
                            };
                        \end{feynman}
                    \end{tikzpicture} \\
                    \hline
                \end{tabular} \\
            \end{center} \\

            & When we do evaluate this amplitude further, we will find one other thing. The two terms that we do ultimately end up with have opposite
            signs in addition to the interchanged outgoing electron momenta. This results from the anticommuting property of fermionic creation and
            annihilation operators. This won't happen for the case of bosons becase their associated operators have commuting properties. \\ 

        \hline

        $e^{+}$ Moller Scattering &
            \begin{equation}
                \langle e^{+}, e^{+} | S^{(2)} | e^{+}, e^{+} \rangle = \langle e^{+}, e^{+} | S_{EM} | e^{+}, e^{+} \rangle
            \end{equation}

            \begin{equation}
                S_{PM} = \frac{-e^2}{2} \int :\overline{\psi}^{+} (x_1) \gamma^{\mu} \psi^{-} (x_1) i D^{F}_{\mu \nu} (x_2 - x_1) \overline{\psi}^{+} (x_2) \gamma^{\nu} \psi^{-} (x_2): d^4 x_1 d^4 x_2
            \end{equation}

            The multiplicity of the graphs follows exactly the same description as in the $e^{-}$ Moller Scattering case.

            \begin{center}
                \begin{tabular}{|c|c|}
                    \hline
                    \multicolumn{2}{|c|}{$\langle e^{+}, e^{+} | S_{EM} | e^{+}, e^{+} \rangle$} \\
                    \hline
                    \begin{tikzpicture}
                        \begin{feynman}
                            \vertex [label = below: $x_1$] (a);
                            \vertex [right = of a,label = above: $x_2$] (b);
                            \vertex [above left = of a, label = $e^{+}$] (c);
                            \vertex [below left = of a, label = $e^{+}$] (d);
                            \vertex [above right = of b, label = $e^{+}$] (e);
                            \vertex [below right = of b, label = $e^{+}$] (f);
            
                            \diagram{
                                (a) -- [boson] (b);
                                (a) -- [fermion] (c);
                                (d) -- [fermion] (a);
                                (e) -- [fermion] (b);
                                (b) -- [fermion] (f);
                            };
                        \end{feynman}
                    \end{tikzpicture} & \begin{tikzpicture}
                        \begin{feynman}
                            \vertex [label = below: $x_1$] (a);
                            \vertex [right = of a,label = above: $x_2$] (b);
                            \vertex [above left = of a, label = $e^{+}$] (c);
                            \vertex [below left = of a, label = $e^{+}$] (d);
                            \vertex [above right = of b, label = $e^{+}$] (e);
                            \vertex [below right = of b, label = $e^{+}$] (f);
            
                            \diagram{
                                (a) -- [boson] (b);
                                (e) -- [fermion] (a);
                                (d) -- [fermion] (a);
                                (c) -- [fermion] (b);
                                (b) -- [fermion] (f);
                            };
                        \end{feynman}
                    \end{tikzpicture} \\
                    \hline
                \end{tabular} \\
            \end{center} \\

        \hline

        Bhabha Scattering &
            \begin{equation}
                \langle e^{-}, e^{+} | S^{(2)} | e^{-}, e^{+}\rangle = \langle e^{-}, e^{+} | S_{\alpha} | e^{-}, e^{+} \rangle = \langle e^{-}, \gamma | S_{\beta} | e^{-}, e^{+} \rangle
            \end{equation}

            \begin{equation}
                S_{\alpha} = - e^{2} \int :\overline{\psi}^{-} (x_1) \gamma^{\mu} \psi^{-} (x_1) i D^{F}_{\mu \nu} (x_2 - x_1) \overline{\psi}^{+} (x_2) \gamma^{\nu} \psi^{+} (x_2): d^4 x_1 d^4 x_2
            \end{equation}

            \begin{equation}
                S_{\beta} = - e^{2} \int :\overline{\psi}^{-} (x_1) \gamma^{\mu} \psi^{+} (x_1) i D^{F}_{\mu \nu} (x_2 - x_1) \overline{\psi}^{+} (x_2) \gamma^{\nu} \psi^{-} (x_2): d^4 x_1 d^4 x_2
            \end{equation}

            \begin{center}
                \begin{tabular}{|c|c|}
                    \hline
                    $\langle e^{-}, e^{+} | S_{\alpha} | e^{-}, e^{+} \rangle$ & $\langle e^{-}, \gamma | S_{\beta} | e^{-}, e^{+} \rangle$ \\
                    \hline
                    \begin{tikzpicture}
                        \begin{feynman}
                            \vertex [label = right: $x_1$] (a);
                            \vertex [below = of a, label = left: $x_2$] (b);
                            \vertex [above right = of a, label = $e^{+}$] (c);
                            \vertex [above left = of a, label = $e^{-}$] (d);
                            \vertex [below left = of b, label = $e^{-}$] (e);
                            \vertex [below right = of b, label = $e^{+}$] (f);
            
                            \diagram{
                                (a) -- [boson] (b);
                                (c) -- [fermion] (a);
                                (a) -- [fermion] (d);
                                (e) -- [fermion] (b);
                                (b) -- [fermion] (f);
                            };
                        \end{feynman}
                    \end{tikzpicture} & \begin{tikzpicture}
                        \begin{feynman}
                            \vertex [label = below: $x_1$] (a);
                            \vertex [right = of a,label = above: $x_2$] (b);
                            \vertex [above left = of a, label = $e^{-}$] (c);
                            \vertex [below left = of a, label = $e^{-}$] (d);
                            \vertex [above right = of b, label = $e^{+}$] (e);
                            \vertex [below right = of b, label = $e^{+}$] (f);
            
                            \diagram{
                                (a) -- [boson] (b);
                                (a) -- [fermion] (c);
                                (d) -- [fermion] (a);
                                (e) -- [fermion] (b);
                                (b) -- [fermion] (f);
                            };
                        \end{feynman}
                    \end{tikzpicture} \\
                    \hline
                \end{tabular} \\
            \end{center} \\

        \hline

        Electron Self Energy &
            \begin{equation}
                \langle e^{-} | S^{(2)} | e^{-} \rangle = \langle e^{-} | S_{ESE} | e^{-} \rangle
            \end{equation}

            \begin{equation}
                S_{ESE} = -e^{2} \int :\overline{\psi}^{-} (x_1) \gamma^{\mu} i S_{F} (x_2 - x_1) \gamma^{\nu} i D^{F}_{\mu \nu} (x_2 - x_1) \psi^{+} (x_2): d^4 x_1 d^4 x_2
            \end{equation}

            \begin{center}

                \begin{tikzpicture}
                    \begin{feynman}
                        \vertex [label = right: $e^{-}$] (a);
                        \vertex [below = of a, label = below: $x_1$] (b);
                        \vertex [below = of b, label = above: $x_2$] (c);
                        \vertex [below = of c, label = right: $e^{-}$] (d);
        
                        \diagram{
                            (a) -- [fermion] (b);
                            (b) -- [fermion] (c);
                            (c) -- [boson, half left] (b);
                            (c) -- [fermion] (d);
                        };
                    \end{feynman}
                \end{tikzpicture}
    
            \end{center} \\

        \hline

        Positron Self Energy &
            \begin{equation}
                \langle e^{+} | S^{(2)} | e^{+} \rangle = \langle e^{+} | S_{PSE} | e^{+} \rangle
            \end{equation}

            \begin{equation}
                S_{PSE} = -e^{2} \int :\overline{\psi}^{+} (x_1) \gamma^{\mu} i S_{F} (x_2 - x_1) \gamma^{\nu} i D^{F}_{\mu \nu} (x_2 - x_1) \psi^{-} (x_2): d^4 x_1 d^4 x_2
            \end{equation}

            \begin{center}

                \begin{tikzpicture}
                    \begin{feynman}
                        \vertex [label = right: $e^{+}$] (a);
                        \vertex [below = of a, label = below: $x_2$] (b);
                        \vertex [below = of b, label = above: $x_1$] (c);
                        \vertex [below = of c, label = right: $e^{+}$] (d);
        
                        \diagram{
                            (a) -- [fermion] (b);
                            (b) -- [fermion] (c);
                            (c) -- [boson, half left] (b);
                            (c) -- [fermion] (d);
                        };
                    \end{feynman}
                \end{tikzpicture}
    
            \end{center} \\

        \hline

        Photon Self Energy &
            \begin{equation}
                \langle \gamma | S^{(2)} | \gamma \rangle = \langle \gamma | S_{PhSE} | \gamma \rangle
            \end{equation}

            \begin{equation}
                S_{PhSE} = -e^{2} \int :Tr[i S_{F} (x_1 - x_2) \gamma^{\mu} A_{\mu}^{+} (x_1) S_{F} (x_2 - x_1) \gamma^{\nu} A_{\nu}^{-} (x_2)]: d^4 x_1 d^4 x_2
            \end{equation}

            \begin{center}

                \begin{tikzpicture}
                    \begin{feynman}
                        \vertex [label = right: $\gamma$] (a);
                        \vertex [below = of a, label = below: $x_2$] (b);
                        \vertex [below = of b, label = above: $x_1$] (c);
                        \vertex [below = of c, label = right: $\gamma$] (d);
        
                        \diagram{
                            (a) -- [boson] (b);
                            (b) -- [fermion, half left] (c);
                            (c) -- [fermion, half left] (b);
                            (c) -- [boson] (d);
                        };
                    \end{feynman}
                \end{tikzpicture}
    
            \end{center} \\

        \hline

        Vacuum Energy &
            \begin{equation}
                \langle 0 | S | 0 \rangle = \langle 0 | S^{(2)}_{3} | 0 \rangle
            \end{equation}

            \begin{equation}
                S^{(2)}_{3} = \frac{(-ie)^2}{2!} \int :Tr[i S_{F} (x_1 - x_2) \gamma^{\mu} i S_{F} (x_2 - x_1) \gamma^{\nu} i D^{F}_{\mu \nu} (x_2 - x_1)]: d^4 x_1 d^4 x_2
            \end{equation}

            \begin{center}

                \begin{tikzpicture}
                    \begin{feynman}
                        \vertex [label = above: $x_2$] (a);
                        \vertex [below = of a, label = below: $x_1$] (b);
        
                        \diagram{
                            (a) -- [boson] (b);
                            (a) -- [fermion, half right] (b);
                            (b) -- [fermion, half right] (a);
                        };
                    \end{feynman}
                \end{tikzpicture}
    
            \end{center} \\

        \hline

        \end{longtable}
    \endgroup

    \hspace{0.25cm}

    After carefully scrutinizing each Feynman amplitude and its Associated Feynman diagram, one concludes that the Associated Feynman
    amplitude can be generated from its diagram with the following Feynman rules:

    \begingroup
        \begin{longtable}{| p{.10\textwidth} | p{.25\textwidth} | p{.50\textwidth} |}
            \hline
            \multicolumn{3}{|p{0.75\textwidth}|}{\centering{\textbf{The Feynman Rules for QED}}} \\
            \hline
            1 & \multicolumn{2}{|p{0.75\textwidth}|}{The first step in  the use of Feynman Rules in QED is to write out all topologically distinct Feynman Diagrams containing
            at most a number of vertices equal to which one wishes to study the S-matrix. These diagrams can contain external and internal
            photon lines, external electron lines, external positron lines, internal fermion lines, and two fermion one photon vertices.
            These diagrams must also conserve charge, and respect conservation of angular momentum.} \\
            \hline
            2 & Factor Ordering & 
            \begin{enumerate}
                \item Pick a fermion line and travel along it. Add the factos as described in the following entries of this
                table based on what features are encountered in the diagram.
                \item As one completes this step, each new factor must be placed from right to left (opposite to how reading
                normally works)
                \item Do this for all non-directly connected fermion lines
            \end{enumerate} \\
            \hline
            3 & Momentum conservation & 
            Many of the various factors that show up in the amplitude terms are momentum
            dependent. The momentum values that they are evaluated at are determined by the incoming and outgoing momentum
            values (the sums are equal due to overall conservation), and by \textbf{momentum conservation at each vertex
            within the corresponding diagram.}\\
            \hline
            4 & Two-Fermion One-Photon vertices & For every such vertex, insert the following factor: $- i e \gamma^{\mu}$ \\
            \hline
            5 & Incoming external Electron lines & For Incoming external Electron lines, insert the following factor: $u_{s} (p)$ \\
            \hline
            6 & Outgoing external Electron lines & For Outgoing external Electron lines, insert the following factor: $\overline{u}_{s} (p)$ \\
            \hline
            7 & Incoming external Positron lines & For Incoming external Positron lines, insert the following factor: $v_{s} (p)$ \\
            \hline
            8 & Outgoing external Positron lines & For Outgoing external Positron lines, insert the following factor: $\overline{v}_{s} (p)$ \\
            \hline
            9 & Interchanging external fermion lines & 
            When one encounters a process with two contributing diagrams that are identical apart from interchanged external
            fermion lines, the associated terms have opposite signs. \\
            \hline
            10 & External Photon lines & 
            Contract the Lorentz index on the photon polarization $\epsilon_{\mu}$ vector with the one
            on the associated vertex function. \\
            \hline
            11 & Internal Fermion lines & 
            For internal fermion lines, insert the following factor (momentum space Dirac fermion propagator):
            \begin{equation}
                i S_{F} (p) = \frac{i}{\slashed{p} - m + i (\epsilon = 0)} = \frac{\slashed{p} + m}{p^2 - m^2}
            \end{equation} \\
            \hline
            12 & Internal Photon lines & 
            For internal photon lines, insert the following factor (momentum space photon propagator):
            \begin{equation}
                i D^{F}_{\mu \nu} (p) = - \frac{i g_{\mu \nu}}{q^2 + i(\epsilon = 0)}
            \end{equation} \\
            \hline
            13 & Loop integrations & 
            Integrate over all loop momentum variables that aren't fixed by momentum conservation at vertices. When doing this, use
            the following integration measure:
            \begin{equation}
                \frac{d^4 q}{(2 \pi)^4}
            \end{equation} \\
            \hline
            14 & Loop Integration Coefficient & 
            Divide each term by the factoria of the number of loops in its corresponding diagram (remember that the factoral of zero
            is one). \\
            \hline
            15 & Trace rule & Take a trace anytime that applying the previous rules leads to any uncontracted spinor indices. \\
            \hline
        \end{longtable}
    \endgroup

    Inspection analysis has already shown that these rules can generate the second order terms in the perturbation series, but we have not yet
    shown that these rules can be used to generate the entire perturbative expansion of the s-matrix to all orders of a perturbation theory. This
    is however, our claim. Specifically, the claim is that one can draw out all the topologically distinct Feynman diagrams with the allowed vertex
    and particle lines up to a given order (number of vertices) and then use the rest of the Feynman rules to convert them to every term that shows
    up in the Perturbation series to that order, without missing anything.

    It is however, not to hard to see that it is true

    First, we know that the form of the Interacting Hamiltonian will control the ordering of the factors such that the two-fermion-one-photon vertex
    is the only one possible. Second, Wick expansion ensures that all possible combinations of propagators (internal lines) and spinors and photon
    polarization vectors (external lines) at each vertex shows up in the perturbation series. Third, because each quantum field contains a creation
    and annihilation part, we can see that regardless of perturbative order, multiplying out a given term will ensure that every sensical temporal
    orientation of each vertex will show up in the perturbation series, and with every sensical combinations of external positrons and electrons.

    From the considerations laid out in the last paragraph, we can see that to all orders in perturbation theory, the Feynman Rules yield every kind
    of term that will show up in the perturbation series, and nothing more. The only remaining question is: does it do so with the right multiplicity?
    In the second order example, that we have so far treated, for multiple different reasons, every distinct term showed up twice, (except for vacuum
    energy term). This factor of two canceled against the factor of two factorial in the denominator that arose from the Taylor expansion of the 
    exponential. The Feynman Rules don't require the inclusion of such a factor for most of the terms generated from various Feynman Diagrams (the only
    exception to this is for loop diagrams). Therefore, if the Feynman rules are correct as they currently stand (to which they are), then expanding
    the S-matrix must yield each type of term with multiplicity equal to the factorial of its order divided by the factorial of the loop number, thus
    canceling the multiplicative factors not included by the Feynman rules. One can see that this is the case via the following inspection analysis:

    When one inserts (into the S-matrix term), the creation and annihilation two-term expansion of the quantum field, and multiplies everything out,
    the complete space-time integration over each factor of the Interacting Hamiltonian will ensure that all terms within somme number of interchanges
    of Hamiltonian factors of each other are identical. Therefore, all terms with nonidentical incoming external legs, and nonidentical outgoing external
    legs, we will have the right multiplicity. Fewer identical terms can of course be,  generated this way for term types where some of the quantum fields
    have been contracted into propagators. This is already accounted for in the Feynman rules (rule 14). When one has identical particles in the incoming
    and outgoing set, things are slightly different. As we saw in the second order case, different pairings between the creation and annihilation operators
    yield terms that are topologically distinct Feynman diagrams, and may also yield additional multiplicity on top of that. Multiplicity is only yielded
    if some number of identical pairs of particles shows up in both the incoming set, and the outgoing set (the only example that we saw of this was Moller
    scattering). In order for such situation to occur, the original wick expansion term must contain the same product of quantum fields at least twice,
    each evaluated at different dummy variables. This is however, exactly the situation that inhibits multiplicity via the first mechanism (multiple terms
    proving identical under dummy variable interchange) to exactly the extent that the second mechanism yields multiplicity. Thus the perturbation series
    yields each type of term with the correct multiplicity to match what is produced by the listed Feynman Rules to all order in Perturbation theory.   

\end{document}