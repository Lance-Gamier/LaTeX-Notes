\documentclass[a4]{article}

\usepackage{amsmath}
\usepackage{amssymb}
\usepackage{framed}
\usepackage{mathrsfs}

\usepackage[left = 1cm,right = 1cm, top = 2cm]{geometry}

\begin{document}

    \title{Deriving Yang-Mills Theory And Quantum Chromodynamics}
    \maketitle

    Before moving on to Yang-Mills Theory, there is one more aspect of Electrodynamics that must be noted.
    Remember from the last video that the Quantum Electrodynamics Lagrangian Density in natural units is:

    \begin{equation}
        \mathcal{L}_{QED} = i \hbar c \bar{\psi} \gamma^{\mu} D_{\mu} \psi - m c^{2} \bar{\psi} \psi - \frac{1}{4 \mu_{0}} F_{\mu \nu} F^{\mu \nu}
    \end{equation}

    Where $F_{\mu \nu}$ is the Faraday tensor given by the following Formula:

    \begin{equation}
        F_{\mu \nu} = \partial_{\mu} A_{\nu} - \partial_{\nu} A_{\mu}
    \end{equation}

    The Faraday tensor is also called the elecromagnetic field strength tesor because its components are
    electric and magnetic field, and terefore, this tensor does exactly what it says on the tin, it just 
    gives the electric field strength. Also, the gauge covariant derivative is:

    \begin{equation}
        D_{\mu} \psi = [\partial_{\mu} - i e A_{\mu}] \psi
    \end{equation}

    where e is the charge assiciated with $\psi$, and I am using the opposite sign convention for the gauge
    term compared to my last video. Doing this changes nothing very meaningful. In fact, the convention used
    here seems to be the more common one. The key new piece of insight that I mentioned above as being
    prerequisite for deriving Yang-Mills theory, comes when one evaluates the commutator of covariant derivatives
    applied to $\psi$

    \begin{equation}
        [D_{\mu}, D_{\nu}] \psi = D_{\mu} D_{\nu} \psi - D_{\nu} D_{\mu} \psi = i e (\partial_{\mu} A_{\nu} \partial_{\nu} A_{\mu}) \psi = - i e F_{\mu \nu} \psi
    \end{equation}

    One can see that vghis commutator yields the field strength. This idea of commuting covariant derivatives to get
    the field strength tensor of the theory is very analogous to gravity. Remember that in gravity physics commutes
    spacetime covariant derivativesaplied to a vector to get the Riemannian curvature tensor, which is just the
    gravitational field strength tensor. It is possible to take this analogy quite far. One can define an abstract
    space (usually called a gauge space) who's parametrizations correspond to gauges. One can then call the field
    strength tensor "curvature of gauge space". One can consider the general covariance of GR (invariance under
    arbitrary coordinate transformations) to be its gauge symmetry with gauge group GL(4) (the group of
    4-dimensional arbitrary coordinate transformations). Coordinate sytems are then the gauges, and spacetime
    the gauge space. 

    The key fact from above, however, is that commuting gauge covariant derivatives gives the field strength tensor,
    where the gauge covariant derivative is defined as the derivative operator that lwhen applied to a field, yields
    a derivative that transorms like the original field (of course there is a nuace in the case of gravity physics
    where the covariant derivative adds another vector index, and therefore transforms as a tensor of one higher rank,
    instead of identically, but it is still the same in that the covariant derivative still transforms like a tensor,
    and that is the important thing).

    Yang-Mills theory is very similar to Maxwell's Electromagnetism in that it is a gauge theory based off of a unitary
    isospin group (gauge group), but it differs in one important way. The gauge group is taken to be $SU(N)$ instead of
    $U(1)$. The goal is to construct the simplest free space vector Lagrangian Density, that has a gauge symmetry
    consistent with vhe multiplication table of $SU(N)$ instead of $U(1)$, and has no higher than second order field
    equations so as to yield a unitary quantum field theory. In short, the goal is to try and see what happens if one
    enlarges the gauge group of free-pace electrodynamics, $\mathcal{L} = - \frac{1}{4} F^(\mu \nu) F_{\mu \nu}$, from
    $U(1)$ to $SU(N)$.

    The approach to answering this question that is taken in this video makes use of what was learned from constructing
    the Lagrangian density of quantum electrodynamics. This includes the relationship between the covariant derivatives
    and the field strength tensor introduced at the beginning of this video. Our need to make use of this relationship
    in constructing Yang-Mills theory and QCD was the reason for introducing the QED version of this relationship at the
    beginning. Let's review the process used to construct QED.

    \begin{enumerate}
        \item A global U(1) invariance in the Dirac Lagrangian was noticed
        \item The global invariance was upgraded to a local invariance using a gauge covariant derivative that achieved
        covariance through the introduction of a new four-vector field (called a gauge field) which was taken to transform
        in the manner required to yield invariance of the Lagrangian.
        \item Given the Identical $U(1)$ gauge trasformation properties, the gauge field was taken to be the elecromagnetic
        four-vector potential, and its dynamical Lagrangian was taken to be the free space Maxwell Lagrangian $\mathcal{L}
        = - \frac{1}{4} F_{\mu \nu} F^{\mu \nu}$. This Lagrangian represented the required unique $U(1)$ gauge invariant,
        Lorentz invariant, and unitary (in the qft sense, no higher than second order field equations) gauge field dynamical
        Lagrangian
        \item It was noted that $F_{\mu \nu}$ is given by the commutator of gauge covariant derivatives. This means that the
        gauge field dynamical Lagrangian could have been derived by squaring the field strength yielded by a commutator of
        gauge covariant derivatives.
    \end{enumerate}

    This process gives us a plan of attack for our first attempt at constructing QCD/Yang-Mills theory, the SU(N) generalization
    of QED/Electromagnetism. This approach will turn out to work, so it is also our last attempt. Specifically, the plan is to:

    \begin{enumerate}
        \item Modify the Dirac Lagrangian slightly to have a global $SU(N)$ symmetry. Specifically, the goal is to construct isospin
        multiplets of Dirac spinors that transform globally under fundamental representation of $SU(N)$.
        \item Upgrade this global invriance to a local one by introducing a covariant derivative. One that includes new gauge fields,
        which transform in the manner required to yield gauge covariance. 
        \item Evaluate the commutator of these gauge covariant derivatives to obtain the field strength.
        \item Then square the field strength to get the gauge field dynamical Lagrangian. It turns out that the trace of the square
        must be taken to yield a gauge invariant scalar. You will see what I'm talking about.
    \end{enumerate}

    The gauge field dynamical Lagrangian will turn out to be the unique gauge invariant, Lorentz invariant, and unitary (no higher
    than second order field equations) $SU(N)$ gauge field dynamical Lagrangian, It is the Lagrangian of Yang-Mills theory, and the
    direct $SU(N)$ generalization of Electromagnetism. The $SU(N)$ gauge invariant Dirac Lagrangian added to te Yang-Mills Lagrangian
    is the Lagrangian of QCD. Let us now perform this process.

    First, remember that generators of the fundamental representation of $SU(N)$ satisfy the following Lie Algebra:

    \begin{equation}
        [\tau^{\alpha}, \tau^{\beta}] = i f^{\alpha \beta}_{c} \tau^{c} \qquad \tau^{\alpha} = \frac{\lambda^{\alpha}}{2}
    \end{equation}

    where $\lambda^{\alpha}$ are the Gell-Mann matrices in the case of $SU(3)$, but are a larger or smaller set of matrices for other
    $SU(N)$. Now, one can define an Isospin multiplet in the fundamental representation as follows: 

    \begin{equation}
        \psi_{i} \rightarrow \overline{\psi}_{j} \Omega_{-1 j}^{i} 
    \end{equation}

    where obviously this is a unitary matrix, so:

    \begin{equation}
        \Omega_{i}^{j} = [exp(-i \Theta^{\alpha} \tau_{\alpha})]_{i}^{j}
    \end{equation}

    If one takes this isospin multiplet to be a multiplet of Dirac spinors, then their Lagrangian is as follows:

    \begin{eqnarray}
        \mathcal{L} = \bar{\psi}_{i} (i \delta^{ij} \gamma^{\mu} \partial_{\mu} - m \delta^{ij}) \psi_{i}
    \end{eqnarray}

    This Lagrangian is like the original Dirac Lagrangian, but with global $SU(N)$ symmetry, with the Dirac field transforming in the
    fundamental representation. Next on our list of step is to upgrade this to a local symmetry using covariant derivatives and gauge
    fields. This can be done with the following gauge covariant derivative:

    \begin{equation}
        D^{ij}_{\mu} = (\delta^{i j} \partial_{\mu} - i g \tau_{a} A_{\mu}^{a}) \psi_{j}
    \end{equation}

    \begin{equation}
        \Omega_{i}^{j} (x) = [exp (-i \theta^{a} (x) \tau_{a})]_{i}^{j}
    \end{equation}

    Where

    \begin{equation}
        (\tau_{a}^{ij}) = - i \frac{i}{g} [\partial_{\mu} \Omega (x)] \Omega^{-1} (x) + \Omega (x) \tau_{b} (x) \Omega^{-1} (x)
    \end{equation}

    \begin{equation}
        A^{a}_{\mu} \rightarrow A'^{a}_{\mu} = 2 Tr [- i \tau^{a} \frac{i}{g} [\partial_{\mu} \Omega (x)] \Omega^{-1} (x)] + 2 Tr [\tau^{a} \Omega (x) \tau_{b} (x) \Omega^{-1} (x)] A_{\mu}^{b}
    \end{equation}

    One can show that these transformation relations are consistent with each other by finding the second one through algebraic manipulations
    of the first one:

    \begin{equation}
        \tau_{b} A'^{b}_{\mu} = - i \tau^{a} \frac{i}{g} [\partial_{\mu} \Omega (x)] \Omega^{-1} (x) + \Omega (x) \tau_{b} (x) \Omega^{-1} (x)
    \end{equation}

    \begin{equation}
        \tau^{a} \tau_{b} A'^{b}_{\mu} (x) = - i \tau^{a} \frac{i}{g} [\partial_{\mu} \Omega (x)] \Omega^{-1} (x) + \tau^{a} \Omega (x) \tau_{b} (x) \Omega^{-1} (x)
    \end{equation}

    \begin{equation}
        Tr[\tau^{a} \tau_{b}] A'^{b}_{\mu} (x) = Tr [- i \tau^{a} \frac{i}{g} [\partial_{\mu} \Omega (x)] \Omega^{-1} (x) + \tau^{a} \Omega (x) \tau_{b} (x) \Omega^{-1} (x)]
    \end{equation}

    Remember the identity:

    \begin{equation}
        Tr[\tau^{a} \tau_{b}] = \frac{1}{2} \delta^{a}_{b}
    \end{equation}

    Applying this:

    \begin{equation}
        \frac{1}{2} \delta_{b}^{a} A'^{b}_{\mu} (x) = Tr [- i \tau^{a} \frac{i}{g} [\partial_{\mu} \Omega (x)] \Omega^{-1} (x)] + Tr [\tau^{a} \Omega (x) \tau_{b} (x) \Omega^{-1} (x)] A_{\mu}^{b}
    \end{equation}

    \begin{equation}
        A'^{b}_{\mu} (x) = 2 Tr [- i \tau^{a} \frac{i}{g} [\partial_{\mu} \Omega (x)] \Omega^{-1} (x)] + 2 Tr [\tau^{a} \Omega (x) \tau_{b} (x) \Omega^{-1} (x)] A_{\mu}^{b}
    \end{equation}

    So, they are consistent. If one considers an infinitesimal gauge transformation, then one finds that gauge field transformations are
    generated by the structure constants $f^{ab}_c$. This means that the gauge fields transforms in the adjoint representation. Now
    looking back at the covariant derivative, with these gauge field transformations, the covariant derivative transforms covariantly:

    \begin{equation}
        \begin{aligned}
            D_{\mu} \psi \rightarrow (D_{\mu} \psi)' = \partial_{\mu} - i g A'_{\mu} \psi' \\
            = \partial_{\mu} (\Omega (x) \psi) + i g [\partial_{\mu} \Omega (x)] \Omega ^{-1} (x) \Omega (x) \psi - i g \Omega (x) A_{\mu} (x) \Omega^{-1} (x) \Omega (x) \psi \\
            = \Omega (x) \partial_{\mu} \psi - i g \Omega (x) A_{\mu} (x) \psi = \Omega (x) D_{\mu} \psi
        \end{aligned}
    \end{equation}

    The gauge invariant Dirac Lagrangian then is:

    \begin{equation}
        \mathcal{L} = \overline{\psi}_{i} (i \gamma^{\mu} D_{\mu}^{ij} - m \delta^{ij}) \psi_{j}
    \end{equation}

    Step three from the above list consists of commuting the covariant derivative to get the field strength tensor:

    \begin{equation}
        [D_{\mu}, D_{\nu}]^{ij} \psi_{j} = D_{k \mu}^{i} D^{k j}_{\nu} \psi_{j} - D_{k \nu}^{i} D^{k j}_{\mu} \psi_{j} = - i g G_{\mu \nu}^{ij} \psi_{j}
    \end{equation}

    \begin{equation}
        G_{\mu \nu}^{ij} = \tau^{ij}_{a} G_{\mu \nu}^{a}
    \end{equation}

    \begin{equation}
        G_{\mu \nu}^{a} = \partial_{\mu} A^{a}_{\nu} - \partial_{\nu} A^{a}_{\mu} + g f^{a}_{bc} A^{b}_{\mu} A^{c}_{\nu}
    \end{equation}

    \begin{equation}
        G_{\mu \nu}^{ij} \rightarrow \Omega^{i}_{k} (x) G_{\mu \nu}^{k l} \Omega^{-1 j}_l (x)
    \end{equation}

    The final step then consists of squaring the field strength tensor to obtain the gauge field dynamical Lagrangian density:

    \begin{equation}
        \mathcal{L}_{Yang-Mills} = - \frac{1}{2} Tr[G_{\mu \nu} G^{\mu \nu}] = - \frac{1}{4} G_{\mu \nu}^{a} G^{\mu \nu}_{a}
    \end{equation}

    The factor of $- \frac{1}{2}$ is a convention. This Lagrangian is gauge invariant because:

    \begin{equation}
        Tr [G_{\mu \nu} G^{\mu \nu}] \rightarrow Tr [\Omega (x) G_{\mu \nu}^{k l} (x) \Omega^{-1} (x) \Omega (x) G_{\mu \nu}^{k l} \Omega^{-1} (x)] = Tr [G_{\mu \nu} G^{\mu \nu}]
    \end{equation}

    because $G_{\mu \nu} G^{\mu \nu}$ is a matrix, and because one must use the cyclic property of the trace to show gauge invariance,
    the trace to show gauge invariance, the trace of the square of the field strength must be taken as the Lagrangian density. This is
    the Lagrangian of Yang-Mills theory, and it is the unique gauge invariant, Lorentz invariant, and unitary (no higher than second
    field equations) $SU(N)$ gauge field dynamical Lagrangian.

    If one takes the special case of $SU(3)$, and then adds it to the locally $SU(3)$ invariant Dirac Lagrangian, one obtains the QCD 
    Lagrangian Density, which is the experimentally correct theory of the strong nuclear interactions:

    \begin{equation}
        \mathcal{L}_{QCD} = \mathcal{L} = \overline{\psi}_{i} (i \gamma^{\mu} D_{\mu}^{ij} - m \delta^{ij}) \psi_{j} - \frac{1}{4} G_{\mu \nu}^{a} G^{\mu \nu}_{a}
    \end{equation}
    
\end{document}