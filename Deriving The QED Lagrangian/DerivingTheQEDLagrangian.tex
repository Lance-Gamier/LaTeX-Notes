\documentclass[a4]{article}

\usepackage{amsmath}
\usepackage{amssymb}
\usepackage{framed}
\usepackage{mathrsfs}

\usepackage[left = 1cm,right = 1cm, top = 2cm]{geometry}

\begin{document}

    \title{Deriving The Quantum Electrodynamics Lagrangian Density}
    \maketitle

    In a previous video I explained how to derive the Dirac Lagrangian from group theory (and therefore the Dirac Equation).
    In that video, I worked in natural units. In this video, I will be working in MKS units. The Lagrangian derived in that
    video, expressed in MKS units is:

    \begin{equation}
        \mathcal{L}_{Dirac} = i \hbar c \bar{\psi} \gamma^{\mu} \partial_{\mu} \psi - m c^{2} \bar{\psi} \psi
    \end{equation}

    One may notice that this action isn't just Lorentz invariant. It also has the following global symmetry:

    \begin{equation}
        \psi \rightarrow e^{\frac{i e \Lambda}{h c}} \psi
    \end{equation}

    This relation of course implies:

    \begin{equation}
        \bar{\psi} \rightarrow e^{- \frac{i e \Lambda}{h c}} \bar{\psi}
    \end{equation}

    This is a manifestly valid symmetry of the Dirac Lagrangian for constant $\Lambda$. Because the exponents of successive
    transformations add, this constitutes a global $U(1)$ invariance. One might then ask the following question. What would
    happen if this was upgraded to a local $U(1)$ transformation, meaniong that $\Lambda$ now depends on the spacetime location:
    $\Lambda \rightarrow \Lambda (x)$. The answer is that the Dirac Lagrangian is not locally invariant because
    $e^{\frac{i e \Lambda (x)}{\hbar c}}$ is a function of spacetime, and can not be commuted with the partial derivative. This
    question is to introduce a new four-vector field that transforms under this $U(1)$ gauge symmetry in whatever manner is
    required to cancel the term that destroys $U(1)$ gauge invariance in the original Dirac Lagrangian. Such a field is called
    a gauge field. Specifically, if one replaces the partial derivative with what is called a gauge covariant derivative:

    \begin{equation}
        \partial_{\mu} \psi \rightarrow D_{\mu} \psi = \Bigg[ \partial_{\mu} + \frac{i e}{\hbar c} A_{\mu} \Bigg] \psi
    \end{equation}

    where $A_{\mu}$ is the new vector field that was mentioned, and is taken to transform under $U(1)$ gauge transformation as
    follows:

    \begin{equation}
        A_{\mu} \rightarrow A_{\mu} - \partial_{\mu} \Lambda (x)
    \end{equation}

    Where this transformation law is consistent with U(1) because transformations simply add to each other. Then one finds that
    the kinetic term is now locally $U(1)$. Specifically, one finds the following Lagrangian Density:

    \begin{equation}
        \mathcal{L} = i \hbar c \bar{\psi} \gamma^{\mu} D_{\mu} \psi - m c ^{2} \bar{\psi} \psi
    \end{equation}

    One can see that it is invariant in the following way:

    \begin{equation}
        \psi \rightarrow e^{\frac{i e \Lambda (x)}{\hbar c}} \psi
    \end{equation}

    \begin{equation}
        \bar{\psi} \rightarrow \bar{\psi} e^{- \frac{i e \Lambda (x)}{\hbar c}}
    \end{equation}

    \begin{equation}
        A_{\mu} \rightarrow A_{\mu} - \partial_{\mu} \Lambda (x)
    \end{equation}

    \begin{equation}
        \begin{aligned}
            D_{\mu} \psi = \Big[ \partial_{\mu} + \frac{i e}{\hbar c} \Big] \psi \rightarrow \Big[ \partial_{\mu} + \frac{i e}{\hbar c} - \frac{i e}{\hbar c} \partial_{\mu} \Lambda (x) \Big] e^{- \frac{i e \Lambda (x)}{\hbar c}} \psi \\
            = \Big[ \frac{i e}{\hbar c} \psi \Big] + \frac{i e}{\hbar c} e^{- \frac{i e \Lambda (x)}{\hbar c}} \psi A_{\mu} - \frac{i e}{\hbar c} \psi \partial_{\mu} \Lambda (x) \\
            = e^{- \frac{i e \Lambda (x)}{\hbar c}} \partial_{\mu} \psi + \psi \partial_{\mu} e^{- \frac{i e \Lambda (x)}{\hbar c}} \psi A_{\mu} + \frac{i e}{\hbar c} e^{- \frac{i e \Lambda (x)}{\hbar c}} -  \frac{i e}{\hbar c} e^{- \frac{i e \Lambda (x)}{\hbar c}} \psi \partial_{\mu} \Lambda (x) \\
            = e^{- \frac{i e \Lambda (x)}{\hbar c}} \partial_{\mu} \psi + e^{- \frac{i e \Lambda (x)}{\hbar c}} \partial_{\mu} \Lambda (x) + e^{- \frac{i e \Lambda (x)}{\hbar c}} \psi A_{\mu} - \partial_{\mu} \psi e^{- \frac{i e \Lambda (x)}{\hbar c}} \psi \partial_{\mu} \Lambda (x) \\
            = e^{- \frac{i e \Lambda (x)}{\hbar c}} \partial_{\mu} \psi + e^{- \frac{i e \Lambda (x)}{\hbar c}} \psi \partial_{\mu} \Lambda (x) + e^{- \frac{i e \Lambda (x)}{\hbar c}} \psi A_{\mu} - \partial_{\mu} \psi e^{- \frac{i e \Lambda (x)}{\hbar c}} \psi \partial_{\mu} \Lambda (x) \\
            = e^{- \frac{i e \Lambda (x)}{\hbar c}} \partial_{\mu} \psi + e^{- \frac{i e \Lambda (x)}{\hbar c}} \psi A_{\mu} = e^{- \frac{i e \Lambda (x)}{\hbar c}} \Big[ \partial_{\mu} + \frac{i e}{\hbar c} \Big] \psi = e^{- \frac{i e \Lambda (x)}{\hbar c}} D_{\mu} \psi
        \end{aligned}
    \end{equation}

    \begin{framed}
        \begin{equation}
            D_{\mu} \psi \rightarrow e^{- \frac{i e \Lambda (x)}{\hbar c}} D_{\mu} \psi
        \end{equation}
    \end{framed}

    \begin{framed}
        \begin{equation}
            \bar{\psi} \gamma^{\mu} D_{\mu} \psi \rightarrow \bar{\psi} \gamma^{\mu} D_{\mu} \psi
        \end{equation}
    \end{framed}

    With this new $U(1)$ gauge invariant Dirac Lagrangian Density established, it is hard not
    to notice that the four-vector field $A_{\mu}$ required to make local invariance possible must have
    exactly the same gauge transformation property that the four-vector potential in Electrodynamics
    has, under its local $U(1)$ invariance. It is therefore natural to take the gauge field to be the
    elecromagnetic field. In a previous video, I showed you how to derive the Maxwell Lagrangian Density
    and its value came out to be:

    \begin{equation}
        \mathcal{L}_{Maxwell} = - \frac{1}{4 \mu_{0}} F_{\mu \nu} F^{\mu \nu} - A_{\mu} J^{\mu}
    \end{equation}

    Where $F_{\mu \nu} = \partial_{\mu} A_{\nu} - \partial_{\nu} A_{\mu}$. The first term is the kinetic
    term and the second is the interaction term.

    If one then takes the couplin between $\psi$ and $A_{\mu}$ to be the interaction term, and thn just adds
    the Maxwell kinetic term for the photon field, then one arrives at the following $U(1)$ gauge invariant
    Lagrangian Density:

    \begin{equation}
        \mathcal{L}_{QED} = i \hbar c \bar{\psi} \gamma^{\mu} D_{\mu} \psi - m c^{2} \bar{\psi} \psi - \frac{1}{4 \mu_{0}} F_{\mu \nu} F^{\mu \nu}
    \end{equation}

    This Lagrangian Density, used as a quantum field theory for electromagnetism, holds the distinction of having
    passed every experimental test yet performed. It is therefore taken as the correct Lagrangian Density for
    Quantum Electrodynamics. It is by this proccess of upgrading te global $U(1)$ invariance of the Dirac Lagrangian
    Density to a local one with a gauge field that it is then taken to be the elecromagnetic field (as a result of its
    identical gauge transformations), that $\mathcal{L}_{QED}$ is usually derived. Then, of course, its verification
    is a proper action of QED is empirical, as is always ultimately the case in physics. The Lagrangian can also be
    written as:

    \begin{equation}
        \mathcal{L}_{QED} = i \hbar c \bar{\psi} \gamma^{\mu} D_{\mu} \psi - m c^{2} \bar{\psi} \psi - \frac{1}{4 \mu_{0}} F_{\mu \nu} F^{\mu \nu}
    \end{equation}

    This reveals that this prescription gives the following results:

    \begin{equation}
        \mathcal{L}_{Maxwell} = - \frac{1}{4 \mu_{0}} F_{\mu \nu} F^{\mu \nu} - e \bar{\psi} \gamma^{\mu \psi} A_{\mu}
    \end{equation}

    \begin{equation}
        \mathcal{L}_{Interaction} = - e \bar{\psi} \gamma^{\mu \psi} A_{\mu}
    \end{equation}

    Which means that electric charge and current densities are:

    \begin{equation}
        J^{\mu} = e \bar{\psi} \gamma^{\mu \psi}
    \end{equation}

\end{document}