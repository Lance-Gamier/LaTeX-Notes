\documentclass[a4]{article}

\usepackage{amsmath}
\usepackage{amssymb}
\usepackage{framed}
\usepackage{mathrsfs}
\usepackage{esint}
\usepackage[compat = 1.1.0]{tikz-feynman}
\usepackage{slashed}

\usepackage[left = 1cm,right = 1cm, top = 2cm]{geometry}

\begin{document}

    \title{Moller Scattering}
    \maketitle

    \section*{Introduction}

    Moller scattering is one of the coolest of the basic scattering processes in Quantum Electrodynamics. It just consists of two electrons scattering off of each other, however despite its simplicity,
    I think it's particularly cool because the tree level differential scattering cross section has kind of a pretty form to it, and it is also one of the first truly famous result that I ever derived
    in QFT. In this video, I will show you how to derive the tree level Moller differential scattering cross section.

    Of course, one could always expand on the tree level with loop diagrams, but that is significantly more complicated. It isn't usually done until long after students have computed the tree level result.
    The tree level result, in calculations like this, just reproduces the classical answer, despite the fact that we are using quantum field theory machinery to compute it. One only gets quantum corrections
    from perturbative quantum field theory if loops are included. The expansion in the number of loops is also an expansion in powers of planks' constant, and therefore naturally just leave us with the
    classical result.

    Let's now get to the actual calcuation. At the tree level, we have the following Feynman diagrams for Moller scattering in QED:

    \begin{center}
        \begin{tabular}{|c|c|}
            \hline
            \multicolumn{2}{|c|}{$\langle e^{-}, e^{-} | S_{EM} | e^{-}, e^{-} \rangle$} \\
            \hline
            \begin{tikzpicture}
                    \begin{feynman}
                        \vertex [label = below: $x_1$] (a);
                        \vertex [right = of a,label = above: $x_2$] (b);
                        \vertex [above left = of a, label = $e^{-}$] (c);
                        \vertex [below left = of a, label = $e^{-}$] (d);
                        \vertex [above right = of b, label = $e^{-}$] (e);
                        \vertex [below right = of b, label = $e^{-}$] (f);
        
                        \diagram{
                            (a) -- [boson] (b);
                            (a) -- [fermion] (c);
                            (d) -- [fermion] (a);
                            (e) -- [fermion] (b);
                            (b) -- [fermion] (f);
                        };
                    \end{feynman}
                \end{tikzpicture} & \begin{tikzpicture}
                \begin{feynman}
                    \vertex [label = below: $x_1$] (a);
                    \vertex [right = of a,label = above: $x_2$] (b);
                    \vertex [above left = of a, label = $e^{-}$] (c);
                    \vertex [below left = of a, label = $e^{-}$] (d);
                    \vertex [above right = of b, label = $e^{-}$] (e);
                    \vertex [below right = of b, label = $e^{-}$] (f);
    
                    \diagram{
                        (a) -- [boson] (b);
                        (e) -- [fermion] (a);
                        (d) -- [fermion] (a);
                        (c) -- [fermion] (b);
                        (b) -- [fermion] (f);
                    };
                \end{feynman}
            \end{tikzpicture} \\
            \hline
        \end{tabular} \\
    \end{center}

    As usual, solid lines are fermion lines, and wavy lines are photon lines. In this case, all of the thermion lines are electron lines as shown by the foward-in-time arrows.

    Including only the ones t hat are relevant at the tree level, the QED Feynman rules are:

    \begin{center}
        \begin{tabular}[center]{|c|c|}
            \hline
            Incoming electron & $U_{e}$ \\
            \hline
            Outgoing electron & $\overline{U}_{e}$ \\
            \hline
            Incoming positron & $V_{e}$ \\
            \hline
            Outgoing positron & $\overline{V}_{e}$ \\
            \hline
            Incoming photon & $\epsilon_{1 \mu} (first polarization)$ \\
            \hline
            Outgoing photon & $\epsilon_{2 \mu} (second polarization)$ \\
            \hline
            Vertex & $-i e \gamma^{\mu}$ \\
            \hline
            Internal fermion & $i S_{F} (p) = \frac{i}{\slashed{p} - m + i (\epsilon = 0)} = \frac{\slashed{p} + m}{p^2 - m^2}$ \\
            \hline
            Internal photon & $i D^{F}_{\mu \nu} (p) = - \frac{i g_{\mu \nu}}{q^2 + i(\epsilon = 0)}$ \\
            \hline
        \end{tabular}
    \end{center}

    There is a link in the description to a video where I show how to derive the complete set Feynman rules for QED. Beyond these pictorial Feynman rules, there is one other important one that
    we must remember for this case, and that is that there is a relative minus sign generated by an interchange of fermion lines. This means that when we write out the Feynman amplitude terms 
    corresponding  to the two diagrams just given, we must make sure that they have a relative minus sign. 

    The calculation of the Moller scattering cross section starts witht the general formula for the differential cross section:

    \begin{center}
        \boxed{d \sigma = m_1 m_2 \frac{(2 \pi)^4 |M_{fi}|^2 \delta^4 (P_f - P_i)}{[(p_1 \cdot p_2)^2 - m_1^2 m_2^2]^{1/2}} \prod_{n = 1}^{N_f} \frac{m_n d^3 \vec{p}_n}{(2 \pi)^3 2 E_n}}
    \end{center}

    There is also a link in the description for a video whre I show how to derive this general formula.

    This general calculation will have several distinct stages. First, we will simplify the general differential scattering cross section formula as we can without knowing the Feynman amplitude.
    Second, we will use the Feynman rules to write out the Feynman amplitude. Third, we will take the absolute square of the Feynman amplitude and then spend a lot of time simplifying it. Fourth,
    we will construct a parametrization for the momentum four-vectors, and insert it into the squared Feynman amplitude, and then write the cross section. Lastly, we will take the ultra-relativistic,
    and low energy limits of the Bhabha scattering cross section formula.

    The center of mass reference frame will be assumed for this for this calculation. This fact is used to simplify things throughout the calculation. This fact is used to simplify things throughout
    the calculation. Towards the end, a specific parametrization based on this reference frame is adopted and used to obtain the final answer. This is the parametrization that I mentioned in the last
    paragraph.

    \section*{Preparation Of the Scattering Cross Section Formula}

    The first thing we will do in this section is insert the momentum variables written in the Feynman diagrams into the differential scattering cross section.We can see from the diagrams that we only
    have two outgoing particles so the immediate result of this will be simpler than the starting point. Beyond this, the standard result also includes averaging over the incoming fermion spins and
    summing over the outgoing fermion spins, wo we need to insert those averages/sums on the absolute squared Feynman amplitude into the differential scattering cross section. Doing all this to the
    formula given in the introduction easily gives us:

    \begin{equation}
        d^3 \sigma = \frac{m_1^2 m_2^2}{[(p_1 \cdot p_2) - m^4]^{1/2}} \frac{d^3 \vec{k}_1}{(2 \pi)^2 E_{k_1}} \frac{d^3 \vec{k}_1}{(2 \pi)^2 E_{k_2}} (\frac{1}{2})^2 \sum_{S_i S_f} |M_{fi}|^2 \delta^4 (p_1 + p_2 - p'_1 - p'_2)
    \end{equation}

    You may notice that I added a 6 superscript to the differential on $\sigma$. This is to indicate how many differentials there are on the other side of the equation. This number will drop as we complete
    phase space integrations over some of the momentum and energy variables.
    
    The standard Moller differential scattering cross section is with respect to the solid angle of one of the outgoing electrons. Therefore one of the things we will need to do in preparing the differential
    scattering cross section is integrate over the other momentum variables. It turns out that this can be done before we can work out the Feynman Amplitude, because of the delta functions that show up in
    the cross section formula. They are there to impose energy and momentum conservation. Therefore, we can do the necessary phase space integration without knowing the Feynman amplitude, as long ase we take
    energy and momentum conservation relations to be true for the Feynman amplitude, which depends on those variables. It turns out that these relations are extremely useful in simplifying the absolute square
    of the Feynman amplitude. 

    I chose to integrate over $\vec{p}_2$ to begin with. This is the first step in getting the differential cross section with respect to the solid angle of the $\vec{p}_1$ electron (this is an arbitrary choice,
    it doesn't matter which one we choose). Because of the momentum conservation delta functions, integration simply yields the following:

    \begin{equation}
        d^3 \sigma = \frac{m_1^2 m_2^2}{(2 \pi)^2 [(p_1 \cdot p_2) - m_1^2 m_2^2]^{1/2}} \frac{d^3 \vec{p'}_1}{E'_1 E'_2} (\frac{1}{2})^2 \sum_{S_i S_f} |M_{fi}|^2 \delta^4 (E_{1} + E_{2} - E'_{1} - E'_{2})
    \end{equation}

    Where the delta function has enforced the following mechanical conservation relation:

    \begin{equation}
        \vec{p'}_1 + \vec{p'}_2 = \vec{p}_1 + \vec{p}_2
    \end{equation}

    The next step is to put the remaining differential in spherical coordinates. This is necessary because it explicitly reveals the solid angle differential that we want to write the differential scattering
    cross section with respect to. Doing this gives:

    \begin{equation}
        d^3 \sigma = \frac{m_1^2 m_2^2}{(2 \pi)^2 [(p_1 \cdot p_2) - m_1^2 m_2^2]^{1/2}} \frac{|\vec{p'}_1|^2 d |\vec{p'}_1| d \Omega}{E'_1 E'_2} (\frac{1}{2})^2 \sum_{S_i S_f} |M_{fi}|^2 \delta^4 (E_{1} + E_{2} - E'_{1} - E'_{2})
    \end{equation}

    Therefore, the final integration that we must perform to get the differential scattering cross section with respect to the solid angle is the following:

    \begin{equation}
        d^2 \sigma = \frac{m_1^2 m_2^2 d \Omega}{(2 \pi)^2 [(p_1 \cdot p_2) - m_1^2 m_2^2]^{1/2}} \int \frac{|\vec{p'}_1|^2 d |\vec{p'}_1|}{E'_1 E'_2} (\frac{1}{2})^2 \sum_{S_i S_f} |M_{fi}|^2 \delta^4 (E_{1} + E_{2} - E'_{1} - E'_{2})
    \end{equation}

    where

    \begin{eqnarray}
        E'_1 = \sqrt{|p'_1|^2 + m_1^2} \\
        E'_2 = \sqrt{|p'_2|^2 + m_1^2} = \sqrt{|p'_1|^2 + m_1^2}
    \end{eqnarray}

    We can use the standard identity to rewrite the energy conservation delta function in an easy to integrate form. Specifically, we have the following: 

    \begin{equation}
        \delta (E_1 + E_2 - E'_1 - E'_2) = \delta (E_1 + E_2 - |\vec{p'}_1| - E (|\vec{p'}_2|) ) = \delta [f (|\vec{p}_1|)] = \frac{\delta [|\vec{p'}_1| - |\vec{p'}_1|_0]}{f' (|\vec{p'}_1|s)}
    \end{equation}

    Where $f'(|\vec{k}_1|)$ is the derivative of the f-function, and $|\vec{k}_1|$ is the root of the f-function, or the actual value of $|\vec{p'}_1|$. Because of the delta function, the integration simply forces:

    \begin{equation}
        |\vec{p'}_1| = |\vec{p'}_1|_0
    \end{equation}

    And through that, it enforces the following energy conservation relation:

    \begin{equation}
        E_1 + E_2 = E'_1 + E'_2
    \end{equation}

    We can then relabel the actual momentum $|\vec{p'}_1|_0$ with the symbol previously just used for the integration variable, to make things simpler. The integration over the magnitude of the momentum gives: 

    \begin{equation}
        d^2 \sigma = \frac{m_1^2 m_2^2 d \Omega}{(2 \pi)^2 [(p_1 \cdot p_2) - m_1^2 m_2^2]^{1/2}} (\frac{1}{2})^2 \sum_{S_i S_f} |M_{fi}|^2 \frac{|\vec{p}_1|^2}{E'_1 E'_2} \frac{1}{f' (|\vec{p}_1|s)}
    \end{equation}

    Where $f' (|\vec{p}_1|)$ works out to be:

    \begin{equation}
        f' (|\vec{k}_1|) = \frac{\vec{k}_1 \cdot \vec{k}_2}{E_{k_1} E_{k_2s}}
    \end{equation}

    It is worth pointing out that the superscript on the differential is uaually dropped. It wa suseful to keep track of things while we were doing the integration, but it is no longer needed, we we will drop it. 
    We therefore will write: 

    \begin{equation}
        d \sigma = \frac{m_1^2 m_2^2 d \Omega}{(2 \pi)^2 [(p_1 \cdot p_2) - m_1^2 m_2^2]^{1/2}} (\frac{1}{2})^2 \sum_{S_i S_f} |M_{fi}|^2 \frac{|\vec{p}_1|^2}{E'_1 E'_2} \frac{1}{f' (|\vec{p}_1|s)}
    \end{equation}

    Keep in mind that because of the integrations that we did $d \sigma$ doesn't quite mean what it did in the general formula in the introduction before any  integration had been done.

    If we insert the value of $f' (|\vec{p}_1|)$, the differential cross section becomes:

    \begin{equation}
        d \sigma = \frac{m_1^2 m_2^2 d \Omega}{(2 \pi)^2 [(p_1 \cdot p_2) - m_1^2 m_2^2]^{1/2}} (\frac{1}{2})^2 \sum_{S_i S_f} |M_{fi}|^2 \frac{|\vec{p}_1|^2}{E'_1 E'_2} \frac{E'_1 E'_2}{E'_1 + E'_2}
    \end{equation}

    \begin{equation}
        d \sigma = \frac{m_1^2 m_2^2 |\vec{p}_1|^2}{(2 \pi)^2 [(p_1 \cdot p_2) - m_1^2 m_2^2]^{1/2}} (\frac{1}{2})^2 \sum_{S_i S_f} |M_{fi}|^2 \frac{1}{(E'_1 + E'_2)^2}
    \end{equation}

    By applying energy conservation that was mandated by the last delta function, we can replace $E'_1 + E'_2$ with $E_1 + E_2$. This gives:

    \begin{equation}
        d \sigma = \frac{m_1^2 m_2^2 |\vec{p}_1|^2}{(2 \pi)^2 (E'_1 + E'_2)^2 [(p_1 \cdot p_2) - m_1^2 m_2^2]^{1/2}} (\frac{1}{2})^2 \sum_{S_i S_f} |M_{fi}|^2
    \end{equation}

    Because we are assuming the center of mass frame, we can write:

    \begin{equation}
        [(p_1 \cdot p_2)^2 - m_1^2 m_2^2]^\frac{1}{2} = |\vec{p}_1| (E_1 + E_2)
    \end{equation}

    Therefore, the differential scattering cross section becomes:

    \begin{center}
        \boxed{\frac{d \sigma_{CM}}{d \Omega} = \frac{m_1^2 m_2^2}{(2 \pi)^2 (E'_1 + E'_2)^2} \frac{|\vec{p'}_1|^2}{|\vec{p}_1|^2} (\frac{1}{2})^2 \sum_{S_i S_f} |M_{fi}|^2}
    \end{center}

    This completes the pre-simplification of the differential scattering cross section. Now we must begin the process of computing the Feynman amplitude.

    \section*{The Feynman Amplitude}

    The next step is to use Feynman diagrams and rules given in the introduction to write down the Feynman amplitude. The feynman amplitude has two terms in it. Two given Feynman diagrams contribute to the
    Tree Level, so we have two terms in the Feynman amplitude:

    \begin{equation}
        M_{fi} = M_{if}^1 + M_{if}^2
    \end{equation}

    The first diagram yields

    \begin{equation}
        M_{fi}^1 = i e \overline{U}_e (p'_2, s'_2) \gamma^{\mu} U_e (p_2, s_2) \frac{i g^{\mu \nu}}{(p'_1 - p_1)^2} \times - i e \overline{U}_e (p'_1, s'_1) \gamma_{\mu} U_e (p_1, s_1)
    \end{equation}

    \begin{equation}
    = i e^2 \overline{U}_e (p'_2, s'_2) \gamma^{\mu} U_e (p_2, s_2) \frac{i g^{\mu \nu}}{(p'_1 - p_1)^2} \overline{U}_e (p'_1, s'_1) \gamma_{\mu} U_e (p_1, s_1)
    \end{equation}

    For the second of the two diagrams (there is a relative minus sign due to the interchange of external fermion legs) one gets:

    \begin{equation}
        M_{fi}^2 = i e \overline{U}_e (p'_1, s'_1) \gamma^{\mu} U_e (p_2, s_2) \frac{i g^{\mu \nu}}{(p'_1 - p_1)^2} \times - i e \overline{U}_e (p'_2, s'_2) \gamma_{\mu} U_e (p_1, s_1)
    \end{equation}

    \begin{equation}
    = - i e^2 \overline{U}_e (p'_1, s'_1) \gamma^{\mu} U_e (p_2, s_2) \frac{i g^{\mu \nu}}{(p'_1 - p_1)^2} \overline{U}_e (p'_2, s'_2) \gamma_{\mu} U_e (p_1, s_1)
    \end{equation}

    The total Feynman amplitude therefore is:

    \begin{eqnarray}
        M_{fi} = M_{if}^1 + M_{if}^2 \\
        = i e^2 \overline{U}_e (p'_2, s'_2) \gamma^{\mu} U_e (p_2, s_2) \frac{1}{(p'_1 - p_1)^2} \overline{U}_e (p'_1, s'_1) \gamma_{\mu} U_e (p_1, s_1) \\
        - i e^2 \overline{U}_e (p'_1, s'_1) \gamma^{\mu} U_e (p_2, s_2) \frac{1}{(p'_1 - p_1)^2} \overline{U}_e (p'_2, s'_2) \gamma_{\mu} U_e (p_1, s_1)
    \end{eqnarray}

    In this particular calculation, there isn't really any more simplification to be done before the Feynman amplitude is squared, so that is what we will now do.

    \section*{Squaring The Feynman Amplitude}

    Of course, taking the absolute square of a quantity entails multiplying it by its complex conjugate. This raises a slight complication here, because that means that we must complex conjugate a complicated product of matrices. Luckily there is an easy identity for that. The identity
    consists of complex conujugating the prefactor, flipping the spinors, and reversing the order of the sandwiched matrices. The Feynman amplitude terms and their complex conjugates are given below:

    \begin{equation}
        M_{fi} = i e^2 \overline{U}_e (p'_2, s'_2) \gamma^{\mu} U_e (p_2, s_2) \frac{1}{(p'_1 - p_1)^2} \overline{U}_e (p'_1, s'_1) \gamma_{\mu} U_e (p_1, s_1)
    \end{equation}

    \begin{equation}
        M_{fi}^* = - i e^2 \overline{U}_e (p'_1, s'_1) \gamma^{\rho} U_e (p_2, s_2) \frac{1}{(p'_1 - p_1)^2} \overline{U}_e (p'_2, s'_2) \gamma_{\rho} U_e (p_1, s_1)
    \end{equation}

    \begin{equation}
        M_{fi} = - i e^2 \overline{U}_e (p'_1, s'_1) \gamma^{\mu} U_e (p_2, s_2) \frac{1}{(p'_1 - p_1)^2} \overline{U}_e (p'_2, s'_2) \gamma_{\mu} U_e (p_1, s_1)
    \end{equation}

    \begin{equation}
        M_{fi}^* = i e^2 \overline{U}_e (p'_2, s'_2) \gamma^{\rho} U_e (p_2, s_2) \frac{1}{(p'_1 - p_1)^2} \overline{U}_e (p'_1, s'_1) \gamma_{\rho} U_e (p_1, s_1)
    \end{equation}

    We can now insert these into the square:

    \begin{equation}
        (\frac{1}{2})^2 \sum_{S_f S_i} |M_{f i}|^2 = (\frac{1}{2})^2 \sum_{S_f S_i} [|M_{f i}^1|^2+M_{fi}^2 M_{fi}^{1*} + M_{fi}^1 M_{fi}^{2*} + |M_{f i}^2|^2]
    \end{equation}

    It is worth beginning the simplification of this square by handling each term separately.

    \begin{equation}
        (\frac{1}{2})^2 \sum_{S_f S_i} |M_{f i}|^2 = (\frac{1}{2})^2 \sum_{S_f S_i} M_{fi} M*_{fi}
    \end{equation}

    \begin{eqnarray}
        = \sum_{S_f S_i} i e^2 \overline{U}_e (p'_2, s'_2) \gamma^{\mu} U_e (p_2, s_2) \frac{1}{(p'_1 - p_1)^2} \overline{U}_e (p'_1, s'_1) \gamma_{\mu} U_e (p_1, s_1) \\
        \times - i e^2 \overline{U}_e (p'_1, s'_1) \gamma^{\rho} U_e (p_2, s_2) \frac{1}{(p'_1 - p_1)^2} \overline{U}_e (p'_2, s'_2) \gamma_{\rho} U_e (p_1, s_1)
    \end{eqnarray}

    Simplifying some gives:

    \begin{eqnarray}
        = \frac{e^4}{4 (p'_1 - p_1)^2} \sum_{S_f S_i} \overline{U}_e (p'_2, s'_2) \gamma^{\mu} U_e (p_2, s_2) \overline{U}_e (p'_1, s'_1) \gamma_{\mu} U_e (p_1, s_1) \\
        \times \overline{U}_e (p'_1, s'_1) \gamma^{\rho} U_e (p_2, s_2)  \overline{U}_e (p'_2, s'_2) \gamma_{\rho} U_e (p_1, s_1)
    \end{eqnarray}

    The order of the four scalar factors can then be optimized for the next step:

    \begin{eqnarray}
        = \frac{e^4}{4 (p'_1 - p_1)^2} \sum_{S_f S_i} U_e (p_1, s_1) \gamma_{\rho} \overline{U}_e (p'_1, s'_1) \overline{U}_e (p'_1, s'_1) \gamma_{\mu} U_e (p_1, s_1) \\
        \times Tr \overline{U}_e (p'_2, s'_2) \gamma^{\rho} U_e (p_2, s_2) \overline{U}_e (p'_2, s'_2) \gamma^{\mu} U_e (p_2, s_2)
    \end{eqnarray}

    One can then introduce traces of specific paris of scalar factors, because taking the trace of a scalar changes nothing:

    \begin{eqnarray}
        = \frac{e^4}{4 (p'_1 - p_1)^2} \sum_{S_f S_i} Tr [U_e (p_1, s_1) \gamma_{\rho} \overline{U}_e (p'_1, s'_1) \overline{U}_e (p'_1, s'_1) \gamma_{\mu} U_e (p_1, s_1)] \\
        \times Tr [\overline{U}_e (p'_2, s'_2) \gamma^{\rho} U_e (p_2, s_2) \overline{U}_e (p'_2, s'_2) \gamma^{\mu} U_e (p_2, s_2)]
    \end{eqnarray}

    The Cyclic property of the trace can then be used to rearranged factors under the trace to yield:

    \begin{eqnarray}
        = \frac{e^4}{4 (p'_1 - p_1)^2} \sum_{S_f S_i} Tr [\gamma_{\mu} U_e (p_1, s_1) \overline{U}_e (p'_1, s'_1) \gamma_{\rho} \overline{U}_e (p'_1, s'_1) U_e (p_1, s_1)] \\
        \times Tr [\gamma^{\mu} \overline{U}_e (p'_2, s'_2) U_e (p_2, s_2) \gamma^{\rho} \overline{U}_e (p'_2, s'_2) U_e (p_2, s_2)]
    \end{eqnarray}

    We are now ready to apply the following identity:

    \begin{equation}
        \sum_{\pm s} U (p, s) \overline{U} (p, s) = \frac{\slashed{p} + m}{2m} \qquad \sum_{\pm s} V (p, s) \overline{V} (p, s) = \frac{\slashed{p} - m}{2m}
    \end{equation}

    Applying it gives the final result for this term for this section:

    \begin{center}
        \boxed{(\frac{1}{2})^2 \sum_{S_f S_i} |M_{f i}|^2 = Tr \Bigg[ \gamma_{\mu} \frac{\slashed{p} + m}{2m} \gamma_{\rho} \frac{\slashed{p} + m}{2m} \Bigg] Tr \Bigg[ \gamma^{\mu} \frac{\slashed{p} + m}{2m} \gamma^{\rho} \frac{\slashed{p} + m}{2m} \Bigg]}
    \end{center}

    Given that the only difference between the two Feynman amplitude terms is a flipped sign and flipped outgoing momentum variables, we can derive a similar expression for the fourth term in the square by simply making those changes
    to the above boxed result (the flip sign cancels). Doing this gives the following:

    \begin{center}
        \boxed{(\frac{1}{2})^2 \sum_{S_f S_i} |M_{f i}|^2 = Tr \Bigg[ \gamma_{\mu} \frac{\slashed{p} + m}{2m} \gamma_{\rho} \frac{\slashed{p} + m}{2m} \Bigg] Tr \Bigg[ \gamma^{\mu} \frac{\slashed{p} + m}{2m} \gamma^{\rho} \frac{\slashed{p} + m}{2m} \Bigg]}
    \end{center}

    Next, the same process can be performed on the first of the cross terms in the square:

    \begin{equation}
        (\frac{1}{2})^2 \sum_{S_f S_i} |M_{f i}|^2 = (\frac{1}{2})^2 \sum_{S_f S_i} M_{fi} M*_{fi}
    \end{equation}

    Simplifying a little:

    \begin{equation}
        M_{fi}^1 = i e \overline{U}_e (p'_2, s'_2) \gamma^{\mu} U_e (p_2, s_2) \frac{i g^{\mu \nu}}{(p'_1 - p_1)^2} \times - i e \overline{U}_e (p'_1, s'_1) \gamma_{\mu} U_e (p_1, s_1)
    \end{equation}

    \begin{eqnarray}
        = \sum_{S_f S_i} i e^2 \overline{U}_e (p'_2, s'_2) \gamma^{\mu} U_e (p_2, s_2) \frac{1}{(p'_1 - p_1)^2} \overline{U}_e (p'_1, s'_1) \gamma_{\mu} U_e (p_1, s_1) \\
        \times - i e^2 \overline{U}_e (p'_1, s'_1) \gamma^{\rho} U_e (p_2, s_2) \frac{1}{(p'_1 - p_1)^2} \overline{U}_e (p'_2, s'_2) \gamma_{\rho} U_e (p_1, s_1)
    \end{eqnarray}

    The order of the four scalar factors can then be optimized for the last step:

    \begin{eqnarray}
        = \frac{e^4}{4 (p'_1 - p_1)^2} \sum_{S_f S_i} U_e (p_1, s_1) \gamma_{\rho} \overline{U}_e (p'_1, s'_1) \overline{U}_e (p'_1, s'_1) \gamma_{\mu} U_e (p_1, s_1) \\
        \times Tr \overline{U}_e (p'_2, s'_2) \gamma^{\rho} U_e (p_2, s_2) \overline{U}_e (p'_2, s'_2) \gamma^{\mu} U_e (p_2, s_2)
    \end{eqnarray}

    One can then take the trace of the scalar quantity under the sum because taking the trace of a scalar changes nothing:

    \begin{eqnarray}
        = \frac{e^4}{4 (p'_1 - p_1)^2} \sum_{S_f S_i} Tr [U_e (p_1, s_1) \gamma_{\rho} \overline{U}_e (p'_1, s'_1) \overline{U}_e (p'_1, s'_1) \gamma_{\mu} U_e (, s_1)] \\
        \times Tr [\overline{U}_e (p'_2, s'_2) \gamma^{\rho} U_e (p_2, s_2) \overline{U}_e (p'_2, s'_2) \gamma^{\mu} U_e (p_2, s_2)]
    \end{eqnarray}

    One can then use the cyclic property of the trace to prepare this for the final step:

    \begin{eqnarray}p_1
        = \frac{e^4}{4 (p'_1 - p_1)^2} \sum_{S_f S_i} Tr [\gamma_{\mu} U_e (p_1, s_1) \overline{U}_e (p'_1, s'_1) \gamma_{\rho} \overline{U}_e (p'_1, s'_1) U_e (p_1, s_1)] \\
        \times Tr [\gamma^{\mu} \overline{U}_e (p'_2, s'_2) U_e (p_2, s_2) \gamma^{\rho} \overline{U}_e (p'_2, s'_2) U_e (p_2, s_2)]
    \end{eqnarray}

    The final step consists of applying the same identity as before:

    \begin{equation}
        \sum_{\pm s} U (p, s) \overline{U} (p, s) = \frac{\slashed{p} + m}{2m}
    \end{equation}

    Applying it gives the final answer for this trace for this section:

    \begin{center}
        \boxed{(\frac{1}{2})^2 \sum_{S_f S_i} |M_{f i}|^2 = Tr \Bigg[ \gamma_{\mu} \frac{\slashed{p} + m}{2m} \gamma_{\rho} \frac{\slashed{p} + m}{2m} \Bigg] Tr \Bigg[ \gamma^{\mu} \frac{\slashed{p} + m}{2m} \gamma^{\rho} \frac{\slashed{p} + m}{2m} \Bigg]}
    \end{center}

    Performing the same interchange that got the four the term from the frist, provides us with the term term from the second:

    \begin{center}
        \boxed{(\frac{1}{2})^2 \sum_{S_f S_i} |M_{f i}|^2 = Tr \Bigg[ \gamma_{\mu} \frac{\slashed{p} + m}{2m} \gamma_{\rho} \frac{\slashed{p} + m}{2m} \Bigg] Tr \Bigg[ \gamma^{\mu} \frac{\slashed{p} + m}{2m} \gamma^{\rho} \frac{\slashed{p} + m}{2m} \Bigg]}
    \end{center}

    Inserting these results into the square of the Feynman amplitude produces the following result:

    \begin{equation}
        (\frac{1}{2})^2 \sum_{S_f S_i} |M_{f i}|^2 = (\frac{1}{2})^2 \sum_{S_f S_i} M_{fi} M*_{fi}
    \end{equation}

    \begin{equation}
        (\frac{1}{2})^2 \sum_{S_f S_i} |M_{f i}|^2 = (\frac{1}{2})^2 \sum_{S_f S_i} [|M_{f i}^1|^2+M_{fi}^2 M_{fi}^{1*} + M_{fi}^1 M_{fi}^{2*} + |M_{f i}^2|^2]
    \end{equation}

    This can be written more simply as follows:

    \begin{center}
        \boxed{(\frac{1}{2})^2 \sum_{S_f S_i} |M_{f i}|^2 = Tr \Bigg[ \gamma_{\mu} \frac{\slashed{p} + m}{2m} \gamma_{\rho} \frac{\slashed{p} + m}{2m} \Bigg] Tr \Bigg[ \gamma^{\mu} \frac{\slashed{p} + m}{2m} \gamma^{\rho} \frac{\slashed{p} + m}{2m} \Bigg]}
    \end{center}

    To simplify this squared Feynman amplitude down further requires the traces to be evaluated. This is the subject of next section.

    \section*{Evaluating The Traces}

    The first trace can be broken up the following way:

    \begin{equation}
        Tr [\gamma^\mu \frac{\slashed{p'}_1 + m}{2m} \gamma^\nu \frac{\slashed{p}_1 + m}{2 m}] = \frac{1}{(2m)^2} Tr[\gamma_\mu (\slashed{p'}_1 + m) \gamma_\rho (\slashed{p}_1 + m)]
    \end{equation}

    \begin{equation}
        \frac{1}{(2m)^2} Tr[\gamma_\mu \slashed{p'}_1 \gamma_\rho \slashed{p}_1 + m \gamma_\mu \gamma_\rho \slashed{p}_1 + m \gamma_\mu \slashed{p'}_1 \gamma_\rho + m^2 \gamma_\mu \gamma_\rho]
    \end{equation}

    \begin{equation}
        \frac{1}{(2m)^2} \big[ Tr[\gamma_\mu \slashed{p'}_1 \gamma_\rho \slashed{p}_1] + m Tr[\gamma_\mu \gamma_\rho \slashed{p}_1] + m Tr[\gamma_\mu \slashed{p'}_1 \gamma_\rho] + m^2 Tr[\gamma_\mu \gamma_\rho] \big]
    \end{equation}

    \begin{equation}
        \frac{1}{(2m)^2} \big[ Tr[\gamma_\mu \slashed{p'}_1 \gamma_\rho \slashed{p}_1] + m^2 Tr[\gamma_\mu \gamma_\rho] \big] = \frac{1}{(2m)^2} \big[ T_1 + m^2 T_2 \big]
    \end{equation}

    The two traces that show up in the break-down of the initial trace can be evaluated using two common gamma matrix identities:

    \begin{eqnarray}
        T_1 = \frac{1}{(2m)^2} = p_1^{\alpha} p_2^{'\beta} Tr [\gamma_\mu \gamma_\rho \gamma_\alpha \gamma_\beta] = 4 p_1^{\alpha} p_2^{'\beta} (g_{\mu \alpha} g_{\rho \beta} - g_{\mu \rho} g_{\alpha \beta} + g_{\mu \beta} g_{\alpha \rho}) \\
        = 4 (p_{1 \mu} p'_{1 \rho} - 4 p_1 \cdot p'_1 g_{\mu \nu} + p_{1 \rho} p'_{1 \mu}) \\ 
        T_2 = Tr [\gamma_\mu \gamma_\rho] = 4 g_{\mu \rho}
    \end{eqnarray}

    Inserting these results for the contributing traces produces the following result for the original trace that we actually wanted:

    \begin{center}
        \boxed{\Bigg[ \gamma^{\mu} \frac{\slashed{p} + m}{2m} \gamma^{\rho} \frac{\slashed{p} + m}{2m} \Bigg] = \mathbf{\frac{1}{m^2} [p_{1 \mu} p'_{1 \rho} - 4 p_1 \cdot p'_1 g_{\mu \nu} + p_{1 \rho} p'_{1 \mu} + m^2 g_{\mu \rho}]} }
    \end{center}

    The only difference between the trace we just calculated and the second one that shows up in the Feynman amplitude square is the raising of indices, and the subscripts being 2 instead of 1. Because of this, the result for the second trace can be
    simply written down from the result we just got:

    \begin{center}
        \boxed{\Bigg[ \gamma^{\mu} \frac{\slashed{p} + m}{2m} \gamma^{\rho} \frac{\slashed{p} + m}{2m} \Bigg] = mathbf{\frac{1}{m^2} [p_{2}^{\mu} p_{2}^{' \rho} - 4 p_2 \cdot p'_2 g^{\mu \nu} + p_{2}^{\rho} p_{2}^{' \mu} + m^2 g^{\mu \rho}]} }
    \end{center}

    We must then contract these two results to get the complete first term numerator in the Feynman amplitude square Doing this is pretty trivial: 

    \begin{equation}
        Tr \Bigg[ \gamma_{\mu} \frac{\slashed{p} + m}{2m} \gamma_{\rho} \frac{\slashed{p} + m}{2m} \Bigg] Tr \Bigg[ \gamma^{\mu} \frac{\slashed{p} + m}{2m} \gamma^{\rho} \frac{\slashed{p} + m}{2m} \Bigg]
    \end{equation}

    \begin{equation}
        = \frac{1}{m^4} [p_{1 \mu} p'_{1 \rho} - 4 p_1 \cdot p'_1 g_{\mu \nu} + p_{1 \rho} p'_{1 \mu} + m^2 g_{\mu \rho}] [p_{2}^{\mu} p_{2}^{' \rho} - 4 p_2 \cdot p'_2 g^{\mu \nu} + p_{2}^{\rho} p_{2}^{' \mu} + m^2 g^{\mu \rho}]
    \end{equation}

    \begin{eqnarray}
        = \frac{1}{m^4} [p_1 \cdot p_2 p'_1 \cdot p'_2 - p_1 \cdot p'_1 p_2 \cdot p'_2 + p_1 \cdot p'_2 p'_1 \cdot p_2 + m^2 p_1 \cdot p'_1 + p_1 \cdot p'_1 p_2 \cdot p'_2 + 4 p_1 \cdot p'_1 p_2 \cdot p'_2 - p_1 \\
        \cdot p'_1 p_2 \cdot p'_2 - 4 m^2 p_1 \cdot p'_1 + p_1 \cdot p'_2 p'_1 \cdot p_2 - p_1 \cdot p'_1 p_2 \cdot p'_2 + p_1 \cdot p'_2 p_1 \cdot p'_2 + m^2 p_1 \cdot p'_1 + m^2 p_2 \\
        \cdot p'_2 - 4 m^2 p_2 \cdot p'_2 + m^2 p_2 \cdot p'_2 + 4 m^2]
    \end{eqnarray}

    \begin{equation}
        = \frac{2}{m^2} [p_1 \cdot p_2 p'_1 \cdot p'_2 + p_1 \cdot p'_2 p'_1 \cdot p_2 - m^2 p_1 \cdot p'_1 - m^2 p_2 \cdot p'_2 + 2 m^4]
    \end{equation}

    The delta functions in the scattering cross section formula enforced energy and momentum conservation which, put together, yields four-momentum conservation implies the following relation:

    \begin{equation}
        p_1 + p_2 = p'_1 + p'_2
    \end{equation}

    We can use this four momentum conservation relation to simplify this contracted product of traces a little bit further. Specifically, four-momentum conservation implies the following relations:

    \begin{equation}
        p_1 \cdot p_2 = p'_1 \cdot p'_2 \qquad p_1 \cdot p'_1 = p'_2 \cdot p'_2 \qquad p_1 \cdot p'_2 = p'_1 \cdot p_2
    \end{equation}

    These relations can be derived from four-momentum conservation by squaring and simplifying for the first one, or rearranging, squaring and simplifying for the last two. Applying these to the contracted trace product allows it to be rewritten as follows:

    \begin{equation}
        Tr [\gamma^\mu \frac{\slashed{p'}_1 + m}{2m} \gamma^\nu \frac{\slashed{p}_1 + m}{2 m}] Tr [\gamma^\mu \frac{\slashed{p'}_1 + m}{2m} \gamma^\nu \frac{\slashed{p}_1 + m}{2 m}]
    \end{equation}

    \begin{equation}
        \frac{2}{m^4} [(p_1 \cdot p_2)^2 + (p_1 \cdot p'_2)^2 - 2 m^2 p_1 \cdot p'_1 + 2 m^4]
    \end{equation}

    The Final simplification of this contracted trace product can be achieved by applying the four momentum conservation relation direactly to the last two terms. Specifically, they can be rewritten as follows:

    \begin{equation}
        -2 m^2 p_1 \cdot p'_1 + 2 m^4 = 2 m^2 (m^2 - p_1 \cdot p'_1) = 2 m^2 p_1 \cdot (p_1 \cdot p'_1) = 2 m^2 (p_1 - p'_1) = 2 m^2 (p_1 \cdot p'_2 - p_1 \cdot p_2)
    \end{equation}

    Inserting this gives the final result for this contracted product of traces:

    \begin{framed}
        \begin{equation}
            \begin{aligned}
                Tr [\gamma_\mu \frac{\slashed{p'}_1 + m}{2m} \gamma_\rho \frac{\slashed{p}_1 + m}{2 m}] Tr [\gamma^\mu \frac{\slashed{p'}_1 + m}{2m} \gamma^\rho \frac{\slashed{p}_1 + m}{2 m}] \\
                = \frac{2}{m^4} [(p_1 \cdot p_2)^2 + (p_1 \cdot p'_2)^2 - 2 m^2 (p_1 \cdot p'_2 - p_1 \cdot p_2)]
            \end{aligned}
        \end{equation}
    \end{framed}

    One can then interchange the final momentum to get the last term in the Feynman amplitude square:

    \begin{framed}
        \begin{equation}
            \begin{aligned}
                Tr [\gamma_\mu \frac{\slashed{p'}_1 + m}{2m} \gamma_\rho \frac{\slashed{p}_1 + m}{2 m}] Tr [\gamma^\mu \frac{\slashed{p'}_1 + m}{2m} \gamma^\rho \frac{\slashed{p}_1 + m}{2 m}] \\
                = \frac{2}{m^4} [(p_1 \cdot p_2)^2 + (p_1 \cdot p'_2)^2 - 2 m^2 (p_1 \cdot p'_2 - p_1 \cdot p_2)]
            \end{aligned}
        \end{equation}
    \end{framed}

    Now we can handle the cross term trace numerators in the Feynman amplitude square. The first one preliminarily simplifies down as follows:

    \begin{equation}
        Tr [\gamma_\mu \frac{\slashed{p'}_1 + m}{2m} \gamma_\rho \frac{\slashed{p}_1 + m}{2 m} \gamma^\mu \frac{\slashed{p'}_1 + m}{2m} \gamma^\rho \frac{\slashed{p}_1 + m}{2 m}]
    \end{equation}

    \begin{equation}
        = \frac{1}{(2 m)^2} Tr [\gamma_\mu (\slashed{p'}_1 + m) \gamma_\rho (\slashed{p}_1 + m) \gamma^\mu (\slashed{p'}_2 + m) \gamma^\rho (\slashed{p}_2 + m)]
    \end{equation}

    \begin{equation}
        = \frac{1}{(2 m)^2} Tr [(\gamma_\mu \slashed{p}_1 \gamma_\rho \slashed{p'}_1 + m \gamma_\mu \gamma_\rho \slashed{p'}_1 + m \gamma_\mu \slashed{p}_1 \gamma_\rho + m^2 \gamma_\mu \gamma_\rho) (\gamma^\mu \slashed{p}_2 \gamma^\rho \slashed{p'}_2 + m \gamma^\mu \gamma^\rho \slashed{p'}_2 + m \gamma^\mu \slashed{p}_2 \gamma^\rho + m^2 \gamma^\mu \gamma^\rho)]
    \end{equation}

    \begin{eqnarray}
        = \frac{1}{(2 m)^2} Tr \Big[ \gamma_\mu \slashed{p}_1 \gamma_\rho \slashed{p'}_1 \gamma^\mu \slashed{p}_2 \gamma^\rho \slashed{p'}_2 + m \gamma_\mu \slashed{p}_1 \gamma_\rho \slashed{p'}_1 \gamma^\mu \gamma^\rho \slashed{p'}_2 + m \gamma_\mu \slashed{p}_1 \gamma_\rho \slashed{p'}_1 \gamma^\mu \slashed{p}_2 \gamma^\rho + m^2 \gamma_\mu \slashed{p}_1 \gamma_\rho \slashed{p'}_1 \gamma^\mu \gamma^\rho \\
        m \gamma_\mu \gamma_\rho \slashed{p'}_1 \gamma^\mu \slashed{p}_2 \gamma^\rho \slashed{p'}_2 + m^2 \gamma_\mu \gamma_\rho \slashed{p'}_1 \gamma^\mu \gamma^\rho \slashed{p'}_2 + m^2 \gamma_\mu \gamma_\rho \slashed{p'}_1 \gamma^\mu \slashed{p}_2 \gamma^\rho + m^2 \gamma_\mu \gamma_\rho \slashed{p'}_1 \gamma^\mu \gamma^\rho \\
        m \gamma_\mu \slashed{p}_1 \gamma_\rho \gamma^\mu \slashed{p}_2 \gamma^\rho \slashed{p'}_2 + m^2 \gamma_\mu \slashed{p}_1 \gamma_\rho \gamma^\mu \gamma^\rho \slashed{p'}_2 + m^2 \gamma_\mu \slashed{p}_1 \gamma_\rho \gamma^\mu \slashed{p}_2 \gamma^\rho + m^2 \gamma_\mu \slashed{p}_1 \gamma_\rho \gamma^\mu \gamma^\rho \\
        m^2 \gamma_\mu \gamma_\rho \gamma^\mu \slashed{p}_2 \gamma^\rho \slashed{p'}_2 + m^3 \gamma_\mu \gamma_\rho \gamma^\mu \gamma^\rho \slashed{p'}_2 + m^3 \gamma_\mu \gamma_\rho \gamma^\mu \slashed{p}_2 \gamma^\rho + m^4 \gamma_\mu \gamma_\rho \gamma^\mu \gamma^\rho \Big]
    \end{eqnarray}

    \begin{eqnarray}
        = \frac{1}{(2 m)^2} \Big[ Tr[ \gamma_\mu \slashed{p}_1 \gamma_\rho \slashed{p'}_1 \gamma^\mu \slashed{p}_2 \gamma^\rho \slashed{p'}_2] + m Tr[\gamma_\mu \slashed{p}_1 \gamma_\rho \slashed{p'}_1 \gamma^\mu \gamma^\rho \slashed{p'}_2] + m Tr[\gamma_\mu \slashed{p}_1 \gamma_\rho \slashed{p'}_1 \gamma^\mu \slashed{p}_2 \gamma^\rho] + m^2 Tr[\gamma_\mu \slashed{p}_1 \gamma_\rho \slashed{p'}_1 \gamma^\mu \gamma^\rho] \\
        m Tr[\gamma_\mu \gamma_\rho \slashed{p'}_1 \gamma^\mu \slashed{p}_2 \gamma^\rho \slashed{p'}_2] + m^2 Tr[\gamma_\mu \gamma_\rho \slashed{p'}_1 \gamma^\mu \gamma^\rho \slashed{p'}_2] + m^2 Tr[\gamma_\mu \gamma_\rho \slashed{p'}_1 \gamma^\mu \slashed{p}_2 \gamma^\rho] + m^2 Tr[\gamma_\mu \gamma_\rho \slashed{p'}_1 \gamma^\mu \gamma^\rho] \\
        m Tr[\gamma_\mu \slashed{p}_1 \gamma_\rho \gamma^\mu \slashed{p}_2 \gamma^\rho \slashed{p'}_2] + m^2 Tr[\gamma_\mu \slashed{p}_1 \gamma_\rho \gamma^\mu \gamma^\rho \slashed{p'}_2] + m^2 Tr[\gamma_\mu \slashed{p}_1 \gamma_\rho \gamma^\mu \slashed{p}_2 \gamma^\rho] + m^2 Tr[\gamma_\mu \slashed{p}_1 \gamma_\rho \gamma^\mu \gamma^\rho] \\
        m^2 Tr[\gamma_\mu \gamma_\rho \gamma^\mu \slashed{p}_2 \gamma^\rho \slashed{p'}_2] + m^3 Tr[\gamma_\mu \gamma_\rho \gamma^\mu \gamma^\rho \slashed{p'}_2] + m^3 Tr[\gamma_\mu \gamma_\rho \gamma^\mu \slashed{p}_2 \gamma^\rho] + m^4 Tr[\gamma_\mu \gamma_\rho \gamma^\mu \gamma^\rho] \Big]
    \end{eqnarray}

    Traces of products of odd numbers of gamma matrices are zero, so.

    \begin{eqnarray}
        = \frac{1}{(2 m)^2} \Big[ Tr[ \gamma_\mu \slashed{p}_1 \gamma_\rho \slashed{p'}_1 \gamma^\mu \slashed{p}_2 \gamma^\rho \slashed{p'}_2] + m^2 Tr[\gamma_\mu \slashed{p}_1 \gamma_\rho \slashed{p'}_1 \gamma^\mu \gamma^\rho] \\
        + m^2 Tr[\gamma_\mu \gamma_\rho \slashed{p'}_1 \gamma^\mu \gamma^\rho \slashed{p'}_2] + m^2 Tr[\gamma_\mu \gamma_\rho \slashed{p'}_1 \gamma^\mu \slashed{p}_2 \gamma^\rho] + m^2 Tr[\gamma_\mu \gamma_\rho \slashed{p'}_1 \gamma^\mu \gamma^\rho] \\
        + m^2 Tr[\gamma_\mu \slashed{p}_1 \gamma_\rho \gamma^\mu \gamma^\rho \slashed{p'}_2] + m^2 Tr[\gamma_\mu \slashed{p}_1 \gamma_\rho \gamma^\mu \slashed{p}_2 \gamma^\rho] + m^2 Tr[\gamma_\mu \slashed{p}_1 \gamma_\rho \gamma^\mu \gamma^\rho] \\
        m^2 Tr[\gamma_\mu \gamma_\rho \gamma^\mu \slashed{p}_2 \gamma^\rho \slashed{p'}_2] + m^4 Tr[\gamma_\mu \gamma_\rho \gamma^\mu \gamma^\rho] \Big]
    \end{eqnarray}

    This can be rewritten in terms of a sequence of t races that we will evaluate separately and then insert back in:

    \begin{equation}
        \frac{1}{(2 m)^4} [T_1 + m T_2 + m^2 T_3 + m T_4 + m^2 T_5 + m^2 T_6 + m^2 T_7 + m^4 T_8]
    \end{equation}

    Where

    \begin{eqnarray}
        T_1 = Tr[ \gamma_\mu \slashed{p}_1 \gamma_\rho \slashed{p'}_1 \gamma^\mu \slashed{p}_2 \gamma^\rho \slashed{p'}_2] & T_2 = Tr[\gamma_\mu \slashed{p}_1 \gamma_\rho \slashed{p'}_1 \gamma^\mu \gamma^\rho] \\
        T_3 = Tr[\gamma_\mu \gamma_\rho \slashed{p'}_1 \gamma^\mu \gamma^\rho \slashed{p'}_2] & T_4 = Tr[\gamma_\mu \gamma_\rho \slashed{p'}_1 \gamma^\mu \slashed{p}_2 \gamma^\rho] \\
        T_5 = Tr[\gamma_\mu \gamma_\rho \slashed{p'}_1 \gamma^\mu \gamma^\rho] & T_6 = Tr[\gamma_\mu \slashed{p}_1 \gamma_\rho \gamma^\mu \gamma^\rho \slashed{p'}_2] \\
        T_7 = Tr[\gamma_\mu \gamma_\rho \gamma^\mu \slashed{p}_2 \gamma^\rho \slashed{p'}_2] & T_8 = Tr[\gamma_\mu \gamma_\rho \gamma^\mu \gamma^\rho] \\
    \end{eqnarray}

    We can then simplify these down separately using the following list of gamma matrices:

    \begin{equation}
        Tr [\gamma^\mu \gamma^\nu] = 4 g^{\mu \nu} \qquad \gamma_\mu \gamma_\nu \gamma^\mu = -2 \gamma_\nu \qquad \gamma_\mu \gamma_\nu \gamma_\rho \gamma^\mu = 4 g_{\mu \nu} I \qquad \gamma_\mu \gamma_\nu \gamma_\rho \gamma_\sigma \gamma^\mu = -2 \gamma_\sigma \gamma_\rho \gamma_\nu \qquad Tr[\gamma_\mu \gamma_\nu] = 4 g_{\mu \nu}
    \end{equation}

    Applying these to the various traces allows for their simplificaiton as follows:

    \begin{eqnarray}
        T_1 Tr[ \gamma_\mu \slashed{p}_1 \gamma_\rho \slashed{p'}_1 \gamma^\mu \slashed{p}_2 \gamma^\rho \slashed{p'}_2] = p_1^\alpha p_1^{' \beta} p_2^\gamma p_2^{' \delta} Tr [\gamma_\mu \gamma_\alpha \gamma_\rho \gamma_\beta \gamma^\mu \gamma_\lambda \gamma^\rho \gamma_\delta] = -2 p_1^\alpha p_1^{' \beta} p_2^\gamma p_2^{' \gamma} Tr [\gamma_beta \gamma_\rho \gamma_\alpha \gamma_\gamma \gamma^\rho \gamma_\delta] \\
        = -8 g_{\alpha \gamma} p_1^\alpha p_1^{' \beta} p_2^\gamma p_2^{' \delta} Tr[\gamma_\beta \gamma_\delta] = -32 g_{\alpha \gamma} p_1^\alpha p_1^{' \beta} p_2^\gamma p_2^{' \gamma} g_{\beta \delta} = \mathbf{-32 p_1 \cdot p_2 p'_1 \cdot p'_2}
    \end{eqnarray}

    % Insert large table here

    \begin{equation}
        T_8 = Tr [\gamma_\mu \gamma_\rho \gamma^\mu \gamma^\rho] = - 2 Tr [\gamma_\mu \gamma^\nu] = - 8 Tr [I] = -32
    \end{equation}

    Substituting all of these simplified traces back in, gives the following:

    \begin{equation}
        Tr [\gamma_\mu \frac{\slashed{p'}_1 + m}{2m} \gamma_\rho \frac{\slashed{p}_1 + m}{2 m} \gamma^\mu \frac{\slashed{p'}_1 + m}{2m} \gamma^\rho \frac{\slashed{p}_1 + m}{2 m}]
    \end{equation}

    \begin{equation}
        \frac{1}{m^4} [-2 p_1 \cdot p_2 + p'_1 \cdot p'_2 + m^2 p_1 \cdot p'_1 + m^2 p'_1 \cdot p'_2 + m^2 p'_1 \cdot p_2 + m^2 p_1 \cdot p'_2 m^2 p_1 \cdot p_2 + m^2 p_2 \cdot p'_2 - 2 m^4]
    \end{equation}

    \begin{equation}
        \frac{1}{m^4} [-2 p_1 \cdot p_2 + m^2 p_1 \cdot p'_2 + m^2 (p_1 \cdot p'_1 + p'_1 \cdot p'_2 + p_1 \cdot p_2 + p_2 \cdot p'_2) + m^2 p'_1 \cdot p_2 - 2 m^4]
    \end{equation}

    We can then apply four momentum conservation to simplify this this quite a bit just like we did above for the contracted trace produce. The algebra proceeds as folows:

    \begin{equation}
        \frac{1}{m^4} [-2 (p_1 \cdot p_2)^2 + m^2 p_1 \cdot p'_2 + 2 m^2 (p_1 \cdot p'_1 + p_1 \cdot p_2 + p_2 \cdot p'_2) - 2 m^4]
    \end{equation}

    \begin{equation}
        \frac{1}{m^4} [-2 (p_1 \cdot p_2)^2 + 2 m^2 (p_1 \cdot (p'_1 \cdot p'_2) + 2 p_1 \cdot p_2 + p'_2 \cdot (p_1 \cdot p_2)) - 2 m^4]
    \end{equation}

    \begin{equation}
        \frac{1}{m^4} [-2 (p_1 \cdot p_2)^2 + 2 m^2 (p_1 \cdot (p_1 \cdot p_2) + 2 p_1 \cdot p_2 + p'_2 \cdot (p'_1 \cdot p'_2)) - 2 m^4]
    \end{equation}

    \begin{equation}
        \frac{1}{m^4} [-2 (p_1 \cdot p_2)^2 + m^2 (2 m^2 + 4 p_1 \cdot p_2) - 2 m^4]
    \end{equation}

    \begin{equation}
        \frac{1}{m^4} [-2 (p_1 \cdot p_2)^2 + 4 m^2 p_1 \cdot p_2]
    \end{equation}

    So the final answer is:

    \begin{framed}
        \begin{eqnarray}
            Tr [\gamma_\mu \frac{\slashed{p'}_1 + m}{2m} \gamma_\rho \frac{\slashed{p}_1 + m}{2 m} \gamma^\mu \frac{\slashed{p'}_1 + m}{2m} \gamma^\rho \frac{\slashed{p}_1 + m}{2 m}] \\
            = \mathbf{\frac{1}{m^4} [-2 (p_1 \cdot p_2)^2 + 4 m^2 p_1 \cdot p_2]}
        \end{eqnarray}
    \end{framed}

    We can then interchange the final momenta in the result we just got to get the final trace that we need:

    \begin{framed}
        \begin{eqnarray}
            Tr [\gamma_\mu \frac{\slashed{p'}_1 + m}{2m} \gamma_\rho \frac{\slashed{p}_1 + m}{2 m} \gamma^\mu \frac{\slashed{p'}_1 + m}{2m} \gamma^\rho \frac{\slashed{p}_1 + m}{2 m}] \\
            = \mathbf{\frac{1}{m^4} [-2 (p_1 \cdot p_2)^2 + 4 m^2 p_1 \cdot p_2]}
        \end{eqnarray}
    \end{framed}

    Now that all of the traces have been computed, the results can be inserted into the equation for the square of the Feynman amplitude:

    \begin{framed}
        \begin{eqnarray}
            (\frac{1}{2})^2 \sum_{S_f S_i} |M_{f i}|^2 = Tr \Bigg[ \gamma_{\mu} \frac{\slashed{p} + m}{2m} \gamma_{\rho} \frac{\slashed{p} + m}{2m} \Bigg] Tr \Bigg[ \gamma^{\mu} \frac{\slashed{p} + m}{2m} \gamma^{\rho} \frac{\slashed{p} + m}{2m} \Bigg]
        \end{eqnarray}
    \end{framed}

    \section*{Parametrization}

    Throughout this calculation so far, we have escaped picking a specific parametrization for the momentum vectors, all that was needed was to remember that we were taking the center of mass frame, and even then, this fact was only applied twice to simplify
    the scattering cross section formula at the end of the second section. To simplify the result any further however, a specific parametrization must be selected. Fig. 1 illustrates the center of mass frame in a manner that makes this task easier:

    % insert diagram here

    From Fig. 1, we can easily write down the standard parametrization: 

    \begin{equation}
        p_{1 \mu} = E \: 0 \: 0 \: |\mathbf{p}|
    \end{equation}
    
    \begin{equation}
        p_{2 \mu} = E \: 0 \: 0 \: - |\mathbf{p}|
    \end{equation}

    \begin{equation}
        p'_{1 \mu} = E \: 0 \: |\mathbf{p}| \sin \theta \: |\mathbf{p}| \cos \theta
    \end{equation}

    \begin{equation}
        p'_{2 \mu} = E \: 0 \: - |\mathbf{p}| \sin \theta \: - |\mathbf{p}| \cos \theta
    \end{equation}

    We can then evaluate all of the momentum variable dependent quantities in the squared Feynman amplitude, and the differential scattering cross section formula in terms of the selected parametrization. This gives the following:

    \begin{equation}
        |\vec{p}_1| = \sqrt{E^2 - m^2}
    \end{equation}

    \begin{equation}
        |\vec{p'}_1| = \sqrt{E^2 - m^2}
    \end{equation}

    \begin{equation}
        p_1 \cdot p_2 = 2 E^2 - m^2
    \end{equation}

    \begin{equation}
        p_1 \cdot p'_1 = E^2 (1 - \cos \theta) + m^2 \cos \theta
    \end{equation}

    \begin{equation}
        p_1 \cdot p'_1 = E^2 (1 - \cos \theta) - m^2 \cos \theta
    \end{equation}

    \begin{eqnarray}
        (p'_1 - p_1)^4 = (p'_1 - p_1) \cdot (p'_1 - p_1) (p'_1 - p_1) \cdot (p'_1 - p_1) = 4 (m^4 - 2 m^2 p_1 \cdot p'_1 + (p'_1 - p_1)^2) \\
        = 4 (m^2 - p'_1 - p_1)^2 = 4 (m^2 - E^2 (1 - \cos \theta) - m^2 \cos \theta)^2
    \end{eqnarray}

    \begin{equation}
        (p'_1 - p_1)^4 = 4 (m^4 - 2 m^2 p_1 \cdot p'_2 + (p_1 \cdot p'_2)^2) = 4 (m^2 - p_1 \cdot p'_2) = 4 (m^2 - E^2 (1 + \cos \theta) + m^2 \cos \theta)^2
    \end{equation}

    \begin{eqnarray}
        (p'_1 - p_1)^2 (p'_1 - p_1)^2 = 4 (m^4 - m^2 p_1 \cdot p'_2 - m^2 p_1 \cdot p'_2 + p_1 \cdot p'_2 p_1 \cdot p'_1) \\
        = 4 (m^4 - m^2 (E^2 (1 + \cos \theta) - m^2 \cos \theta) - m^2 (E^2 (1 + \cos \theta) + m^2 \cos \theta) ) \\
        + (E^2 (1 + \cos \theta) - m^2 \cos \theta) (E^2 (1 + \cos \theta) + m^2 \cos \theta) = 4 (E^2 - m^2)^2 \sin^2 \theta 
    \end{eqnarray}

    We can now plug all of these results into the differential scattering cross section formula, and the squared Feynman amplitude. Doing this produces the following results:

    \begin{equation}
        \frac{d \sigma_{CM}}{d \Omega} = \frac{m_1^2 m_2^2}{(2 \pi)^2 (E'_1 + E'_2)^2} \frac{|\vec{p'}_1|^2}{|\vec{p}_1|^2} (\frac{1}{2})^2 \sum_{S_i S_f} |M_{fi}|^2
    \end{equation}

    \begin{eqnarray}
        (\frac{1}{2})^2 \sum_{S_f S_i} |M_{f i}|^2 \\
        = \frac{e^4}{2 m^4}
    \end{eqnarray}

    \section*{Final Simplification to Yield Main Result}

    Figuring out how to simplify the Feynman amplitude square without help is rather hellish given the complicated dependence on trig functions. However, it is quite easy if someone tells you which algebraic steps to take and which trig identities
    to use. Here follows exactly that explanation. The key to getting this simplification is to manipulate the expression until one can make use of the following trig identities:

    \begin{framed}

        \begin{equation}
            \frac{1}{1 - \cos \theta} + \frac{1}{1 + \cos \theta} = \mathbf{\frac{2}{\sin^2 \theta}}
        \end{equation}

        \begin{equation}
            \frac{1}{(1 - \cos \theta)^2} + \frac{1}{(1 + \cos \theta)^2} = \mathbf{\frac{4}{\sin^4 \theta} - \frac{2}{\sin^2 \theta}}
        \end{equation}

        \begin{equation}
            (\frac{1}{(1 - \cos \theta)^2} + \frac{1}{(1 + \cos \theta)^2}) \cos^2 \theta = \mathbf{\frac{4}{\sin^4 \theta} - \frac{6}{\sin^2 \theta} + 2}
        \end{equation}
 
        \begin{equation}
            (\frac{1 - \cos \theta}{(1 - \cos \theta)^2} + \frac{1 + \cos \theta}{(1 + \cos \theta)^2}) \cos^2 \theta = \mathbf{- \frac{8}{\sin^4 \theta} + \frac{10}{\sin^2 \theta} - 2}
        \end{equation}

        \begin{equation}
            \frac{(1 + \cos \theta)^2}{(1 - \cos \theta)^2} + \frac{(1 - \cos \theta)^2}{(1 + \cos \theta)^2} = \mathbf{\frac{16}{\sin^4 \theta} - \frac{16}{\sin^2 \theta} + 2}
        \end{equation}

    \end{framed}

    These trig identities aren't too hard to prove, but proving them is probably longer than what they would typically give students in trigonometry class.

    The expression for the squared amplitude that we are starting with the one from above:

    \begin{eqnarray}
        (\frac{1}{2})^2 \sum_{S_f S_i} |M_{f i}|^2 \\
        = \frac{e^4}{2 m^4}
    \end{eqnarray}

    The first step in getting the squared amplitude into the form where these trig identities are applicable is to factor the first two denominators:

    \begin{eqnarray}
        (\frac{1}{2})^2 \sum_{S_f S_i} |M_{f i}|^2 \\
        = \frac{e^4}{2 m^4}
    \end{eqnarray}

    One can then factor out $4 (E^2 - m^2)$ from all of the denominators:

    \begin{eqnarray}
        (\frac{1}{2})^2 \sum_{S_f S_i} |M_{f i}|^2 \\
        = \frac{e^4}{2 m^4}
    \end{eqnarray}

    Then we can factor the last term in both of the first two numerators:

    \begin{eqnarray}
        (\frac{1}{2})^2 \sum_{S_f S_i} |M_{f i}|^2 \\
        = \frac{e^4}{2 m^4}
    \end{eqnarray}

    We can then split up the first two fraction into individual terms:

    \begin{eqnarray}
        (\frac{1}{2})^2 \sum_{S_f S_i} |M_{f i}|^2 \\
        = \frac{e^4}{2 m^4}
    \end{eqnarray}

    We can then bring the $\frac{(1 + \cos \theta)^2}{(1 - \cos \theta)^2} + \frac{(1 - \cos \theta)^2}{(1 + \cos \theta)^2}$ to thfe front, and also add and subtract $\mathbf{\frac{16}{\sin^4 \theta} - \frac{16}{\sin^2 \theta} + 2}$. The added
    copy can be combined with the last term, and the subtracted copy can be placed as the third term, giving us:

    \begin{eqnarray}
        (\frac{1}{2})^2 \sum_{S_f S_i} |M_{f i}|^2 \\
        = \frac{e^4}{2 m^4}
    \end{eqnarray}

    Inserting this gives:

    \begin{eqnarray}
        (\frac{1}{2})^2 \sum_{S_f S_i} |M_{f i}|^2 \\
        = \frac{e^4}{2 m^4}
    \end{eqnarray}

    We can then refactor everything in the large curvy parentheses based on the sine function power multiplying the term. This gives:

    \begin{eqnarray}
        (\frac{1}{2})^2 \sum_{S_f S_i} |M_{f i}|^2 \\
        = \frac{e^4}{2 m^4}
    \end{eqnarray}

    We can simplify the nunmerator of the second term in the large curvy parentheses, and we can also write the last term as a perfect square:

    \begin{eqnarray}
        (\frac{1}{2})^2 \sum_{S_f S_i} |M_{f i}|^2 \\
        = \frac{e^4}{2 m^4}
    \end{eqnarray}

    We can then write the first of those numerator of the next term to the right as follows:

    \begin{equation}
        -4 E^2 - 4E^2 m^2 + 5 m^4 = 8 (m^2 - E^2)^2 - 3(2 E^2 - m^2)^2
    \end{equation}

    Inserting that gives:

    \begin{eqnarray}
        (\frac{1}{2})^2 \sum_{S_f S_i} |M_{f i}|^2 \\
        = \frac{e^4}{2 m^4}
    \end{eqnarray}

    We can combine like terms throughout the whole expression:

    \begin{eqnarray}
        (\frac{1}{2})^2 \sum_{S_f S_i} |M_{f i}|^2 \\
        = \frac{e^4}{2 m^4}
    \end{eqnarray}

    Next, we can factor $2 (m^2 - E^2)^2$ out of the large parentheses to get:

    \begin{eqnarray}
        (\frac{1}{2})^2 \sum_{S_f S_i} |M_{f i}|^2 \\
        = \frac{e^4}{2 m^4}
    \end{eqnarray}

    Pulling out the last factor of two gives the final answer:

    \begin{eqnarray}
        (\frac{1}{2})^2 \sum_{S_f S_i} |M_{f i}|^2 \\
        = \frac{e^4}{2 m^4}
    \end{eqnarray}

    Now, recall that the differential cross section in terms of the Feynman amplitude worked out to be:

    \begin{equation}
        \frac{d \sigma_{CM}}{d \Omega} = \frac{m^4 \alpha^2}{E^2} \bigg( \frac{1}{2} \bigg)^2 \sum_{S_i, S_f} \frac{|M_{fi}|}{e^4}
    \end{equation}

    Plugging in the result for the average of the square of the Feynman amplitude gives the Moller formula:

    \begin{framed}
        \begin{equation}
            \mathbf{\frac{d \sigma}{d \Omega} = }
        \end{equation}
    \end{framed}

    \section*{Ultra-Relativistic Limit}

    From here, the ultra-relativistic limit is considered, then the non relativistic limit. In the ultra-relativistic limit, the mass becomes negligible, meaning the ultra-relativistic, or high energy limit is merely the limit as the electron
    mass goes to zero. Taking this limit on the Moller scattering cross section given in the box gives:

    \begin{equation}
        \frac{d \sigma_{CM}}{d \Omega} = \frac{m^4 \alpha^2}{E^2} \bigg( \frac{1}{2} \bigg)^2 \sum_{S_i, S_f} \frac{|M_{fi}|}{e^4}
    \end{equation}

    \begin{equation}
        \frac{d \sigma_{CM}}{d \Omega} = \frac{m^4 \alpha^2}{E^2} \bigg( \frac{1}{2} \bigg)^2 \sum_{S_i, S_f} \frac{|M_{fi}|}{e^4}
    \end{equation}

    \begin{framed}
        \begin{equation}
            \mathbf{\frac{d \sigma}{d \Omega} = }
        \end{equation}
    \end{framed}

    Lastly, we can study the non-relativistic limit.
        
    \section*{Nonrelativistic Limit}

    Of course, in this limit the denominator of the prefactor goes to zero, and the prefactor diverges. However, this isn't actually that relevant to taking the limit because it multiplies the whole expression. We only care to figure out which
    terms become insignificant \textit{relative to others as E approaches m}. No prefactor (even a divergent one) affects \textit{relative} significance of terms. With this in mind, we can see that the last term in the general formula get very
    small compared to the first to, wo we have the following result:

    \begin{equation}
        \frac{d \sigma_{CM}}{d \Omega} = \frac{m^4 \alpha^2}{E^2} \bigg( \frac{1}{2} \bigg)^2 \sum_{S_i, S_f} \frac{|M_{fi}|}{e^4}
    \end{equation}

    \begin{equation}
        \frac{d \sigma_{CM}}{d \Omega} = \frac{m^4 \alpha^2}{E^2} \bigg( \frac{1}{2} \bigg)^2 \sum_{S_i, S_f} \frac{|M_{fi}|}{e^4}
    \end{equation}

    In the non-relativistic limit, we also have $p = mv$. Therefore, the Moller scattering cross section in the non-relativistic limit is the following:

    \begin{framed}
        \begin{equation}
            \mathbf{\frac{d \sigma}{d \Omega} = }
        \end{equation}
    \end{framed}

\end{document}